\documentclass{ximera}

\title{Central projection in \textbf{HG}}

\begin{document}
\begin{abstract}
Here we explore the hyperbolic geometry we find with central projection. 
\end{abstract}
\maketitle


\subsection*{The edge of the universe}

Again \textbf{HG} is a $K$-geometry in the sense of Part \ref{III} since, in
$\left(  x,y,z\right)  $-coordinates, the equation for the $K$-geometry
\begin{equation}
K\left(  x^{2}+y^{2}\right)  +z^{2}=1 \label{74}%
\end{equation}
with%
\[
K<0
\]
and the $K$-dot product. So all the calculations in Part \ref{III} hold, in
particular $\left(  \ref{75}\right)  $. So the $\left(  x_{c},y_{c}\right)
$-coordinates that parametrize the entire $K$-geometry are the $\left(
x_{c},y_{c}\right)  $ such that%
\[
x_{c}^{2}+y_{c}^{2}<\frac{1}{\left\vert K\right\vert }.
\]
So we will call the circle%
\[
x_{c}^{2}+y_{c}^{2}=\frac{1}{\left\vert K\right\vert }%
\]
the \textit{edge of the universe}. (The $\left(  x_{c},y_{c}\right)
$-coordinates are called Klein coordinates for hyperbolic geometry and the
disk of radius $\left\vert K\right\vert ^{-1/2}$ is called the \textit{Klein
model} for \textbf{HG} in honor of the famous German geometer, Felix
Klein.)

\subsection*{Lines go to chords}

Again all the calculations in Part \ref{III} hold, in particular Exercise
\ref{73}a). We conclude that lines in \textbf{HG} correspond to chords on the
Klein $\left(  x_{c},y_{c}\right)  $-disk that connect two points on the edge
of the universe.lines in the $\left(  x_{c},y_{c}\right)  $-disk.

\begin{exercise}\hfil
\begin{enumerate}
\item Explain why the $K$-line $y=0$ is given by the $x_{c}$-axis
and the North Pole $N$ is given by $\left(  x_{c},y_{c}\right)  =\left(
0,0\right)  $.
\item Explain why the point $\left(  \left\vert K\right\vert ^{-1/2}%
\cdot \sinh \varepsilon,0,\cosh %
\varepsilon\right)  $ in $K$-geometry is given by the point%
\[
\left(  x_{c},y_{c}\right)  =\left(  \left\vert K\right\vert ^{-1/2}%
\cdot \frac{e^{\varepsilon}-e^{-\varepsilon}%
}{e^{\varepsilon}+e^{-\varepsilon}},0\right)  .
\]


\item Explain why the $K$-distance between $\left(  x_{c},y_{c}\right)  =\left(
0,0\right)  $ and $\left(  x_{c},y_{c}\right)  =\left(  \left\vert
K\right\vert ^{-1/2}\cdot \frac{e^{\varepsilon
}-e^{-\varepsilon}}{e^{\varepsilon}+e^{-\varepsilon}},0\right)  $ is
$\left\vert K\right\vert ^{-1/2}\cdot\varepsilon$.
\end{enumerate}
\end{exercise}

\begin{exercise}
 Explain why lines in \textbf{HG} extend infinitely in each direction.

Hint: There is a $K$-rigid motion that takes any two points to $\left(
0,0\right)  $ and $\left(  \left\vert K\right\vert ^{-1/2}%
\cdot \frac{e^{\varepsilon}-e^{-\varepsilon}%
}{e^{\varepsilon}+e^{-\varepsilon}},0\right)  $ for some $\varepsilon
>0$.
\end{exercise}

\subsection*{$K$-perpendicularity in the Klein model for \textbf{HG}}

Suppose we are given any three distinct points $P^{\prime}$, $R^{\prime}$ and
$Q^{\prime}$ on the edge of the universe of the Klein $K$-disk. We construct
the line $L^{\prime}$ through $P^{\prime}$ and $Q^{\prime}$ and mark a point
$A^{\prime}$ on it as shown in the figure below.
\[%
%TCIMACRO{\FRAME{itbpF}{1.1225in}{0.9954in}{0pt}{}{}{Figure}%
%{\special{ language "Scientific Word";  type "GRAPHIC";
%maintain-aspect-ratio TRUE;  display "USEDEF";  valid_file "T";
%width 1.1225in;  height 0.9954in;  depth 0pt;  original-width 8.1759in;
%original-height 7.2523in;  cropleft "0";  croptop "1";  cropright "1";
%cropbottom "0";  tempfilename 'MXAJC00O.jpg';tempfile-properties "XPR";}}}%
%BeginExpansion
\raisebox{-0pt}{\includegraphics[
natheight=7.252300in,
natwidth=8.175900in,
height=0.9954in,
width=1.1225in
]%
{MXAJC00O.jpg}%
}%
%EndExpansion
\]


We know that there is a $K$-rigid motion $M_{c}$ that takes $A^{\prime}$ to
$\left(  0,0\right)  $ and $L^{\prime}$ to the $x_{c}$-axis. (Why?) Viewed as
a transformation
\begin{gather*}
\left(  \underline{x_{c}},\underline{y_{c}}\right)  =M_{c}\left(  x_{c}%
,y_{c}\right) \\
=\left(  \frac{m_{11}x_{c}+m_{21}y_{c}+m_{31}}{m_{13}x_{c}+m_{23}y_{c}+m_{33}%
},\frac{m_{12}x_{c}+m_{22}y_{c}+m_{32}}{m_{13}x_{c}+m_{23}y_{c}+m_{33}%
}\right)  .
\end{gather*}
of the entire $\left(  x_{c},y_{c}\right)  $-plane, this transformation takes
the tangent line to the edge of the universe at $P^{\prime}$ to the tangent
line $T_{-}$ to the edge of the universe at $\left(  -\left\vert K\right\vert
^{-1/2},0\right)  $ and the tangent line to the edge of the universe at
$Q^{\prime}$ to the tangent line $T_{+}$ to the edge of the universe at
$\left(  \left\vert K\right\vert ^{-1/2},0\right)  $. Since the tangent lines
at $\left(  -\left\vert K\right\vert ^{-1/2},0\right)  $ and $\left(
\left\vert K\right\vert ^{-1/2},0\right)  $ are vertical, the point
$S^{\prime}$must have gone to infinity under the $K$-rigid motion. So the line
through $A^{\prime}$ and $R^{\prime}$ must go to a line that goes through
$\left(  0,0\right)  $ and that lies between $T_{-}$ and $T_{+}$. But here is
only one such line, namely the $y_{c}$-axis.

\begin{exercise}
\label{99}Explain why there is a $K$-rigid motion $M_{c}$ that takes any three
points $P^{\prime}$, $R^{\prime}$ and $Q^{\prime}$ in order along the edge of
the universe to any other three points $P^{\prime\prime}$, $R^{\prime\prime}$
and $Q^{\prime\prime}$ in order along the edge of the universe.

Hint: Use that the set of $K$-rigid motions form a group under the composition operation.
\end{exercise}

\begin{exercise}
 Explain why the above discussion implies that the angles $\angle
P^{\prime}A^{\prime}R^{\prime}$ and $\angle Q^{\prime}A^{\prime}R^{\prime}$
must both be $K$-right angles, that is, their $K$-measures must each be
$90^{\circ}$. So the line segments $\overline{P^{\prime}Q^{\prime}}$ and
$\overline{A^{\prime}R^{\prime}}$ are $K$-perpendicular. [MJG,238-239]

Hint: You may need to use the fact that, since there is a $K$-rigid motion
that interchanges $\left(  -\left\vert K\right\vert ^{-1/2},0\right)  $ and
$\left(  \left\vert K\right\vert ^{-1/2},0\right)  $ and leaves $\left(
0,0\right)  $ fixed, the $x_{c}$-axis and the $y_{c}$-axis are $K$-perpendicular.
\end{exercise}

\begin{exercise}
 Use the previous Exercise and the fact that $A^{\prime}$ can be
any point along the chord $\overline{P^{\prime}Q^{\prime}}$ in the figure
above to explain why the Klein model is not conformal, that is, it does not
faithfully represent the measure of angles in \textbf{HG}.
\end{exercise}

\subsection*{Quadrilaterals in HG, in fact, in any NG}

\begin{exercise}
 Use $\left( x_{c},y_{c}\right) $-coordinates to show that \textbf{HG}
 satisfies the four Euclidean postulates E1, E2, E3, and E4.Thus
 hyperbolic geometry is a Neutral Geometry.
\end{exercise}

The next Exercise will help us appreciate some important properties of
\textbf{HG} that are properties of any Neutral Geometry. So do the Exercise
assuming that the context is any Neutral Geometry, that is, any
two-dimensional geometry satisfying E1-E4.

\begin{exercise}
Let $ABCD$ be a quadrilateral with $\angle ABC=\angle BCD$
right angles. (We denote polygons by naming their vertices in counterclockwise
order.) [MJG,164-165]
\begin{enumerate}
\item Show in \textbf{NG} that

$\left\vert AB\right\vert =\left\vert CD\right\vert $ implies that $\angle
BAD=\angle ADC$,

$\left\vert AB\right\vert >\left\vert CD\right\vert $ implies that $\angle
BAD<\angle ADC$,

$\left\vert AB\right\vert <\left\vert CD\right\vert $ implies that $\angle
BAD>\angle ADC$.

\item Use a) and pure logic to show that

$\angle BAD<\angle ADC$ implies that $\left\vert AB\right\vert >\left\vert
CD\right\vert $,

$\angle BAD=\angle ADC$ implies that $\left\vert AB\right\vert =\left\vert
CD\right\vert $,

$\angle BAD>\angle ADC$ implies that $\left\vert AB\right\vert <\left\vert
CD\right\vert $.

Hint: For the first implication in a) show that quadrilateral $ABCD$ is
(self-)congruent to the quadrilateral $DCBA$. Now suppose that the second
implication in a) is false for some quadrilateral $ABCD$. Construct
$A^{\prime}$ on $\overline{AB}$ so that $\left\vert A^{\prime}B\right\vert
=\left\vert CD\right\vert $. $B$y Exercise \ref{18}%
\[
\angle BAD<\angle BA^{\prime}D.
\]
By the (already proved) first implication%
\[
\angle BA^{\prime}D=\angle CDA^{\prime}.
\]
Finally%
\[
\angle A^{\prime}DC<\angle ADC
\]
since the segment $DA^{\prime}$ lies between the segment $DA$ and the segment
$DC$. The proof of the third implication in a) is the same as the proof of
the second implication--just interchange $A$ and $D$ and interchange $B$ and
$C$.

For b), just use pure logic: If the first implication is false, then $\angle
BAD<\angle ADC$ and either $\left\vert AB\right\vert <\left\vert CD\right\vert
$ or $\left\vert AB\right\vert =\left\vert CD\right\vert $. If $\left\vert
AB\right\vert <\left\vert CD\right\vert $, then by a)%
\[
\angle BAD>\angle ADC.
\]
Contradiction! Etc., etc.
\end{enumerate}
\end{exercise}

\begin{exercise}
Use $\left(  x_{c},y_{c}\right)  $-coordinates to show that \textbf{HG} does
not satisfy Euclid's postulate E5.That is, through a point not on a line, it
is not true that there passes a unique parallel (i.e. non-intersecting) line.
\end{exercise}



\subsection*{Computing $K$-distances in Klein coordinates}

In fact the tool that will let us compute all $K$-distances in $\left(
x_{c},y_{c}\right)  $-coordinates is the cross-ratio from Definition \ref{44}.
Let $d_{K}\left(  A_{c},B_{c}\right)  $ denote the $K$-distance between two
points $A_{c}$ and $B_{c}$ in the Klein $K$-disk. Now we know that
\[
d_{K}\left(  \left(  0,0\right)  ,\left(  \left\vert K\right\vert
^{-1/2}\cdot \frac{e^{\varepsilon}-e^{-\varepsilon}%
}{e^{\varepsilon}+e^{-\varepsilon}},0\right)  \right)  =\left\vert
K\right\vert ^{-1/2}\cdot \varepsilon.
\]
To see what this has to do with cross-ratio, we begin by computing the cross
ratio%
\[
\left(  0:-\left\vert K\right\vert ^{-1/2}:\left\vert K\right\vert
^{-1/2}\cdot \frac{e^{\varepsilon}-e^{-\varepsilon}%
}{e^{\varepsilon}+e^{-\varepsilon}}:\left\vert K\right\vert ^{-1/2}\right)
\]
given by the two points $\left(  0,0\right)  $, $\left(  \left\vert
K\right\vert ^{-1/2}\cdot \frac{e^{\varepsilon
}-e^{-\varepsilon}}{e^{\varepsilon}+e^{-\varepsilon}},0\right)  $ and the two
points $\left(  -\left\vert K\right\vert ^{-1/2},0\right)  $ and $\left(
\left\vert K\right\vert ^{-1/2},0\right)  $ where the $x_{c}$-axis intersects
the edge of the universe.

\begin{exercise}\hfil
\begin{enumerate}
\item Draw a picture of the Klein $K$-disk, the edge of the universe,
  and the four points on the $x_{c}$-axis.
\item Show that%
\[
\left(  0:-\left\vert K\right\vert ^{-1/2}:\left\vert K\right\vert
^{-1/2}\cdot \frac{e^{\varepsilon}-e^{-\varepsilon}%
}{e^{\varepsilon}+e^{-\varepsilon}}:\left\vert K\right\vert ^{-1/2}\right)
=\left(  0:-1:\cdot \frac{e^{\varepsilon}-e^{-\varepsilon
}}{e^{\varepsilon}+e^{-\varepsilon}}:1\right)  .
\]
In particular, notice that the computation doesn't depend on $K$.

\item Show that%
\[
\left(  0:-1:\cdot \frac{e^{\varepsilon}-e^{-\varepsilon}%
}{e^{\varepsilon}+e^{-\varepsilon}}:1\right)  =e^{-2\varepsilon}.
\]
\end{enumerate}
\end{exercise}

From these two Exercises we conclude that%
\begin{equation}
d_{K}\left(  \left(  0,0\right)  ,\left(  \left\vert K\right\vert
^{-1/2}\cdot \frac{e^{\varepsilon}-e^{-\varepsilon}%
}{e^{\varepsilon}+e^{-\varepsilon}},0\right)  \right)  =\frac{\left\vert
K\right\vert ^{-1/2}}{2}\cdot \left\vert \ln %
\left(  0:-\left\vert K\right\vert ^{-1/2}:\left\vert K\right\vert
^{-1/2}\cdot \frac{e^{\varepsilon}-e^{-\varepsilon}%
}{e^{\varepsilon}+e^{-\varepsilon}}:\left\vert K\right\vert ^{-1/2}\right)
\right\vert . \label{78}%
\end{equation}


Now suppose we are given any two $A_{c}$ and $B_{c}$ in the Klein $K$-disk.
They determine a line
\begin{equation}
\alpha x_{c}+\beta y_{c}+\gamma=0 \label{76}%
\end{equation}
and so points $P_{c}$ and $Q_{c}$ where that line meets the edge of the
universe as shown in the figure below.
\[%
%TCIMACRO{\FRAME{itbpF}{1.6942in}{1.5748in}{0in}{}{}{Figure}%
%{\special{ language "Scientific Word";  type "GRAPHIC";
%maintain-aspect-ratio TRUE;  display "USEDEF";  valid_file "T";
%width 1.6942in;  height 1.5748in;  depth 0in;  original-width 9.89in;
%original-height 9.1869in;  cropleft "0";  croptop "1";  cropright "1";
%cropbottom "0";  tempfilename 'MXAJC00P.jpg';tempfile-properties "XPR";}}}%
%BeginExpansion
{\includegraphics[
natheight=9.186900in,
natwidth=9.890000in,
height=1.5748in,
width=1.6942in
]%
{MXAJC00P.jpg}%
}%
%EndExpansion
\]
We are now ready to prove the following Theorem.

\begin{theorem}
 For any two points $A_{c}$ and $B_{c}$ on the Klein $K$-disk,
the $K$-distance between them $d_{K}\left(  A_{c},B_{c}\right)  $ is given by
the formula%
\[
d_{K}\left(  A_{c},B_{c}\right)  =\frac{\left\vert K\right\vert ^{-1/2}}%
{2}\cdot \left\vert \ln \left(  x_{c}\left(
A_{c}\right)  :x_{c}\left(  P_{c}\right)  :x_{c}\left(  B_{c}\right)
:x_{c}\left(  Q_{c}\right)  \right)  \right\vert
\]
where $P_{c}$ and $Q_{c}$ are the endpoints of the chord through $A_{c}$ and
$B_{c}$. (Compare with [MJG,268].)
\end{theorem}

\begin{proof}
We know that there is a $K$-rigid motion $\left(  \underline{x_{c}}%
,\underline{y_{c}}\right)  =M_{c}\left(  x_{c},y_{c}\right)  $ of the Klein
disk that takes $A_{c}$ to $\left(  0,0\right)  $ and $B_{c}$ to some point
$\left(  \left\vert K\right\vert ^{-1/2}\cdot %
\frac{e^{\varepsilon}-e^{-\varepsilon}}{e^{\varepsilon}+e^{-\varepsilon}%
},0\right)  $ on the positive $x_{c}$-axis. From $\left(  \ref{77}\right)  $
we know that%
\[
\underline{x_{c}}=\frac{m_{11}x_{c}+m_{21}y_{c}+m_{31}}{m_{13}x_{c}%
+m_{23}y_{c}+m_{33}}.
\]
But from $\left(  \ref{76}\right)  $ we know that for our four points $A_{c}$,
$B_{c}$, $P_{c}$, and $Q_{c}$
\begin{gather*}
\alpha x_{c}+\beta y_{c}+\gamma=0\\
y_{c}=\frac{-\alpha x_{c}-\gamma}{\beta}.
\end{gather*}
So if we calculate $M_{c}$ only for these four points we have%
\begin{align*}
\underline{x_{c}}  &  =\frac{m_{11}x_{c}+m_{21}\left(  \frac{-\alpha
x_{c}-\gamma}{\beta}\right)  +m_{31}}{m_{13}x_{c}+m_{23}\left(  \frac{-\alpha
x_{c}-\gamma}{\beta}\right)  +m_{33}}\\
&  =\frac{\left(  m_{11}-\frac{m_{21}\alpha}{\beta}\right)  x_{c}+\left(
m_{31}-\frac{m_{21}\gamma}{\beta}\right)  }{\left(  m_{13}-\frac{m_{23}\alpha
}{\beta}\right)  x_{c}+\left(  m_{33}-\frac{m_{23}\gamma}{\beta}\right)  .}%
\end{align*}
That is, the function $x_{c}\mapsto\underline{x_{c}}$ is a linear fractional
transformation! So by Exercise \ref{42}%
\begin{align*}
\left(  x_{c}\left(  A_{c}\right)  :x_{c}\left(  P_{c}\right)  :x_{c}\left(
B_{c}\right)  :x_{c}\left(  Q_{c}\right)  \right)   &  =\left(  \underline
{x_{c}}\left(  A_{c}\right)  :\underline{x_{c}}\left(  P_{c}\right)
:\underline{x_{c}}\left(  B_{c}\right)  :\underline{x_{c}}\left(
Q_{c}\right)  \right) \\
&  =\left(  0:-\left\vert K\right\vert ^{-1/2}:\left\vert K\right\vert
^{-1/2}\cdot \frac{e^{\varepsilon}-e^{-\varepsilon}%
}{e^{\varepsilon}+e^{-\varepsilon}}:\left\vert K\right\vert ^{-1/2}\right) \\
&  =e^{-2\varepsilon}.
\end{align*}
Therefore%
\begin{align*}
d_{K}\left(  A_{c},B_{c}\right)   &  =d_{K}\left(  \left(  0,0\right)
,\left(  \left\vert K\right\vert ^{-1/2}\cdot %
\frac{e^{\varepsilon}-e^{-\varepsilon}}{e^{\varepsilon}+e^{-\varepsilon}%
},0\right)  \right) \\
&  =\frac{\left\vert K\right\vert ^{-1/2}}{2}\cdot \left\vert \ln \left(  0:-\left\vert K\right\vert ^{-1/2}:\left\vert
K\right\vert ^{-1/2}\cdot \frac{e^{\varepsilon
}-e^{-\varepsilon}}{e^{\varepsilon}+e^{-\varepsilon}}:\left\vert K\right\vert
^{-1/2}\right)  \right\vert \\
&  =\frac{\left\vert K\right\vert ^{-1/2}}{2}\cdot \left\vert \ln \left(  x_{c}\left(  A_{c}\right)  :x_{c}\left(
P_{c}\right)  :x_{c}\left(  B_{c}\right)  :x_{c}\left(  Q_{c}\right)  \right)
\right\vert .
\end{align*}

\end{proof}

\begin{exercise}
For $K=-1$, calculate the $K$-distance between the two points given in
$\left(  x_{c},y_{c}\right)  $-coordinates by $\left(  0,0\right)  $ and
$\left(  1/2,0\right)  $
\end{exercise}

\subsection*{Areas of hyperbolic lunes}

Finally there is one $K$-area computation that it is convenient to do in Klein
coordinates. Namely suppose that we take the ordinary triangle $T_{c}$ in the
$\left(  x_{c},y_{c}\right)  $-plane with vertices $\left(  0,0\right)  $,
$\left(  \left\vert K\right\vert ^{-1/2}\cos \beta,\left\vert
K\right\vert ^{-1/2}\sin \beta\right)  $ and $\left(  \left\vert
K\right\vert ^{-1/2}\cos \beta,-\left\vert K\right\vert ^{-1/2}%
\sin \beta\right)  $. Notice that two of the three vertices lie on the
edge of the universe of the Klein $K$-disk and that the $K$-angle at the third
vertex is%
\[
\alpha=2\beta.
\]
(In fact $\left(  0,0\right)  $ is the \textit{one} point in the Klein
$K$-disk where $K$-angels \textit{are} faithfully represented. (Why?)) We will
call the interior of this triangle, or any $K$-rigid motion of it, an $\alpha
$\textbf{-lune}. So we wish to compute the $K$-area of an $\alpha$-lune. Since
\textbf{HG} is a $K$-geometry we know from Exercise \ref{79} that this area
$A_{K}\left(  \alpha\right)  $ is given by the formula%
\[
A_{K}\left(  \alpha\right)  =%
%TCIMACRO{\dint \nolimits_{T_{c}}}%
%BeginExpansion
{\displaystyle\int\nolimits_{T_{c}}}
%EndExpansion
\frac{1}{\left(  1-\left\vert K\right\vert \left(  x_{c}^{2}+y_{c}^{2}\right)
\right)  ^{3/2}}dx_{c}dy_{c}.
\]


\begin{exercise}
\label{97}Show that%
\[
A_{K}\left(  \alpha\right)  =\left\vert K\right\vert ^{-1}\left(  \pi
-\alpha\right)  .
\]


Hint: Use the substitution%
\begin{align*}
\underline{x_{c}}  &  =\left\vert K\right\vert ^{1/2}x_{c}\\
\underline{y_{c}}  &  =\left\vert K\right\vert ^{1/2}y_{c}%
\end{align*}
to reduce the computation to the computation in the case that $\left\vert
K\right\vert =1$. Then use polar coordinates to get
\[
A_{K}\left(  \alpha\right)  =\left\vert K\right\vert ^{-1}%
%TCIMACRO{\dint \nolimits_{\theta=-\beta}^{\theta=\beta}}%
%BeginExpansion
{\displaystyle\int\nolimits_{\theta=-\beta}^{\theta=\beta}}
%EndExpansion%
%TCIMACRO{\dint \nolimits_{r=0}^{r=\frac{\cos \beta}{\cos \theta
%}}}%
%BeginExpansion
{\displaystyle\int\nolimits_{r=0}^{r=\frac{\cos \beta}{\cos %
\theta}}}
%EndExpansion
\frac{1}{\left(  1-r^{2}\right)  ^{3/2}}rdrd\theta
\]
Then do the substitution%
\begin{align*}
u  &  =1-r^{2}\\
du  &  =-2rdr
\end{align*}
to compute
\[%
%TCIMACRO{\dint \nolimits_{r=0}^{r=\frac{\cos \beta}{\cos \theta
%}}}%
%BeginExpansion
{\displaystyle\int\nolimits_{r=0}^{r=\frac{\cos \beta}{\cos %
\theta}}}
%EndExpansion
\frac{1}{\left(  1-r^{2}\right)  ^{3/2}}rdr.
\]
In the final step use the substitution%
\[
t=\sin \left(  \theta\right)  .
\]

\end{exercise}





\end{document}
