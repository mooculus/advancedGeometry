\documentclass[newpage,hints,handout,12pt,noauthor,nooutcomes]{ximera}

%\usepackage{microtype}
%\usepackage{tikz}
\usepackage{tkz-euclide}
%\usetkzobj{all}
\tikzstyle geometryDiagrams=[rounded corners=.5pt,ultra thick,color=blue!50!black]

\usepackage{tikz-cd}

\colorlet{penColor}{blue!50!black} % Color of a curve in a plot

%% \hypersetup{
%%     colorlinks = false,
%%     }


\tikzset{%% partial ellipse
    partial ellipse/.style args={#1:#2:#3}{
        insert path={+ (#1:#3) arc (#1:#2:#3)}
    }
}

\graphicspath{
{./}
{sphericalLunesAndTriangles/}
{hyperbolicLunesAndTriangles/}
{centralProjection/}
{stereographicProjection/}
{linesAnglesAndAreasInCentralProjection/}
{linesAnglesAndAreasInStereographicProjection/}
{stereographicProjection/}
{centralProjectionInHG/}
{stereographicProjectionInHG/}
{linesInSphericalGeometry/}
{linesInHyperbolicGeometry/}
{theArtOfEscher/}
}


\newcommand{\transpose}{\intercal}
\newcommand{\eval}[1]{\bigg[ #1 \bigg]}

\renewcommand{\epsilon}{\varepsilon}
\renewcommand{\l}{\ell}
\renewcommand{\d}{\,d}

\DeclareMathOperator{\arccosh}{arccosh}
\DeclareMathOperator{\arctanh}{arctanh}
\renewcommand{\tilde}{\widetilde}
\newcommand{\R}{\mathbb R}
\newcommand{\dd}[2][]{\frac{d #1}{d #2}}
\newcommand{\pp}[2][]{\frac{\partial #1}{\partial #2}}
\newcommand{\dfn}{\textbf}

\renewcommand{\bar}{\overline}
\renewcommand{\hat}{\widehat}


\ifxake
\NewEnviron{freeResponse}{}
\fi


\title{Dilations and similarity in euclidean geometry}

\begin{document}
\begin{abstract}
Now we explore some properties that are somewhat unique to euclidean
geometry.
\end{abstract}
\maketitle

\subsection{Dilations}

We start this section off by asking you to prove a basic result.

\begin{problem}
  Let $X$ be a set and $f: X \to X$. Prove that $f^{-1}$ exists if and
  only if $f$ is one-to-one and onto.
  \begin{freeResponse}
    $(\Rightarrow)$ Suppose $f^{-1}:A\to A$ exists. Since the domain
    of $f^{-1}$ is all of $A$, $f$ must be onto. Moreover, since
    $f^{-1}$ is a function, this means $f$ must be onto. 
    
    $(\Leftarrow)$ Suppose $f$ is one-to-one and onto, so we may simply
    ``reverse the arrows'' that define $f$ and obtain a function.
  \end{freeResponse}
\end{problem}


\begin{definition}
A \textbf{dilation} is a transformation of the cartesian plane to
itself that:
\begin{enumerate}
\item Is one-to-one and onto.
\item Fixes one point called the \textbf{center} of the dilation.
\item Takes each line through the center of the dilation to itself.
\item Multiplies all distances by a fixed positive real number called
  the \textbf{magnification factor} of the dilation.
\end{enumerate}
\end{definition}

\begin{definition}
Given a point $\left(x_{0},y_{0}\right)$ in the plane and a positive
real number $r$, we define a mapping $D$ with center $\left(
x_{0},y_{0}\right)$ and magnification factor $r$ by the formula
\[
D\left(  x,y\right)  =\left(  x_{0},y_{0}\right)  +r\left(  x-x_{0}%
,y-y_{0}\right).
\]
We will also denote the output $D\left(x,y\right)$ of the dilation
as $\left(\underline{x},\underline{y}\right)$.
\end{definition}

\begin{problem}
Using cartesian coordinates for the plane, show that the mapping $D$
defined above is a dilation with magnification factor $r$ and center
$\left( x_{0},y_{0}\right) $.
\begin{hint}
  Use the parametric formula for a line:
  \[
  \l(t) = \text{point} + t \cdot \text{vector}
  \]
\end{hint}
\begin{freeResponse}
First we must show that the center of the dilation is fixed, that is we must show that 
\[
D(x_0,y_0) = (x_0,y_0).
\]
Write
\begin{align*}
D(x_0,y_0) &= (x_0,y_0) + r(x_0-x_0,y_0-y_0)\\
&= (x_0,y_0) + r(0,0)\\
&=(x_0,y_0).
\end{align*}
Next we must show that every line through the fixed point goes to
itself. To do this, consider the line
\[
\l(t) = (x_0,y_0) + t(u,v)
\]
that goes through the center of the dilation in the direction of the
vector $(u,v)$. Write
\begin{align*}
D(\l(t)) &= D(x_0+tu,y_0+tv)\\
&= (x_0,y_0) + r(x_0+tu-x_0,y_0+tv-y_0)\\
&= (x_0,y_0) + r(tu,tv)\\
&= (x_0,y_0) + t(ru,rv). 
\end{align*}
This is the line that passes through the center of the dilation in the
direction of $(ru,rv)$, but since $r$ is a positive real number, the
vector $(u,v)$ goes in the same direction as the vector
$(ru,rv)$. Hence a dilation maps any line that passes through its
center to itself.

Finally, we must show that all distances are multiplied by the
magnification factor of the dilation. Consider two points $(a_1,b_1)$
and $(a_2,b_2)$.  Write
\begin{align*}
D(a_1,b_1) &= (x_0,y_0)+r(a_1-x_0,b_1-y_0)\\
&=(x_0+ra_1-rx_0,y_0+rb_1-ry_0)\\
&=(\underline{a_1},\underline{b_1}).
\end{align*}
and
\begin{align*}
D(a_2,b_2) &= (x_0,y_0)+r(a_2-x_0,b_2-y_0)\\
&=(x_0+ra_2-rx_0,y_0+rb_2-ry_0)\\
&=(\underline{a_2},\underline{b_2}).
\end{align*}
Now 
\begin{align*}
d((\underline{a_1},\underline{b_1}),(\underline{a_2},\underline{b_2})) &=
\sqrt{(\underline{a_2}-\underline{a_1})^2+(\underline{b_2}-\underline{b_1})^2} \\
&=\sqrt{(x_0+ra_2-rx_0-x_0-ra_1+rx_0)^2+(y_0+rb_2-ry_0-y_0-rb_1+ry_0)^2} \\
&=\sqrt{(ra_2-ra_1)^2+(rb_2-rb_1)^2} \\
&=\sqrt{r^2(a_2-a_1)^2+r^2(b_2-b_1)^2} \\
&=r\sqrt{(a_2-a_1)^2+(b_2-b_1)^2} \\
&=r\cdot d((a_1,b_1),(a_2,b_2)).
\end{align*}
Hence we see that the distance between any two points is multiplied by
the magnification factor.
\end{freeResponse}
\end{problem}

\begin{problem}
Show that a dilation by a factor of $r$ takes any vector to $r$
times itself.
\begin{hint}
View the vector as the difference between two points.
\end{hint}
\begin{freeResponse}
Consider the vector with its tip at $(a,b)$ and its tail at
$(c,d)$. This is vector $(a-c, b-d)$. Now if apply the dilation,
\begin{align*}
D(a,b) &= (x_0,y_0) + r(a-x_0, b-y_0)\\
&= (ra + (1-r)x_0, rb+(1-r)y_0).
\end{align*}
Likewise, $D(c,d) = (rc + (1-r)x_0, rd+(1-r)y_0)$. This is vector
\[
(ra-rc,rb-rd) = r(a-c,b-d).
\]
Hence a dilation takes any vector to $r$ times itself.
\end{freeResponse}
\end{problem}

\begin{problem}
Show that a dilation takes a line to a line parallel (or equal) to
itself.
\begin{hint}
Use the parametric formula for a line.
\end{hint}
\begin{freeResponse}
Using the parametric formula for a line
\[
\l(t) = (a,b) + t(u,v)
\]
we see that
\begin{align*}
D(\l(t)) &=D(a+tu,b+tv)\\
&= (x_0,y_0) + r(a+tu-x_0,b+tv-y_0)\\
&= (x_0,y_0) + (ra - rx_0,rb-ry_0)+ r(tu,tv)\\
&= (x_0(1-r) + ra,y_0(1-r)+rb)) + t(ru,rv). 
\end{align*}
This is the line that passes through $(x_0(1-r) + ra,y_0(1-r)+rb))$ in
the direction of $(ru,rv)$, but since $r$ is a positive real number,
the vector $(u,v)$ goes in the same direction as the vector
$(ru,rv)$. Hence a dilation maps any line to a line parallel (or
equal) to itself.
\end{freeResponse}
\end{problem}

\begin{problem}
Show that a dilation of the plane preserves angles.
\begin{freeResponse}
Since an angle can be determined by two vectors, and dilations do not
change the direction of vectors, dilations of the plane preserve
angles.
\end{freeResponse}
\end{problem}


\begin{problem}
Show that the inverse mapping of a dilation is again a dilation with
the same center but with magnification factor $r^{-1}$.
\begin{hint}
Solve for $(x,y)$ in terms of $(\underline{x},\underline{y})$.
\end{hint}
\begin{freeResponse}
Write
\begin{align*}
(\underline{x},\underline{y}) &= (x_{0},y_{0})  +r(x-x_{0},y-y_{0}), \\
(\underline{x},\underline{y}) &= (x_{0},y_{0})  +r(x,y) -r(x_{0},y_{0}), \\
r(x_0,y_0) + (\underline{x},\underline{y}) - (x_0,y_0) &= r(x,y), \\
(x_0,y_0) + r^{-1}(\underline{x}-x_0,\underline{y}-y_0) &= (x,y). \\
\end{align*}
Hence the inverse mapping of a dilation is again a dilation with the
same center but with magnification factor $r^{-1}$.
\end{freeResponse}
\end{problem}


\begin{problem}
Show using several-variable calculus that a dilation $D$ with
magnification factor $r$ multiplies all areas by a factor of $r^{2}$.
\begin{hint}
Recall that if $R$ is a region in the plane, $D(R)$ is the dialated region and 
\[
\iint_{D(R)} f(x,y)  \d xdy = \iint_{R} f(D(x,y)) \left|\det J_D(x,y)\right| \d xdy,
\]
where
\[
J_D(x,y) =
{\renewcommand\arraystretch{2}
\begin{bmatrix}
\pp[D_x]{x} & \pp[D_y]{x} \\
\pp[D_x]{y} & \pp[D_y]{y}
\end{bmatrix}}
\]
and $D_x$ and $D_y$ are the components of $D(x,y)$.
\end{hint}
\begin{freeResponse}
Let $R$ be the region in the plane. Write
\[
\int_{D(R)} 1  \d xdy = \int_{R}  \left|\det J_D(x,y)\right| \d xdy,
\]
where
\[
J_D(x,y) =
{\renewcommand\arraystretch{2}
\begin{bmatrix}
\pp[D_x]{x} & \pp[D_x]{y} \\
\pp[D_y]{x} & \pp[D_y]{y}
\end{bmatrix}} = 
\begin{bmatrix}
r & 0 \\
0 & r
\end{bmatrix}.
\]
Hence $\left|\det J_D(x,y)\right| = r^2$ and so 
\[
\int_{D(R)} 1  \d xdy = r^2\cdot \int_R 1\d xdy.
\]
This shows that a dilation with magnification factor $r$ multiplies
all areas by a factor of $r^{2}$.
\end{freeResponse}
\end{problem}

\begin{problem}
Give an explanation that a middle grades student would understand that
a dilation with magnification factor $r$ multiplies all areas by a
factor of $r^{2}$.
\begin{hint}
Break the region into rectangles. 
\end{hint}
\begin{freeResponse}
Break an arbitrary region into rectangles. Once we have the
rectangles, multiply each of the lengths and widths of these
rectangles by $r$. Now the area of each of these rectangles has
increased by a factor of $r^2$. Hence, the area of the entire region
has increased by a factor of $r^2$.
\end{freeResponse}
\end{problem}



\subsection{Similarity}

\begin{definition}
Two triangles are \textbf{similar} if there is a dilation of the
plane that takes one to a triangle which is congruent to the other. We write%
\[
\triangle ABC\sim\triangle A'B'C'%
\]
to denote that these two triangles are similar (where the order of the
vertices tells us which vertices correspond).
\end{definition}

\begin{problem}\hfil
\begin{enumerate}
\item Show that, if two triangles are similar, then corresponding
sides are proportional with the same constant of proportionality.

\item Show that, if corresponding sides of two triangles are
  proportional with the same constant of proportionality, then the two
  triangles are similar.


\end{enumerate}
\begin{hint}
For the first part, you have to start from the supposition that the
two triangles satisfy our definition of similar triangles.
\end{hint}
\begin{hint}
For the second part, you have to start from the supposition that
corresponding sides of the two triangles are proportional and use SSS
to show that there is a dilation of $\triangle ABC$ is congruent to
$\triangle A'B'C'$.
\end{hint}
\begin{freeResponse}
For the first part, suppose that we have two similar triangles. This
means that there is a dilation that takes one triangle to a triangle
congruent to the other. Since a dilation with magnification factor
$r$ takes a vector to $r$ times that vector, and congruence preserves
lengths, we have that
\begin{align*}
|AB| &= r\cdot |A'B'|\\
|AC| &= r\cdot |A'C'|\\
|BC| &= r\cdot |B'C'|.
\end{align*}


For the second part, suppose that corresponding sides are proportional
with the same constant of proportionality. So
\begin{align*}
|AB| &= r\cdot |A'B'|\\
|AC| &= r\cdot |A'C'|\\
|BC| &= r\cdot |B'C'|.
\end{align*}
Now a dilation with magnification factor $r$ will take corresponding
sides of $\triangle ABC$ to corresponding sides of $\triangle
A'B'C'$. After this, by SSS we know that the two triangles are
congruent.
\end{freeResponse}
\end{problem}

\begin{problem}\hfil
\begin{enumerate}
\item Show that, if two triangles are similar, then corresponding
angles are equal.

\item Show that, if corresponding angles of two triangles are equal, then the two
triangles are similar.
\end{enumerate}

\begin{hint}
For the first part, you have to start from the supposition that the
two triangles satisfy our definition of similar triangles.
\end{hint}
\begin{hint}
For the second part, you have to start from the supposition that
corresponding angles of the two triangles are equal, then use a
dilation with $r=|A'B'|/|AB|$ and ASA to show that the dilation of one
triangle is congruent to the other.
\end{hint}
\begin{freeResponse}
For the first part, suppose that we have two similar triangles. This
means that there is a dilation that takes one triangle to a triangle
congruent to the other. Since dilations preserve angles,
corresponding angles must be equal.

For the second part, suppose that corresponding angles of two
triangles are equal. Set the magnification factor of the dilation $D$
to be
\[
r = \frac{|A'B'|}{|AB|}.
\]
In this case, $D$ maps $\bar{AB}$ to a segment congruent to
$\bar{A'B'}$. Moreover, dilations preserve angles. Hence by ASA, $D$
maps $\triangle ABC$ to a triangle congruent to $\triangle A'B'C'$,
and so $\triangle ABC \sim \triangle A'B'C'$.
\end{freeResponse}
\end{problem}

\begin{problem}
\label{39} Show that two triangles are similar if corresponding
sides are parallel.
\begin{hint}
Use the fact that angles are equal if corresponding rays are parallel.
\end{hint}
\begin{freeResponse}
Since for our triangles, corresponding sides are parallel, the rays of
the corresponding angles are parallel. This means that corresponding
angles are equal. Hence the triangles are similar.
\end{freeResponse}
\end{problem}

\begin{problem}
Show that two triangles are similar if corresponding sides are perpendicular.
\begin{hint}
Just think about one angle at a time.
\end{hint}
\begin{freeResponse}
It is sufficient to show that given two angles with perpendicular
legs, then the angles are equal.

Extend the legs of the angles until they cross at right angles. There
could be several configurations, but all proofs will be similar. Right
triangles will be formed containing the angle in question. Using the
fact that right triangles are similar if and only if two nonright
angles are equal one can show the result for any configuration. 
\end{freeResponse}
\end{problem}


\begin{problem}
Summarize the results from this section. In particular, indicate which
results follow from the others.
\begin{freeResponse}
\end{freeResponse}
\end{problem}
\end{document}
