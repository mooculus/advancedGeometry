\documentclass{ximera}

\usepackage{microtype}
\usepackage{tikz}
\usepackage{tkz-euclide}
\usetkzobj{all}
\tikzstyle geometryDiagrams=[ultra thick,color=blue!50!black]

\graphicspath{
{./}
{areasOnSpheresInEuclidean3Space/}
{centralProjection/}
{stereographicProjection/}
{centralProjectionInHG/}
{stereographicProjectionInHG/}
}


\newcommand{\transpose}{\intercal}
\renewcommand{\epsilon}{\varepsilon}
\renewcommand{\l}{\ell}
\renewcommand{\d}{\,d}

\newcommand{\R}{\mathbb R}


\renewcommand{\bar}{\overline}


%% \prerequisites{euclideanGeometry}
%% \outcome{circles}

\title{Properties of circles in \textbf{EG}}

\begin{document}
\begin{abstract}
In this activity, we study circles in euclidean geometry.
\end{abstract}
\maketitle


The next topic in euclidean geometry is circles. We include this
partly for its own interest, and partly because the properties we
visit here will be useful later on. Again we explore the topic through
a sequence of problems. We begin with perhaps the most basic fact of
all about circles in \textbf{EG}.

\begin{problem}
The circle of radius $1$ has (interior) area $\pi$. Use this to reason
to the fact that the circle of radius $1$ has circumference $2\pi$.%
\begin{image}
\begin{tikzpicture}[geometryDiagrams]
\draw (0,0) circle (2cm);
\foreach \coeff in {0,10,20,...,350}
         {
           \draw[thin] (0,0)--({2*cos(\coeff)},{2*sin(\coeff)});
         }
\end{tikzpicture}
\end{image}

\begin{hint}
Approximate a rectangle by rearranging the slices in the picture.
Compute the area of the ``rectangle.''
\end{hint}
\begin{freeResponse}
Slice up the circle as indicated and rearrange into the following
``rectangle.'' 
\begin{image}
\begin{tikzpicture}[geometryDiagrams] % r= 2cm
\foreach \coeff in {0,1,2,...,17}
         {
           \draw[thin] ({2*\coeff *sin(5)},0)--({(2*\coeff+ 2)* sin(5)},{2*cos(5)});
           \draw[thin] ({(2*\coeff + 2)*sin(5)},{2*cos(5)})--({(2+2*\coeff)*sin(5)},0); %one too many!
         }
         \draw[thin] ({2*18 *sin(5)},0)--({(2*18+ 2)* sin(5)},{2*cos(5)});
         \draw (0,0)--({2*18 *sin(5)},0);
         \draw ({2*sin(5)},2)--({2*19 *sin(5)},2);
\end{tikzpicture}
\end{image}
Note, while this is not truly a rectangle, as the slices become finer,
and finer, the shape will more, and more, closely approximate a
rectangle. This rectangle has area $\pi$, with the (short) side length
being $1$. Hence the length of the top and bottom are both $\pi$. From
this we see that the circumference of the circle is $2\pi$.
\end{freeResponse}
\end{problem}


Before we turn to some facts about chords in circles and angles
inscribed in circles, we need the following prelimary fact.

\begin{problem}
Prove that given an isoceles triangle, angles opposite the conrguent
sides are also congruent.
\begin{hint}
Use a congruence theorem.
\end{hint}
\begin{freeResponse}
Consider an isoceles triangle and construct a median to the
``nonequal'' side.
\begin{image}
\begin{tikzpicture}[geometryDiagrams]
\coordinate (A) at (0,0);
\coordinate (B) at (2,0);
\coordinate (C) at (1,3);
\coordinate (D) at (1,0);
\draw (A)--(B)--(C)--cycle;
\draw[thin] (C)--(D);
\tkzMarkSegments[mark=|](A,C B,C)
\tkzMarkSegments[mark=||](A,D B,D)
\tkzLabelPoints[above](C)
\tkzLabelPoints[below](A,D,B)
%\draw[step=.5cm] (0,0) grid (10,5);
\end{tikzpicture}
\end{image}
By SSS, we see that triangle $\triangle ADC \cong \triangle
BDC$. Hence $\angle DAC \cong \angle DBC$.
\end{freeResponse}
\end{problem}

\begin{problem}
On the circle with center $O$ below,
\begin{image}
\begin{tikzpicture}[geometryDiagrams]
\coordinate (O) at (0,0);
\coordinate (A) at ({2*cos(40)},{2*sin(40)});
\coordinate (B) at ({2*cos(350)},{2*sin(350)});
\coordinate (X) at ({2*cos(170)},{2*sin(170)});

\draw[thin] (A)--(O);
\draw[thin] (A)--(X);
\draw[thin] (X)--(B);
\draw (0,0) circle (2cm);

\tkzLabelPoints[above](O)
\tkzLabelPoints[left](X)
\tkzLabelPoints[right](A,B)

\end{tikzpicture}
\end{image}
show that%
\[
\angle BXA=(1/2)(\angle BOA).
\]
\begin{hint}
$\triangle OAX$ is isosceles.
\end{hint}
\begin{freeResponse}
Since all radii of a circle are equal in length, we have the following
diagram with an isoceles triangle. 
\begin{image}
\begin{tikzpicture}[geometryDiagrams]
\coordinate (O) at (0,0);
\coordinate (A) at ({2*cos(40)},{2*sin(40)});
\coordinate (B) at ({2*cos(350)},{2*sin(350)});
\coordinate (X) at ({2*cos(170)},{2*sin(170)});

\draw[thin] (A)--(O);
\draw[thin] (A)--(X);
\draw[thin] (X)--(B);
\draw (0,0) circle (2cm);

\tkzLabelPoints[above](O)
\tkzLabelPoints[left](X)
\tkzLabelPoints[right](A,B)
\tkzMarkSegments[mark=|](X,O O,A)
\end{tikzpicture}
\end{image}
Hence by our previous result, $\angle OXA \cong \angle OAX$. From this we see that 
\begin{align*}
\angle OXA + \angle OAX + \angle XOA &= \angle BOA + \angle XOA,\\
2\cdot \angle OXA &= \angle BOA.
\end{align*}
Since $\angle OXA = \angle BXA$, we have shown $\angle BXA = (1/2)\angle BOA$.
\end{freeResponse}
\end{problem}



\begin{problem}
On the circle with center $O$ below,
\begin{image}
\begin{tikzpicture}[geometryDiagrams]
\coordinate (O) at (0,0);
\coordinate (A) at ({2*cos(40)},{2*sin(40)});
\coordinate (B) at ({2*cos(290)},{2*sin(290)});
\coordinate (X) at ({2*cos(170)},{2*sin(170)});

\draw[thin] (A)--(O);
\draw[thin] (A)--(X);
\draw[thin] (X)--(B);
\draw[thin] (O)--(B);
\draw (0,0) circle (2cm);

\tkzLabelPoints[above](O)
\tkzLabelPoints[left](X)
\tkzLabelPoints[right](A)
\tkzLabelPoints[below](B)

\end{tikzpicture}
\end{image}
show that%
\[
\angle BXA=(1/2)(\angle BOA).
\]

\begin{hint}
Draw the diameter through $O$ and $X$ and add.
\end{hint}

\begin{freeResponse}
\begin{image}
\begin{tikzpicture}[geometryDiagrams]
\coordinate (O) at (0,0);
\coordinate (A) at ({2*cos(40)},{2*sin(40)});
\coordinate (B) at ({2*cos(290)},{2*sin(290)});
\coordinate (X) at ({2*cos(170)},{2*sin(170)});
\coordinate (Y) at ({2*cos(350)},{2*sin(350)});

\draw[thin] (A)--(O);
\draw[thin] (A)--(X);
\draw[thin] (X)--(B);
\draw[thin] (O)--(B);
\draw[thin] (X)--(Y);
\draw (0,0) circle (2cm);

\tkzLabelPoints[above](O)
\tkzLabelPoints[left](X)
\tkzLabelPoints[right](Y)
\tkzLabelPoints[right](A)
\tkzLabelPoints[below](B)

\tkzMarkSegments[mark=|](X,O O,A O,B)


\end{tikzpicture}
\end{image}
\end{freeResponse}
\end{problem}



\begin{problem}
On the circle with center $O$ below,
\begin{image}
\begin{tikzpicture}[geometryDiagrams]
\coordinate (O) at (0,0);
\coordinate (A) at ({2*cos(60)},{2*sin(60)});
\coordinate (B) at ({2*cos(20)},{2*sin(20)});
\coordinate (X) at ({2*cos(170)},{2*sin(170)});

\draw[thin] (A)--(O);
\draw[thin] (A)--(X);
\draw[thin] (X)--(B);
\draw[thin] (O)--(B);
\draw (0,0) circle (2cm);

\tkzLabelPoints[below](O)
\tkzLabelPoints[left](X)
\tkzLabelPoints[above](A)
\tkzLabelPoints[right](B)

\end{tikzpicture}
\end{image}
show that
\[
\angle BXA=(1/2)(\angle BOA).
\]

\begin{hint}
Draw the diameter through $O$ and $X$ and subtract.
\end{hint}
\begin{freeResponse}
\begin{image}
\begin{tikzpicture}[geometryDiagrams]
\coordinate (O) at (0,0);
\coordinate (A) at ({2*cos(60)},{2*sin(60)});
\coordinate (B) at ({2*cos(20)},{2*sin(20)});
\coordinate (X) at ({2*cos(170)},{2*sin(170)});

\draw[thin] (A)--(O);
\draw[thin] (A)--(X);
\draw[thin] (X)--(B);
\draw[thin] (O)--(B);
\draw (0,0) circle (2cm);

\tkzLabelPoints[below](O)
\tkzLabelPoints[left](X)
\tkzLabelPoints[above](A)
\tkzLabelPoints[right](B)

\end{tikzpicture}
\end{image}
\end{freeResponse}

\end{problem}



We can summarize the results of the last three problems into the following theorem.

\begin{theorem}
\label{43}The measure of any angle inscribed in a circle is one-half of the
measure of the corresponding central angle.
\end{theorem}

\begin{problem}
Use similar triangles and the previous problems to show that
\[
\left\vert AX\right\vert \cdot \left\vert
XB\right\vert =\left\vert A'X\right\vert\cdot \left\vert
XB'\right\vert
\]
in the figure below.%
\begin{image}
\begin{tikzpicture}[geometryDiagrams]
\coordinate (X) at (-.02,1.02);
\coordinate (B') at ({2*cos(60)},{2*sin(60)});
\coordinate (B) at ({2*cos(20)},{2*sin(20)});
\coordinate (A) at ({2*cos(140)},{2*sin(140)});
\coordinate (A') at ({2*cos(190)},{2*sin(190)});

\draw[thin] (A)--(B);
\draw[thin] (A')--(B');
\draw (0,0) circle (2cm);

\tkzLabelPoints[left](A,A')
\tkzLabelPoints[right](B)
\tkzLabelPoints[above](B')
\tkzLabelPoints[above](X)

\end{tikzpicture}
\end{image}
\begin{hint}
Draw $\overline{AB'}$ and $\overline{A'B}$.
\end{hint}
\end{problem}


\begin{problem}
Use similar triangles and the previous problems to show that
\[
\left\vert AX\right\vert\cdot\left\vert XB\right\vert
=\left\vert A'X\right\vert \cdot \left\vert XB'\right\vert
\]
in the figure below.%
\begin{image}
\begin{tikzpicture}[geometryDiagrams]
\coordinate (X) at (4.51,.22);
\coordinate (B') at ({2*cos(0)},{2*sin(0)});
\coordinate (B) at ({2*cos(20)},{2*sin(20)});
\coordinate (A) at ({2*cos(140)},{2*sin(140)});
\coordinate (A') at ({2*cos(190)},{2*sin(190)});

\draw[thin] (A)--(X);
\draw[thin] (A')--(X);
\draw (0,0) circle (2cm);

\tkzLabelPoints[left](A,A')
\tkzLabelPoints[above right](B)
\tkzLabelPoints[below right](B')
\tkzLabelPoints[right](X)

\end{tikzpicture}
\end{image}

\begin{hint}
Draw $\overline{AB^{\prime}}$ and $\overline{A^{\prime}B}$.
\end{hint}
\end{problem}




\begin{problem}
Show that, given any three non-collinear points in the Euclidean
plane, there is a unique circle passing through the three points.

\begin{hint}
Show that the center of the circle must be the intersection of
the perpendicular bisectors of any two of the sides of the triangle
whose vertices are the three given points.
\end{hint}
\end{problem}

But how about four points in the plane, no three of which are
collinear?

\begin{problem}
\begin{enumerate}\hfil
\item Draw four points in the Euclidean plane, no $3$ of which are collinear, that cannot lie on a single circle.
\item Draw four points in the Euclidean plane that do lie on a single
circle.
\end{enumerate}
\end{problem}



\end{document}
