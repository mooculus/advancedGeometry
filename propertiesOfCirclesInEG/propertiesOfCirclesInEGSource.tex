\shortdescription{In this activity we study circles in Euclidean geometry.}
\activitytitle{Properties of circles in \textbf{EG}}
\prerequisites{euclideanGeometry}
\outcomes{outcome1}


\subsection{Basics}

Our final topic before leaving Euclidean Geometry is circles. We include this
partly for its own interest, and partly because the properites we visit here
will be useful later on. Again we explore the topic through a sequence of
Exercises (with Hints to their solutions to ease the way). We begin with
perhaps the most basic fact of all about circles in \textbf{EG}.

\begin{exercise}
(\textbf{EG}) The circle of radius $1$ has (interior) area $\pi$. Use this to
reason to the fact that the circle of radius $1$ has circumference $2\pi$.%
\[%
%TCIMACRO{\FRAME{itbpF}{1.3413in}{1.3206in}{0in}{}{}{Figure}%
%{\special{ language "Scientific Word";  type "GRAPHIC";
%maintain-aspect-ratio TRUE;  display "USEDEF";  valid_file "T";
%width 1.3413in;  height 1.3206in;  depth 0in;  original-width 5.5383in;
%original-height 5.4509in;  cropleft "0";  croptop "1";  cropright "1";
%cropbottom "0";  tempfilename 'MXAJBX04.png';tempfile-properties "XP";}}}%
%BeginExpansion
{\includegraphics[
natheight=5.450900in,
natwidth=5.538300in,
height=1.3206in,
width=1.3413in
]%
{MXAJBX04.png}%
}%
%EndExpansion
\]


Hint: Approximate a rectangle by rearranging the slices in the picture.
Compute the area of the ``rectangle.''
\end{exercise}

Next we turn to some facts about chords in circles and angels inscribed in circles.

\begin{exercise}
(\textbf{EG}) On the circle with center O below,
\[%
%TCIMACRO{\FRAME{itbpF}{1.8127in}{1.4814in}{0in}{}{}{Figure}%
%{\special{ language "Scientific Word";  type "GRAPHIC";
%maintain-aspect-ratio TRUE;  display "USEDEF";  valid_file "T";
%width 1.8127in;  height 1.4814in;  depth 0in;  original-width 6.9453in;
%original-height 5.6697in;  cropleft "0";  croptop "1";  cropright "1";
%cropbottom "0";  tempfilename 'MXAJBY05.png';tempfile-properties "XPR";}}}%
%BeginExpansion
{\includegraphics[
natheight=5.669700in,
natwidth=6.945300in,
height=1.4814in,
width=1.8127in
]%
{MXAJBY05.png}%
}%
%EndExpansion
\]
show that%
\[
\angle AXB=(1/2)(\angle AOB).
\]


Hint: $\triangle OAX$ is isosceles.
\end{exercise}

\begin{exercise}
(\textbf{EG}) On the circle with center $O$ below,
\[%
%TCIMACRO{\FRAME{itbpF}{1.6509in}{1.5878in}{0in}{}{}{Figure}%
%{\special{ language "Scientific Word";  type "GRAPHIC";  display "USEDEF";
%valid_file "T";  width 1.6509in;  height 1.5878in;  depth 0in;
%original-width 6.3296in;  original-height 6.2855in;  cropleft "0";
%croptop "1";  cropright "1";  cropbottom "0";
%tempfilename 'MXAJBY06.png';tempfile-properties "XPR";}}}%
%BeginExpansion
{\includegraphics[
natheight=6.285500in,
natwidth=6.329600in,
height=1.5878in,
width=1.6509in
]%
{MXAJBY06.png}%
}%
%EndExpansion
\]
show that%
\[
\angle AXB=(1/2)(\angle AOB).
\]


Hint: Draw the diameter through $O$ and $X$ and add.
\end{exercise}

\begin{exercise}
(\textbf{EG}) On the circle with center $O$ below,
\[%
%TCIMACRO{\FRAME{itbpF}{1.7218in}{1.5031in}{0in}{}{}{Figure}%
%{\special{ language "Scientific Word";  type "GRAPHIC";
%maintain-aspect-ratio TRUE;  display "USEDEF";  valid_file "T";
%width 1.7218in;  height 1.5031in;  depth 0in;  original-width 6.5492in;
%original-height 5.7147in;  cropleft "0";  croptop "1";  cropright "1";
%cropbottom "0";  tempfilename 'MXAJBY07.png';tempfile-properties "XPR";}}}%
%BeginExpansion
{\includegraphics[
natheight=5.714700in,
natwidth=6.549200in,
height=1.5031in,
width=1.7218in
]%
{MXAJBY07.png}%
}%
%EndExpansion
\]
show that{}%
\[
\angle AXB=(1/2)(\angle AOB).
\]


Hint: Draw the diameter through $O$ and $X$ and subtract.
\end{exercise}

We can summarize the results of the last three exercises into the following Theorem.

\begin{theorem}
\label{43}The measure of any angle inscribed in a circle is one-half of the
measure of the corresponding central angle.
\end{theorem}

\begin{exercise}
(\textbf{EG}) Use similar triangles and the previous Exercises to show that
$\left\vert AX\right\vert $\textperiodcentered$\left\vert XB\right\vert
=\left\vert A^{\prime}X\right\vert $\textperiodcentered$\left\vert XB^{\prime
}\right\vert $ in the figure below.%
\[%
%TCIMACRO{\FRAME{itbpF}{1.6319in}{1.3206in}{0pt}{}{}{Figure}%
%{\special{ language "Scientific Word";  type "GRAPHIC";
%maintain-aspect-ratio TRUE;  display "USEDEF";  valid_file "T";
%width 1.6319in;  height 1.3206in;  depth 0pt;  original-width 7.0776in;
%original-height 5.7147in;  cropleft "0";  croptop "1";  cropright "1";
%cropbottom "0";  tempfilename 'MXAJBY08.png';tempfile-properties "XPR";}}}%
%BeginExpansion
\raisebox{-0pt}{\includegraphics[
natheight=5.714700in,
natwidth=7.077600in,
height=1.3206in,
width=1.6319in
]%
{MXAJBY08.png}%
}%
%EndExpansion
\]


Hint: Draw $\overline{AB^{\prime}}$ and $\overline{A^{\prime}B}$.
\end{exercise}

\begin{exercise}
(\textbf{EG}) Use similar triangles and the previous Exercises to show that
$\left\vert AX\right\vert $\textperiodcentered$\left\vert XB\right\vert
=\left\vert A^{\prime}X\right\vert $\textperiodcentered$\left\vert XB^{\prime
}\right\vert $ in the figure below.%
\[%
%TCIMACRO{\FRAME{itbpF}{2.9032in}{1.3284in}{0in}{}{}{Figure}%
%{\special{ language "Scientific Word";  type "GRAPHIC";
%maintain-aspect-ratio TRUE;  display "USEDEF";  valid_file "T";
%width 2.9032in;  height 1.3284in;  depth 0in;  original-width 12.4395in;
%original-height 5.6697in;  cropleft "0";  croptop "1";  cropright "1";
%cropbottom "0";  tempfilename 'MXAJBY09.png';tempfile-properties "XPR";}}}%
%BeginExpansion
{\includegraphics[
natheight=5.669700in,
natwidth=12.439500in,
height=1.3284in,
width=2.9032in
]%
{MXAJBY09.png}%
}%
%EndExpansion
\]


Hint: Draw $\overline{AB^{\prime}}$ and $\overline{A^{\prime}B}$.
\end{exercise}

\begin{exercise}
Show that, given any three non-collinear points in the Euclidean plane, there
is a unique circle passing through the three points.

Hint: Show that the center of the circle must be the intersection of the
perpendicular bisectors of any two of the sides of the triangle whose vertices
are the three given points.
\end{exercise}

But how about four points in the plane, no three of which are collinear?

\begin{exercise}
a) Draw four points in the Euclidean plane, no $3$ of which are collinear,
that cannot lie on a single circle.

b) Draw four points in the Euclidean plane that do lie on a single circle.
\end{exercise}

The issue we will explore in the next two sections is the question of finding
a numerical condition about the four points that tells us exactly when they
all lie on a single circle. For that, we will need a very famous mathematical
relationship, one very closely related to the notion of perspective in
painting. That is, how do you faithfully render depth on a flat canvas? This
relationship is called the \textit{cross-ratio} of the four points. \pagebreak

\subsection{Cross-ratio of points on the number line}

We begin by studying the cross-ratio of four points on a line. Start with the
set of points on the real number line with coordinate $t$ and add one extra
point called $t=\infty$. Call the resulting set $\overline{\mathbb{R}}$. You
could think of the resulting set as the set of all lines through the origin in
$\mathbb{R}^{2}$ by assigning to each line the real number that is its slope
and to the $y$-axis the slope $\infty$.

\begin{exercise}
a) Show that the transformation%
\[
\left(  x,y\right)  \mapsto\left(  \underline{x},\underline{y}\right)
=\left(  x,y\right)  \cdot\left(
\begin{array}
[c]{cc}%
d & b\\
c & a
\end{array}
\right)
\]
is a $1-1$, onto (linear) transformation of $\mathbb{R}^{2}$ as long as%
\begin{equation}
\left\vert
\begin{array}
[c]{cc}%
d & b\\
c & a
\end{array}
\right\vert \neq0. \label{52}%
\end{equation}


b) For the transformation in the previous Exercise, show that every line
through the origin in $\left(  x,y\right)  $-space is sent to a line through
the origin in $\left(  \underline{x},\underline{y}\right)  $-space. The slope
$t$ of the line through $\left(  0,0\right)  $ and $\left(  x,y\right)  $ is
of course $t=\frac{y}{x}$. What is the slope \underline{$t$} of the line
through $\left(  \underline{0},\underline{0}\right)  $ and $\left(
\underline{x},\underline{y}\right)  $? Show that%
\begin{equation}
\underline{t}=\frac{at+b}{ct+d} \label{41}%
\end{equation}

\end{exercise}

\begin{definition}
Functions $\left(  \ref{41}\right)  $ for which the condition $\left(
\ref{52}\right)  $ holds are called \textbf{linear fractional transformations}.
\end{definition}

\begin{exercise}
Show that a linear fractional transformation%
\begin{gather*}
\overline{\mathbb{R}}\rightarrow\overline{\mathbb{R}}\\
t\mapsto\underline{t}=\frac{at+b}{ct+d}%
\end{gather*}
is $1-1$ and onto. What is its inverse function? (Your answer should show that
the inverse function is also a linear fractional transformation.)

Hint: By algebra solve for $t$ in terms of \underline{$t$}. Then graph%
\[
\underline{t}=\frac{at+b}{ct+d}%
\]
in the $\left(  t,\underline{t}\right)  $-plane. If $c=0$ show that the graph
is a straight line with non-zero slope and%
\[
\infty\mapsto\infty.
\]
If $c\neq0$, show that the graph has exactly one horizontal asymptote where
$t\mapsto\infty$ and one vertical asymptote where $\underline{t}\mapsto\infty$.
\end{exercise}

\begin{exercise}
Show that the set of linear fractional transformations form a group under the
operation of composition of functions. That is, check associativity, identity
element and existence of inverses.
\end{exercise}

\begin{exercise}
\label{59}Show that, for any three distinct points $t_{2},t_{3}$ and $t_{4}$,
the function of $t$ given by the formula%
\[
\underline{t}=\frac{t_{3}-t_{4}}{t_{3}-t_{2}}\frac{t-t_{2}}{t-t_{4}}%
=\frac{t-t_{2}}{t_{3}-t_{2}}\div\frac{t-t_{4}}{t_{3}-t_{4}}%
\]
takes $t_{2}$ to $0$, takes $t_{3}$ to $1$ and takes $t_{4}$ to $\infty$. Show
that this function is a linear fractional transformation, that is, a function
of the form $\left(  \ref{41}\right)  $ for which the condition $\left(
\ref{52}\right)  $ holds.
\end{exercise}

\begin{exercise}
\label{57}Show that any linear fractional transformation $\left(
\ref{41}\right)  $ that leaves $0$, $1$, and $\infty$ fixed is the identity map.
\end{exercise}

\begin{exercise}
\label{42}Suppose that you are given a function $\left(  \ref{41}\right)  $
and four points $t_{1},t_{2},t_{3}$ and $t_{4}$. Let
\[
\underline{t_{i}}=\frac{at_{i}+b}{ct_{i}+d}%
\]
for $i=1,2,3,4$. Show that%
\[
\frac{\underline{t_{1}}-\underline{t_{2}}}{\underline{t_{3}}-\underline{t_{2}%
}}\div\frac{\underline{t_{1}}-\underline{t_{4}}}{\underline{t_{3}}%
-\underline{t_{4}}}=\frac{t_{1}-t_{2}}{t_{3}-t_{2}}\div\frac{t_{1}-t_{4}%
}{t_{3}-t_{4}}.
\]
[MJG,288]

Hint: Just write out the formula for each side and do the high school algebra.
There is a fancier way that uses that the set of linear fractional
transformations form a group whose operation is composition. It goes like
this. Use Exercise \ref{59} to show that the inverse of the linear fractional
transformation
\[
t\mapsto\frac{t-t_{2}}{t_{3}-t_{2}}\div\frac{t-t_{4}}{t_{3}-t_{4}}%
\]
followed by%
\[
t\mapsto\underline{t}%
\]
and then followed by
\[
t\mapsto\frac{t-\underline{t_{2}}}{\underline{t_{3}}-\underline{t_{2}}}%
\div\frac{t-\underline{t_{4}}}{\underline{t_{3}}-\underline{t_{4}}}%
\]
fixes $0$, $1$, and $\infty$ and so is the identity transformation by Exercise
\ref{57}. So%
\[
t\mapsto\frac{t-t_{2}}{t_{3}-t_{2}}\div\frac{t-t_{4}}{t_{3}-t_{4}}%
\]
is the same transformation as%
\[
t\mapsto\frac{\underline{t}-\underline{t_{2}}}{\underline{t_{3}}%
-\underline{t_{2}}}\div\frac{\underline{t}-\underline{t_{4}}}{\underline
{t_{3}}-\underline{t_{4}}}.
\]

\end{exercise}

\begin{definition}
\label{44}The cross-ratio $\left(  t_{1}:t_{2}:t_{3}:t_{4}\right)  $ of four
(ordered) points $t_{1},t_{2},t_{3}$ and $t_{4}$ is defined by%
\[
\left(  t_{1}:t_{2}:t_{3}:t_{4}\right)  =\frac{t_{1}-t_{2}}{t_{3}-t_{2}}%
\div\frac{t_{1}-t_{4}}{t_{3}-t_{4}}.
\]

\end{definition}

Exercise \ref{42} shows that if four points are moved by any function $\left(
\ref{41}\right)  $ the cross-ratio $\left(  \underline{t_{1}}:\underline
{t_{2}}:\underline{t_{3}}:\underline{t_{4}}\right)  $ of the output four
points is the same as the cross-ratio $\left(  t_{1}:t_{2}:t_{3}:t_{4}\right)
$ of the original four points.\pagebreak

\subsection{Cross-ratio of points on a circle}

\begin{exercise}
\label{46}(\textbf{EG}) a) In the diagram
\[%
%TCIMACRO{\FRAME{itbpF}{1.1087in}{1.0032in}{0in}{}{}{Figure}%
%{\special{ language "Scientific Word";  type "GRAPHIC";
%maintain-aspect-ratio TRUE;  display "USEDEF";  valid_file "T";
%width 1.1087in;  height 1.0032in;  depth 0in;  original-width 6.0217in;
%original-height 5.4509in;  cropleft "0";  croptop "1";  cropright "1";
%cropbottom "0";  tempfilename 'MXAJBY0A.png';tempfile-properties "XPR";}}}%
%BeginExpansion
{\includegraphics[
natheight=5.450900in,
natwidth=6.021700in,
height=1.0032in,
width=1.1087in
]%
{MXAJBY0A.png}%
}%
%EndExpansion
\]
show that%
\[
\frac{\left\vert AB\right\vert }{\left\vert CB\right\vert }=\frac
{\mathrm{sin}\alpha}{\mathrm{sin}\beta}=\frac{\mathrm{sin}\left(  \angle
AOB\right)  }{\mathrm{sin}\left(  \angle COB\right)  }.
\]


Hint: Notice that by Theorem \ref{43}
\[
m\left(  \angle BAO\right)  +m\left(  OCB\right)  =180^{\circ}%
\]
so that%
\[
\mathrm{sin}\left(  \angle BAO\right)  =\mathrm{sin}\left(  OCB\right)  .
\]
Now use the Law of Sines.
\end{exercise}

\begin{exercise}
\label{47}(\textbf{EG}) Show that if, in the above figure, $B$ moves along the
circle to the other side of $C$, it is still true that%
\[
\frac{\left\vert AB\right\vert }{\left\vert CB\right\vert }=\frac
{\mathrm{sin}\left(  \angle AOB\right)  }{\mathrm{sin}\left(  \angle
COB\right)  }%
\]

\end{exercise}

\begin{exercise}
\label{48}(\textbf{EG})\ In the diagram%
\begin{equation}%
%TCIMACRO{\FRAME{itbpF}{1.9666in}{1.1312in}{0in}{}{}{Figure}%
%{\special{ language "Scientific Word";  type "GRAPHIC";
%maintain-aspect-ratio TRUE;  display "USEDEF";  valid_file "T";
%width 1.9666in;  height 1.1312in;  depth 0in;  original-width 11.7364in;
%original-height 6.7256in;  cropleft "0";  croptop "1";  cropright "1";
%cropbottom "0";  tempfilename 'MXAJBZ0B.png';tempfile-properties "XPR";}}}%
%BeginExpansion
{\includegraphics[
natheight=6.725600in,
natwidth=11.736400in,
height=1.1312in,
width=1.9666in
]%
{MXAJBZ0B.png}%
}%
%EndExpansion
\label{45}%
\end{equation}

\end{exercise}

show that%
\[
\frac{\left\vert A^{\prime}B^{\prime}\right\vert }{\left\vert C^{\prime
}B^{\prime}\right\vert }=\frac{\mathrm{sin}\alpha}{\mathrm{sin}\beta}\div
\frac{\mathrm{sin}\gamma}{\mathrm{sin}\delta}=\frac{\mathrm{sin}\left(  \angle
A^{\prime}OB^{\prime}\right)  }{\mathrm{sin}\left(  \angle C^{\prime
}OB^{\prime}\right)  }\div\frac{\mathrm{sin}\left(  \angle B^{\prime}%
A^{\prime}O\right)  }{\mathrm{sin}\left(  \angle B^{\prime}C^{\prime}O\right)
}.
\]
[MJG,266-267]

\begin{exercise}
\label{49}(\textbf{EG}) Show that if, in the above figure, $B^{\prime}$ moves
along the line to the other side of $C^{\prime}$, it is still true that%
\[
\frac{\left\vert A^{\prime}B^{\prime}\right\vert }{\left\vert C^{\prime
}B^{\prime}\right\vert }=\frac{\mathrm{sin}\left(  \angle A^{\prime}%
OB^{\prime}\right)  }{\mathrm{sin}\left(  \angle C^{\prime}OB^{\prime}\right)
}\div\frac{\mathrm{sin}\left(  \angle B^{\prime}A^{\prime}O\right)
}{\mathrm{sin}\left(  \angle B^{\prime}C^{\prime}O\right)  }.
\]

\end{exercise}

These last two Exercises allow us to define the cross-ratio of four points on
a circle.

\begin{definition}
(\textbf{EG}) For a sequence of four (ordered) points $A,B,C,$ and $D$ on a
circle, we define%
\[
\left(  A:B:C:D\right)  =\frac{\left\vert AB\right\vert }{\left\vert
CB\right\vert }\div\frac{\left\vert AD\right\vert }{\left\vert CD\right\vert }%
\]
which we call the cross-ratio of the ordered sequence of the four points.
Similarly for a sequence of four (ordered) points $A^{\prime},B^{\prime
},C^{\prime},$ and $D^{\prime}$ on a line, we define%
\[
\left(  A^{\prime}:B^{\prime}:C^{\prime}:D^{\prime}\right)  =\frac{\left\vert
A^{\prime}B^{\prime}\right\vert }{\left\vert C^{\prime}B^{\prime}\right\vert
}\div\frac{\left\vert A^{\prime}D^{\prime}\right\vert }{\left\vert C^{\prime
}D^{\prime}\right\vert }%
\]
which we call the cross-ratio of the ordered sequence of the four points.
\end{definition}

Notice that Definition \ref{44} is just a refinement of the definition of
$\left(  A^{\prime}:B^{\prime}:C^{\prime}:D^{\prime}\right)  $ just above. In
Definition \ref{44} we are keeping track of the signs of the terms in the
quotients whereas $\left(  A^{\prime}:B^{\prime}:C^{\prime}:D^{\prime}\right)
$ is always non-negative.

\begin{exercise}
\label{50}a) Show that, in the figure%
\[%
%TCIMACRO{\FRAME{itbpF}{2.0522in}{0.9842in}{0in}{}{}{Figure}%
%{\special{ language "Scientific Word";  type "GRAPHIC";
%maintain-aspect-ratio TRUE;  display "USEDEF";  valid_file "T";
%width 2.0522in;  height 0.9842in;  depth 0in;  original-width 14.0662in;
%original-height 6.7256in;  cropleft "0";  croptop "1";  cropright "1";
%cropbottom "0";  tempfilename 'MXAJBZ0C.png';tempfile-properties "XPR";}}}%
%BeginExpansion
{\includegraphics[
natheight=6.725600in,
natwidth=14.066200in,
height=0.9842in,
width=2.0522in
]%
{MXAJBZ0C.png}%
}%
%EndExpansion
\]
we have the equality%
\[
\left(  A:B:C:D\right)  =\left(  A^{\prime}:B^{\prime}:C^{\prime}:D^{\prime
}\right)  .
\]


Hint: Use Exercises \ref{46}-\ref{49}.

b) What happens in a) if we move $B$ to the other side of $C$?
\end{exercise}

We say that ``Cross-ratio is invariant under stereographic
projection.''\pagebreak

\subsection{Ptolemy's Theorem}

Given any three non-collinear points in the Euclidean plane, there is one and
only one circle that passes through the three points. (How do you construct
it?) You can easily convince yourself with a few examples that, given four
non-collinear points $A,B,C$ and $D$ in the plane, it is not always true that
there is a circle that passes through all four. A famous theorem of classical
Euclidean geometry gives the condition that there is a circle that passes
through all four.

\begin{theorem}
(Ptolemy) If the ordered sequence of points $A,B,C$ and $D$ lies on a circle,
\[%
%TCIMACRO{\FRAME{itbpF}{1.4278in}{1.2263in}{0in}{}{}{Figure}%
%{\special{ language "Scientific Word";  type "GRAPHIC";
%maintain-aspect-ratio TRUE;  display "USEDEF";  valid_file "T";
%width 1.4278in;  height 1.2263in;  depth 0in;  original-width 6.1981in;
%original-height 5.3186in;  cropleft "0";  croptop "1";  cropright "1";
%cropbottom "0";  tempfilename 'MXAJBZ0D.png';tempfile-properties "XPR";}}}%
%BeginExpansion
{\includegraphics[
natheight=5.318600in,
natwidth=6.198100in,
height=1.2263in,
width=1.4278in
]%
{MXAJBZ0D.png}%
}%
%EndExpansion
\]
then%
\[
\left\vert AC\right\vert \text{\textperiodcentered}\left\vert BD\right\vert
=\left\vert AD\right\vert \text{\textperiodcentered}\left\vert BC\right\vert
+\left\vert AB\right\vert \text{\textperiodcentered}\left\vert CD\right\vert
.
\]
That is, the product of the diagonals of the quadrilateral $ABCD$ is the sum
of the products of pairs of opposite sides.
\end{theorem}

\begin{proof}
We need to check that%
\[
\left\vert AC\right\vert \text{\textperiodcentered}\left\vert BD\right\vert
=\left\vert AD\right\vert \text{\textperiodcentered}\left\vert BC\right\vert
+\left\vert AB\right\vert \text{\textperiodcentered}\left\vert CD\right\vert
\]
or, what is the same, we need to check that%
\[
\frac{\left\vert AC\right\vert \text{\textperiodcentered}\left\vert
BD\right\vert }{\left\vert AD\right\vert \text{\textperiodcentered}\left\vert
BC\right\vert }=1+\frac{\left\vert AB\right\vert \text{\textperiodcentered
}\left\vert CD\right\vert }{\left\vert AD\right\vert \text{\textperiodcentered
}\left\vert BC\right\vert }.
\]
That is, we need to check that
\[
\left(  A:C:B:D\right)  =1+\left(  A:B:C:D\right)  .
\]
But by Exercise \ref{50} this is the same as checking that%
\[
\left(  A^{\prime}:C^{\prime}:B^{\prime}:D^{\prime}\right)  =1+\left(
A^{\prime}:B^{\prime}:C^{\prime}:D^{\prime}\right)
\]
for the projection of the four points onto a line from a point $O$ on the
circle. But that is the same thing as showing that
\[
\frac{\left\vert A^{\prime}C^{\prime}\right\vert \text{\textperiodcentered
}\left\vert B^{\prime}D^{\prime}\right\vert }{\left\vert A^{\prime}D^{\prime
}\right\vert \text{\textperiodcentered}\left\vert B^{\prime}C^{\prime
}\right\vert }=1+\frac{\left\vert A^{\prime}B^{\prime}\right\vert
\text{\textperiodcentered}\left\vert C^{\prime}D^{\prime}\right\vert
}{\left\vert A^{\prime}D^{\prime}\right\vert \text{\textperiodcentered
}\left\vert B^{\prime}C^{\prime}\right\vert }%
\]
which is the same thing as showing that%
\[
\left\vert A^{\prime}C^{\prime}\right\vert \text{\textperiodcentered
}\left\vert B^{\prime}D^{\prime}\right\vert =\left\vert A^{\prime}D^{\prime
}\right\vert \text{\textperiodcentered}\left\vert B^{\prime}C^{\prime
}\right\vert +\left\vert A^{\prime}B^{\prime}\right\vert
\text{\textperiodcentered}\left\vert C^{\prime}D^{\prime}\right\vert .
\]
Now check this last equality by high-school algebra.
\end{proof}
