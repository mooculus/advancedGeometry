\pdfobjcompresslevel=1
\documentclass[numbers,nooutcomes,twoside,hints,noauthor,12pt,]{xourse}



%\usepackage{microtype}
%\usepackage{tikz}
\usepackage{tkz-euclide}
%\usetkzobj{all}
\tikzstyle geometryDiagrams=[rounded corners=.5pt,ultra thick,color=blue!50!black]

\usepackage{tikz-cd}

\colorlet{penColor}{blue!50!black} % Color of a curve in a plot

%% \hypersetup{
%%     colorlinks = false,
%%     }


\tikzset{%% partial ellipse
    partial ellipse/.style args={#1:#2:#3}{
        insert path={+ (#1:#3) arc (#1:#2:#3)}
    }
}

\graphicspath{
{./}
{sphericalLunesAndTriangles/}
{hyperbolicLunesAndTriangles/}
{centralProjection/}
{stereographicProjection/}
{linesAnglesAndAreasInCentralProjection/}
{linesAnglesAndAreasInStereographicProjection/}
{stereographicProjection/}
{centralProjectionInHG/}
{stereographicProjectionInHG/}
{linesInSphericalGeometry/}
{linesInHyperbolicGeometry/}
{theArtOfEscher/}
}


\newcommand{\transpose}{\intercal}
\newcommand{\eval}[1]{\bigg[ #1 \bigg]}

\renewcommand{\epsilon}{\varepsilon}
\renewcommand{\l}{\ell}
\renewcommand{\d}{\,d}

\DeclareMathOperator{\arccosh}{arccosh}
\DeclareMathOperator{\arctanh}{arctanh}
\renewcommand{\tilde}{\widetilde}
\newcommand{\R}{\mathbb R}
\newcommand{\dd}[2][]{\frac{d #1}{d #2}}
\newcommand{\pp}[2][]{\frac{\partial #1}{\partial #2}}
\newcommand{\dfn}{\textbf}

\renewcommand{\bar}{\overline}
\renewcommand{\hat}{\widehat}


\ifxake
\NewEnviron{freeResponse}{}
\fi


\title{Workbook of two-dimensional geometries}
\author{C.\ Herbert Clemens and Bart Snapp}
\begin{document}
\begin{abstract}
  Two-dimensional geometries.
\end{abstract}
\maketitle

\chapterstyle

\activity{./introduction/introduction.tex}

\part{From synthetic to coordinate geometry}

\activity{./euclidsPostulatesForPlaneGeometry/euclidsPostulatesForPlaneGeometry.tex}
\activity{./rectanglesAndCartesianCoordinates/rectanglesAndCartesianCoordinates.tex} 
\activity{./dilationsAndSimilarityInEG/dilationsAndSimilarityInEG.tex}
\activity{./concurrenceTheoremsInEG/concurrenceTheoremsInEG.tex} 
\activity{./centralAndInscribedAnglesInEG/centralAndInscribedAnglesInEG.tex}
%% \activity{./theCrossRatioAndPtolemysTheorem/theCrossRatioAndPtolemysTheorem.tex}
\activity{./surfaceAreaAndVolumeOfTheRSphere/surfaceAreaAndVolumeOfTheRSphere.tex}
\activity{./sphericalLunesAndTriangles/sphericalLunesAndTriangles.tex}


\part{Changing coordinates}

\activity{./euclideanThreeSpaceAsAMetricSpace/euclideanThreeSpaceAsAMetricSpace.tex}
\activity{./congruencesThatIsRigidMotions/congruencesThatIsRigidMotions.tex}
\activity{./changingCoordinates/changingCoordinates.tex}
\activity{./rigidMotionsInXYZCoordinates/rigidMotionsInXYZCoordinates.tex}
\activity{./linesInSphericalGeometry/linesInSphericalGeometry.tex}
\activity{./linesInHyperbolicGeometry/linesInHyperbolicGeometry.tex}

\part{Projections}


\activity{./centralProjection/centralProjection.tex}
\activity{./rigidMotionsInCentralProjection/rigidMotionsInCentralProjection.tex}
\activity{./linesAnglesAndAreasInCentralProjection/linesAnglesAndAreasInCentralProjection.tex}
\activity{./stereographicProjection/stereographicProjection.tex}
\activity{./linesAnglesAndAreasInStereographicProjection/linesAnglesAndAreasInStereographicProjection.tex}
\activity{./hyperbolicLunesAndTriangles/hyperbolicLunesAndTriangles.tex}
\activity{./connectionsToSpecialRelativity/connectionsToSpecialRelativity.tex}
\activity{./theArtOfEscher/theArtOfEscher.tex}

\end{document}
