\documentclass[newpage,hints,handout,12pt,nooutcomes,noauthor]{ximera}

%\usepackage{microtype}
%\usepackage{tikz}
\usepackage{tkz-euclide}
%\usetkzobj{all}
\tikzstyle geometryDiagrams=[rounded corners=.5pt,ultra thick,color=blue!50!black]

\usepackage{tikz-cd}

\colorlet{penColor}{blue!50!black} % Color of a curve in a plot

%% \hypersetup{
%%     colorlinks = false,
%%     }


\tikzset{%% partial ellipse
    partial ellipse/.style args={#1:#2:#3}{
        insert path={+ (#1:#3) arc (#1:#2:#3)}
    }
}

\graphicspath{
{./}
{sphericalLunesAndTriangles/}
{hyperbolicLunesAndTriangles/}
{centralProjection/}
{stereographicProjection/}
{linesAnglesAndAreasInCentralProjection/}
{linesAnglesAndAreasInStereographicProjection/}
{stereographicProjection/}
{centralProjectionInHG/}
{stereographicProjectionInHG/}
{linesInSphericalGeometry/}
{linesInHyperbolicGeometry/}
{theArtOfEscher/}
}


\newcommand{\transpose}{\intercal}
\newcommand{\eval}[1]{\bigg[ #1 \bigg]}

\renewcommand{\epsilon}{\varepsilon}
\renewcommand{\l}{\ell}
\renewcommand{\d}{\,d}

\DeclareMathOperator{\arccosh}{arccosh}
\DeclareMathOperator{\arctanh}{arctanh}
\renewcommand{\tilde}{\widetilde}
\newcommand{\R}{\mathbb R}
\newcommand{\dd}[2][]{\frac{d #1}{d #2}}
\newcommand{\pp}[2][]{\frac{\partial #1}{\partial #2}}
\newcommand{\dfn}{\textbf}

\renewcommand{\bar}{\overline}
\renewcommand{\hat}{\widehat}


\ifxake
\NewEnviron{freeResponse}{}
\fi


\title{Rigid motions in central projection}

\begin{document}
\begin{abstract}
  Here we study rigid motions in central projection coordinates.
\end{abstract}
\maketitle


Suppose now we have a $K$-rigid motion given by a $K$-orthogonal matrix%
\[
M=\begin{bmatrix}
m_{11} & m_{12} & m_{13}\\
m_{21} & m_{22} & m_{23}\\
m_{31} & m_{32} & m_{33}
\end{bmatrix}.
\]
Let's convert this $K$-rigid motion to a rigid motion in central
projection coordinates. This new rigid motion will not necessarily be
a mapping defined by a matrix, so we'll have to use some new notation.

\begin{center}
  \begin{tikzcd}[column sep=large,row sep = huge,ampersand replacement=\&]
    (x_c,y_c)\in\R^2\ar[r,|->,"\mu_c",dashed] \ar[d,|->,"{K_c}"] \&  (\underline{x_c},\underline{y_c})\in\R^2\ar[d,|->,"{K_c}"] \\
    (x,y,z)\in\R^3\ar[r,|->,"M"] \&  (\underline{x},\underline{y},\underline{z})\in\R^3
  \end{tikzcd}
\end{center}


\begin{problem}
  Using the diagram above, explain why the formula for
  $(\underline{x_{c}},\underline{y_{c}})=\mu_{c}(x_{c},y_{c})$ is
  $$\mu_c(x_c,y_c)=\left(
    \frac{m_{11}x_c+m_{12}y_c+m_{13}}{m_{31}x_c+m_{32}y_c+m_{33}},
    \frac{m_{21}x_c+m_{22}y_c+m_{23}}{m_{31}x_c+m_{32}y_c+m_{33}}
    \right).$$
  \begin{freeResponse}
    Let's follow this diagram around, starting at the right-most
    corner. We want to start with $(x_c,y_c)$ and end with
    $(\underline{x_c},\underline{y_c}) = \mu_c(x_c,y_c)$. Since $\R^2$
    embeds canonically into $\R^2\times\{1\}$, we move to the
    upper-middle position almost immediately. Now, via multiplication
    by $\lambda$, $(x_c,y_c,1)$ maps to $(x,y,z)$. From here we may use
    matrix multiplication
    \[
    \begin{bmatrix}
      \underline{x} & \underline{y} & \underline{z}
    \end{bmatrix}
    =
    \begin{bmatrix}
    x & y & z
    \end{bmatrix}
    \cdot\begin{bmatrix}
    m_{11} & m_{12} & m_{13}\\
    m_{21} & m_{22} & m_{23}\\
    m_{31} & m_{32} & m_{33}
    \end{bmatrix}.
    \]
    Expanded out, this is
    \[
    \begin{bmatrix}
      x\cdot m_{11}+y\cdot m_{21}+z\cdot m_{31} & x\cdot m_{12}+y\cdot m_{22}+z\cdot m_{32} & 
      x\cdot m_{13}+y\cdot m_{23}+z\cdot m_{33}
    \end{bmatrix}.
    \]
    We are now at the bottom left-hand corner of our diagram. To move
    to the lower middle position, we multiply by $\underline{\lambda}^{-1}$, which
    is equivalent to dividing by $\underline{z}$. Hence our element
    becomes
     \[
    \begin{bmatrix}
      x\cdot \frac{m_{11}+y\cdot m_{21}+z\cdot m_{31}}{x\cdot m_{13}+y\cdot m_{23}+z\cdot m_{33}} &
      \frac{x\cdot m_{12}+y\cdot m_{22}+z\cdot m_{32}}{x\cdot m_{13}+y\cdot m_{23}+z\cdot m_{33}} & 1
    \end{bmatrix}.
    \]
    To write this in terms of $x_c$ and $y_c$, we must divide each
    numerator and denominator by $z$
    \[
    \begin{bmatrix}
      \frac{(x/z)\cdot m_{11}+(y/z)\cdot m_{21}+m_{31}}{(x/z)\cdot m_{13}+(y/z)\cdot m_{23}+m_{33}}
      &
      \frac{(x/z)\cdot m_{12}+(y/z)\cdot m_{22}+m_{32}}{(x/z)\cdot m_{13}+(y/z)\cdot m_{23}+m_{33}}
      & 1
    \end{bmatrix}.
    \]
    Now since $x/z = x_c$ and $y/z =y_c$, we may pull this back to
    $\R^2$ in the lower right-hand corner and write
    \[
    \mu_c(x_c,y_c) = \left(
    \frac{x_c\cdot m_{11} + y_c\cdot m_{21} + m_{31}}{x_c\cdot m_{13} + y_c\cdot m_{23} + m_{33}},
    \frac{x_c\cdot m_{12} + y_c\cdot m_{22} + m_{32}}{x_c\cdot m_{13} + y_c\cdot m_{23} + m_{33}}
    \right).
    \]
  \end{freeResponse}
\end{problem}


\subsection{Finding rigid motions in central projection}

Now let's use our new tool to convert $K$-rigid motions to rigid motions in central projection. 



\begin{problem}
  For any $K$, consider the $K$-rigid motion of $K$-geometry
  \[
  M_\theta=
  \begin{bmatrix}
    \cos\theta & -\sin\theta & 0\\
    \sin\theta & \cos\theta & 0\\
    0 & 0 & 1
  \end{bmatrix}.
  \]
  Can you describe geometrically what this mapping is doing to the
  points in central projection?
  
  \begin{freeResponse}
  In $K$-geometry $M_\theta$ rotates points around the $z$-axis so $M_\theta$ rotates points around the origin in central projection
  \end{freeResponse}
\end{problem}


\begin{problem}
  For any $K$, consider the $K$-rigid motion of $K$-geometry
  \[
  M_\theta=
  \begin{bmatrix}
    \cos\theta & -\sin\theta & 0\\
    \sin\theta & \cos\theta & 0\\
    0 & 0 & 1
  \end{bmatrix}.
  \]
  Convert this to a rigid motion in central projection.
  \begin{freeResponse}
   With $M$ = $M_\theta$,
  \begin{align*}
  \mu_c(x_c,y_c) &= \left( \frac{x_c\cdot m_{11} + y_c\cdot m_{21} + m_{31}}{x_c\cdot m_{13} + y_c\cdot m_{23} + m_{33}},
    \frac{x_c\cdot m_{12} + y_c\cdot m_{22} + m_{32}}{x_c\cdot m_{13} + y_c\cdot m_{23} + m_{33}} \right) \\
    &= \left( \frac{x_c\cdot \cos\theta - y_c\cdot \sin\theta}{1},
    \frac{x_c\cdot \sin\theta + y_c \cdot \cos\theta}{1} \right) \\
    &= \left( x_c\cdot \cos\theta - y_c\cdot \sin\theta,
    x_c\cdot \sin\theta+ y_c \cdot \cos\theta \right)
  \end{align*}
  \end{freeResponse}
\end{problem}


\begin{problem}
  Assuming $K > 0$, consider the $K$-rigid motion of the $R$-sphere
  \[
  N_\psi=
  \begin{bmatrix}
    \cos\psi & 0 & -R\cdot\sin\psi\\
    0 & 1 & 0\\
    R^{-1}\cdot\sin\psi & 0 & \cos\psi
  \end{bmatrix}.
  \]
Can you describe geometrically what this mapping is doing to the
points in central projection?  
\end{problem}


\begin{problem}
  Assuming $K > 0$, consider the $K$-rigid motion of the $R$-sphere
  \[
  N_\psi=
  \begin{bmatrix}
    \cos\psi & 0 & -R\cdot\sin\psi\\
    0 & 1 & 0\\
    R^{-1}\cdot\sin\psi & 0 & \cos\psi
  \end{bmatrix}.
  \]
  Convert this to a rigid motion in central projection.
  
  \begin{freeResponse}
   With $M$ = $N_\psi$,
  \begin{align*}
  \mu_c(x_c,y_c) &= \left( \frac{x_c\cdot n_{11} + y_c\cdot n_{21} + n_{31}}{x_c\cdot m_{13} + y_c\cdot n_{23} + n_{33}},
    \frac{x_c\cdot n_{12} + y_c\cdot n_{22} + n_{32}}{x_c\cdot n_{13} + y_c\cdot n_{23} + n_{33}} \right) \\
    &= \left( \frac{x_c\cdot \cos\psi - R\cdot \sin\psi}{x_c\cdot R^{-1} \cdot \sin\psi + \cos\psi},
    \frac{y_c}{x_c\cdot R^{-1} \cdot \sin\psi + \cos\psi} \right) 
  \end{align*}
  \end{freeResponse}

\end{problem}


\begin{problem}
  Assuming $K < 0$, consider the $K$-rigid motion of the $K$-surface
  \[
  N_\psi=
  \begin{bmatrix}
    \cosh\psi & 0 & |K|^{-1/2}\cdot\sinh\psi\\
    0 & 1 & 0\\
    |K|^{1/2}\cdot\sinh\psi & 0 & \cosh\psi
  \end{bmatrix}
  \]
  Can you describe geometrically what this mapping is doing to the
  points in central projection?
\end{problem}

  
\begin{problem}
  Assuming $K < 0$, consider the $K$-rigid motion of the $K$-surface
  \[
  N_\psi=
  \begin{bmatrix}
    \cosh\psi & 0 & |K|^{-1/2}\cdot\sinh\psi\\
    0 & 1 & 0\\
    |K|^{1/2}\cdot\sinh\psi & 0 & \cosh\psi
  \end{bmatrix}
  \]
  Convert this to a rigid motion in central projection.
  
    \begin{freeResponse}
   With $M$ = $N_\psi$,
  \begin{align*}
  \mu_c(x_c,y_c) &= \left( \frac{x_c\cdot n_{11} + y_c\cdot n_{21} + n_{31}}{x_c\cdot m_{13} + y_c\cdot n_{23} + n_{33}},
    \frac{x_c\cdot n_{12} + y_c\cdot n_{22} + n_{32}}{x_c\cdot n_{13} + y_c\cdot n_{23} + n_{33}} \right) \\
    &= \left( \frac{x_c\cdot \cosh\psi -  |K|^{-1/2}\cdot \sinh\psi}{x_c \cdot |K|^{1/2}\cdot \sinh\psi + \cosh\psi},
    \frac{y_c}{x_c\cdot R^{-1} \cdot \sin\psi + \cos\psi} \right) 
  \end{align*}
  \end{freeResponse}
\end{problem}


\begin{problem}
Summarize the results from this section. In particular, indicate which
results follow from the others.
\begin{freeResponse}
\end{freeResponse}
\end{problem}


\end{document}
