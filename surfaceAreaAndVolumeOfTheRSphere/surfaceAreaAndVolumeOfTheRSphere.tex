\documentclass[hints,handout,noauthor,nooutcomes,12pt]{ximera}

%\usepackage{microtype}
%\usepackage{tikz}
\usepackage{tkz-euclide}
%\usetkzobj{all}
\tikzstyle geometryDiagrams=[rounded corners=.5pt,ultra thick,color=blue!50!black]

\usepackage{tikz-cd}

\colorlet{penColor}{blue!50!black} % Color of a curve in a plot

%% \hypersetup{
%%     colorlinks = false,
%%     }


\tikzset{%% partial ellipse
    partial ellipse/.style args={#1:#2:#3}{
        insert path={+ (#1:#3) arc (#1:#2:#3)}
    }
}

\graphicspath{
{./}
{sphericalLunesAndTriangles/}
{hyperbolicLunesAndTriangles/}
{centralProjection/}
{stereographicProjection/}
{linesAnglesAndAreasInCentralProjection/}
{linesAnglesAndAreasInStereographicProjection/}
{stereographicProjection/}
{centralProjectionInHG/}
{stereographicProjectionInHG/}
{linesInSphericalGeometry/}
{linesInHyperbolicGeometry/}
{theArtOfEscher/}
}


\newcommand{\transpose}{\intercal}
\newcommand{\eval}[1]{\bigg[ #1 \bigg]}

\renewcommand{\epsilon}{\varepsilon}
\renewcommand{\l}{\ell}
\renewcommand{\d}{\,d}

\DeclareMathOperator{\arccosh}{arccosh}
\DeclareMathOperator{\arctanh}{arctanh}
\renewcommand{\tilde}{\widetilde}
\newcommand{\R}{\mathbb R}
\newcommand{\dd}[2][]{\frac{d #1}{d #2}}
\newcommand{\pp}[2][]{\frac{\partial #1}{\partial #2}}
\newcommand{\dfn}{\textbf}

\renewcommand{\bar}{\overline}
\renewcommand{\hat}{\widehat}


\ifxake
\NewEnviron{freeResponse}{}
\fi


\title{Surface area and volume of the sphere}
\begin{document}
\begin{abstract}
In this activity we begin to explore spherical geometry.
\end{abstract}
\maketitle


We are now going to study the geometry of the $R$-sphere in euclidean
$3$-space. This is the sphere of radius $R$ in normal $3$D space.

%% We will show why there is a factor of $1/3$ in many formulas for
%% volumes in $3$-dimensional euclidean geometry, just like there is a
%% factor of $1/2$ in many formulas for areas in $2$-dimensional
%% euclidean geometry.


In what follows we dare work with $\pi$ without computing it. Do not
judge us too harshly! After all, Euclid merely says (in modern language)
\begin{quote}
  \textbf{Euclid: Proposition II of Book XII}
  
  Given two circles, the ratio of their circumference to the square of
  their diameters are equal.
  \[
  \frac{(\text{area of a circle})}{(\text{diameter of a circle})^2} =
  \frac{(\text{area of another circle})}{(\text{diameter of another
      circle})^2}
  \]
\end{quote}

So let's make a definition.

\begin{definition}
  The \textbf{circumference of a circle of radius $\boldsymbol R$} is $2\pi R$.
\end{definition}

\subsection{Circumference and area of circles}

We begin with perhaps the most basic fact of all about circles in
euclidean geometry: If a circle has a circumference of $2\pi R$, then
its area is $\pi R^2$.

\begin{problem}
  If we are allowed to use calculus, we can deduce this immediately with
  \[
  \int_0^R 2\pi r \d r = \pi R^2.
  \]
  Explain using words and pictures why this makes sense.
\end{problem}


Now we will take a more foundational route.


\begin{problem}
Consider a circle of radius $R$ with $n$ equal inscribed triangles
meeting at the center.  Here we have shown the $n$th triangle:
\begin{image}
\begin{tikzpicture}[geometryDiagrams]
\draw [domain=-20:65] plot ({4*cos(\x)}, {4*sin(\x)});
\coordinate (A) at (4,0);
\coordinate (O) at (0,0);
\coordinate (C) at (2.83,2.83);
\draw[thin] (A)--(O)--(C)--(A);
\node[left] at (3.41,1.41) {$b_n$};
\node[right] at (4,0) {$(R,0)$};
\node[left] at (0,0) {$(0,0)$};
%\node[above] at (C) {$P$};
\tkzMarkAngle[size=0.8cm,thin](A,O,C)
\tkzLabelAngle[pos = 0.6](A,O,C){$\theta_n$}
\end{tikzpicture}
\end{image}
Find $\theta_n$ and $b_n$ in terms of $n$.
\end{problem}


\begin{problem}
Explain what the limit
\[
\lim_{n\to\infty} \left(n\cdot b_n\right)
\]
computes and compute the limit.
\end{problem}



The circle of radius $1$ has circumference $2\pi$. We can see that the
circle has area $\pi$ by cutting the circle in to tiny wedges:
\begin{image}
\begin{tikzpicture}[geometryDiagrams]
\draw (0,0) circle (2cm);
\foreach \coeff in {0,10,20,...,350}
         {
           \draw[thin] (0,0)--({2*cos(\coeff)},{2*sin(\coeff)});
         }
\end{tikzpicture}
\end{image}

Next we approximate a rectangle by rearranging the slices in the
picture. Note, this picture is to scale with the previous one.
\begin{image}
\begin{tikzpicture}[geometryDiagrams] % r= 2cm
  \foreach \coeff in {0,1,2,...,17}
   {
     \draw[thin] ({2*\coeff *sin(5)},0)--({(2*\coeff+ 2)* sin(5)},{2*cos(5)});
     \draw[thin] ({(2*\coeff + 2)*sin(5)},{2*cos(5)})--({(2+2*\coeff)*sin(5)},0); %one too many!
   }
   \draw[thin] ({2*18 *sin(5)},0)--({(2*18+ 2)* sin(5)},{2*cos(5)});
   \draw (0,0)--({2*18 *sin(5)},0);
   \draw ({2*sin(5)},2)--({2*19 *sin(5)},2);
\end{tikzpicture}
\end{image}
Note, while this is not truly a rectangle, as the slices become finer,
and finer, the shape will more, and more, closely approximate a
rectangle. This rectangle has perimeter $2+2\pi$, and hence is a
$1\times \pi$ rectangle. Thus the area of the rectangle is $\pi$ as is
the area of the initial circle.



Not content to give just one explanation, let's give another. The
techniques we use in this explanation will be used in later problems.

\begin{problem}
Again, consider the circle of radius $1$ with circumference $2\pi$.
Suppose we inscribe $n$ triangles (with one vertex at the center of
the circle), here we have shown the $n$th triangle:
\begin{image}
\begin{tikzpicture}[geometryDiagrams]
\draw [domain=-20:65] plot ({4*cos(\x)}, {4*sin(\x)});
\coordinate (A) at (4,0);
\coordinate (B) at (3.41,1.41);
\coordinate (O) at (0,0);
\coordinate (C) at (2.83,2.83);
\draw[thin] (A)--(O)--(C)--(A);
\draw[thin] (B)--(O);
\node at (2,1.1) {$h_n$};
\node[below] at (3.1,2) {$b_n$};

\tkzMarkRightAngle[thin](A,B,O)

\end{tikzpicture}
\end{image}
In each triangle, label $b_n$ and $h_n$, where $b_n$ is conceptualized
as a ``base'' and $h_n$ is a ``height.''  Consider the limit
\[
\lim_{n\to\infty} \left(\frac{n\cdot b_n\cdot h_n}{2}\right).
\]
Explain what this limit is computing and compute this limit.
\begin{freeResponse}
This limit is computing the sum of the areas of the triangles as $n$
goes to infinity. This will be the area of the circle of radius
$1$. Since $n\to 1$, and $n\cdot b_n\to 2\pi$
\begin{align*}
\lim_{n\to\infty} \frac{n\cdot b_n\cdot h_n}{2} &= \frac{2\pi\cdot 1}{2}\\
&=\pi.
\end{align*}
\end{freeResponse}
\end{problem}


\subsection{Surface area of spheres}

To compute the surface area of the sphere of radius $R$ in $3$-dimensional
euclidean space, we will show that its surface area is equal to the surface
area of something we can lay out flat. The argument for this goes way back to
the great physicist and mathematician, Archimedes of Alexandria, in the $2$nd
century B.C. To follow his argument, we have to begin by computing the area of
a `lamp shade' or `collar.' We think of a circular collar as in the figure
below:

\begin{image}
\begin{tikzpicture}[geometryDiagrams]
    \draw (2,2) arc (0:360:2cm and 0.5cm);
    \draw (-4,0) arc (180:360:4cm and 1cm);
    \draw[dashed,thin] (4,0) arc (8:172: 4cm and 1cm);
    \draw(-4,0)  -- (-2,2);
    \draw(4,0)   -- (2,2);
    %\node [left] at (2,-1) {$s$};
\end{tikzpicture}
\end{image}
Next, we will approximate this collar by using regular $n$-gons to
approximate the circles and arrangement of trapezoids to approximate
the sides. Below we see our collar approximate by two regular hexagons
and an arrangement of six congruent trapezoids. 
\begin{image}
\begin{tikzpicture}[geometryDiagrams]
  \draw (2,2) -- (1,2.5) -- (-1,2.5) -- (-2,2) -- (-1,1.5) -- (1,1.5) -- cycle;
  \draw (-4,0) -- (-1.5,-1) -- (1.5,-1) -- (4,0);
  \draw[dashed,thin] (4,0) -- (1.5,1)  -- (-1.5,1) -- (-4,0);
  \draw (-4,0)  -- (-2,2);
  \draw (4,0)   -- (2,2);
  \draw[thin] (-1,1.5) -- (-1.5,-1);
  \draw[thin] (1,1.5) -- (1.5,-1);
  \draw[thin] (-1,2.5) -- (-1.3,1.6);
  \draw[thin,dashed]  (-1.3,1.6) -- (-1.5,1);
  \draw[thin] (1,2.5) -- (1.3,1.6);
  \draw[thin,dashed]  (1.3,1.6) -- (1.5,1);

  \draw[thin] (0,-1) -- (0,1.5);
  
  \node[below] at (0,-1) {$b_6$};
  \node[above] at (0,2.5) {$t_6$};


  \node[right] at (0,0) {$h_6$};
  
\end{tikzpicture}
\end{image}
Generally we will let
\begin{itemize}
\item $b_{n}$ is the length of a side of the bottom regular $n$-gon,
\item $t_{n}$ is the length of a side of the top $n$-gon, and
\item $h_{n}$ is the slant height of the trapezoid.
\end{itemize}
Now each the trapezoid has area
\[
\left(  \frac{b_{n}+t_{n}}{2}\right)  \cdot h_{n}%
\]
and so the $n$th approximation of the collar has area
\[
n\cdot \left(  \frac{b_{n}+t_{n}}{2}\right)
\cdot h_{n}=\left(  \frac{n\cdot %
b_{n}+n\cdot t_{n}}{2}\right) \cdot h_{n}.
\]
As $n$ goes to infinity, the area of the approximation approaches the
area of the collar. Moreover, if we consider the collar once more,

\begin{image}
\begin{tikzpicture}[geometryDiagrams]
    \draw (2,2) arc (0:360:2cm and 0.5cm);
    \draw (-4,0) arc (180:360:4cm and 1cm);
    \draw[dashed,thin] (4,0) arc (8:172: 4cm and 1cm);
    \draw(-4,0)  -- (-2,2);
    \draw(4,0)   -- (2,2);
    \node[above right] at (3,1) {$s=$ the slant height};
    %\draw[thin] (0,2) -- (2,2);
    %\draw[thin] (0,0) -- (4,0);
    \node[above right] at (4,0) {$c_b=$ the circumference of the bottom};
    \node[above right] at (2,2) {$c_t=$ the circumference of the top};
\end{tikzpicture}
\end{image}
then
\[
\lim_{n\to \infty} n \cdot t_{n} =c_{t},\qquad \lim_{n\to \infty} h_{n}   =s,\qquad \lim_{n\to \infty} n\cdot b_{n}  =c_{b}.
\]


\begin{problem}
Now consider the collar
\begin{image}
\begin{tikzpicture}[geometryDiagrams]
    \draw (2,2) arc (0:360:2cm and 0.5cm);
    \draw (-4,0) arc (180:360:4cm and 1cm);
    \draw[dashed,thin] (4,0) arc (8:172: 4cm and 1cm);
    \draw(-4,0)  -- (-2,2);
    \draw(4,0)   -- (2,2);
    \node[above right] at (3,1) {$s$};
    \draw[thin] (0,2) -- (2,2);
    \draw[thin] (0,0) -- (4,0);
    \node[above] at (2,0) {$r_b$};
    \node at (1,2.2) {$r_t$};
\end{tikzpicture}
\end{image}
where
\begin{itemize}
\item $r_b$ is the radius of the circle defining the base of the collar,
\item $r_t$ is the radius of the circle defining the top of the collar,
\item $r_a$ is the average of $r_b$ and $r_t$,
\item $s$ is the slant height of the collar.
\end{itemize}
Explain why the area of the collar is%
\[
\pi \cdot \left( r_{b}+r_{t}\right)\cdot s.
\]
This can also be written as $2\pi\cdot r_{a}\cdot s$.
\end{problem}


\begin{theorem}
 The surface area of the sphere of radius $R$ is the same as the
 surface area of the label of the smallest can into which the sphere
 will fit.
\begin{image}
  \begin{tikzpicture}[geometryDiagrams]
    \draw (1.1,1.4) arc (51:129:1.75);
    \draw[dashed] (-1.1,1.4) arc (129:411:1.75);
    
    %%%% Equator
    \draw (-1.75,0) arc (180:360:1.75 and .5);
    \draw[dashed,thin] (1.75,0) arc (0:180:1.75 and .5);

    \draw (1.75,1.75) arc (0:360:1.75 and .5);

    %\draw (1.75,-1.75) arc (0:360:1.75 and .5);

    \draw (1.75,1.75) -- (1.75,-1.75);

    \draw (-1.75,1.75) -- (-1.75,-1.75);
    


    \draw (-1.75,-1.75) arc (180:360:1.75 and .5);
    \draw[dashed,thin] (1.75,-1.75) arc (0:180:1.75 and .5);
  \end{tikzpicture}
\end{image}
Namely the surface area of the sphere of radius $R$ is
\[
2\pi R\cdot 2R=4\pi R^{2}.
\]

\end{theorem}

\begin{problem}
Show why the above theorem is true.

\begin{hint}
Slice the picture above into $n$ horizontal slices. Approximate the
piece of the surface of the sphere between the $i$th pair of
successive slices by a collar $C_{i}$. Let $a\left( C_{i}\right) $
denote the area of $C_{i}$, let $r_{i}$ denote its average radius and
explain why the surface area of the sphere is
\[
\lim_{n\to \infty}\sum_{i=1}^{n} a(C_i).
\]
Explain further how we can conclude that the area of the sphere is
given by
\[%
\lim_{n\to \infty}\sum_{i=1}^{n} 2\pi\cdot r_{i}\cdot s_{i}.
\]

\end{hint}
\begin{hint}
Let $h_{i}$ denote the vertical height of the label on the can between
the $i$th pair of successive slices. Explain why the area of the label is exactly%
\[%
\sum_{i=1}^{n} 2\pi R \cdot h_{i}.
\]
\end{hint}
\begin{hint}
Explain why the relationship between each $r_{i}$, $s_{i}$ and $h_{i}$
is given by the picture below.
\begin{image}
\begin{tikzpicture}[geometryDiagrams]
\coordinate (A) at (0,1);
\coordinate (B) at (4,1);
\coordinate (C) at (4,-1);
\coordinate (D) at (.5,2);
\coordinate (E) at (-.5,0);
\coordinate (F) at (-.5,2);
\draw (A)--(B)--(C)--cycle;
\draw (D)--(E)--(F)--cycle;
\node[below] at (2,1) {$r_i$};
\node[below] at (2,0) {$R$};
\node[left] at (-.5,1) {$h_i$};
\node[above right] at (.2,1) {$s_i$};
\end{tikzpicture}
\end{image}
Now use facts about similar triangles to explain why
\[
r_{i}\cdot s_{i}=h_{i}\cdot R.
\]
\end{hint}
\end{problem}




\subsection{Volumes of pyramids and spheres}

Now we will deduce that the volume of the $R$-sphere is $(4/3) \pi
R^3$.

\begin{problem}
  If we are allowed to use calculus, we can deduce this immediately with
  \[
  \int_0^R 4\pi r^2 \d r = \frac{4\pi R^3}{3}.
  \]
  Explain using words and pictures why this makes sense.
\end{problem}

We will take a more basic route, and will start by looking at the
volumes of pyramids.

\begin{problem}
  Show that an $r\times r\times r$ cube can be constructed from three
  equal pyramids with an $r\times r$ square base. Conclude that the
  volume of each pyramid is $1/3$ the volume of the cube, namely
  \[
  \frac{r^{3}}{3}.
  \]

\begin{hint}
  Suppose the cube had a hollow interior and infinitely thin
  faces. Put your (infinitely tiny) eye at one vertex of the cube and
  look inside. How many faces of the cube can you see?
\end{hint}

\begin{freeResponse}
\begin{image}
\begin{tikzpicture}
\coordinate (A) at (0,0,0);
\coordinate (B) at (4,0,0);
\coordinate (C) at (4,4,0);
\coordinate (D) at (0,4,0);
\coordinate (A') at (0,0,4);
\coordinate (B') at (4,0,4);
\coordinate (C') at (4,4,4);
\coordinate (D') at (0,4,4);



\draw[fill=red,opacity=0.3] (A)--(B)--(C)--(D)--cycle; % base
\draw[fill=blue,opacity=0.3] (A')--(B')--(C')--(D')--cycle; % base
%% \draw[fill=orange,opacity=0.3] (B1)--(B2)--(peak);
%% \draw[fill=yellow,opacity=0.3] (B2)--(B3)--(peak);
%% \draw[fill=green,opacity=0.3] (B3)--(B4)--(peak);
%% \draw[fill=blue,opacity=0.3] (B4)--(B5)--(peak);
%% \draw[fill=purple,opacity=0.3] (B5)--(B1)--(peak);


\draw (A)--(B)--(C)--(D)--cycle;
\draw (A)--(A');
\draw (B)--(B');
\draw (C)--(C');
\draw (D)--(D');
\draw (A')--(B')--(C')--(D')--cycle;

\draw[ultra thick] (A)--(A')--(B')--(B)--cycle;
\draw[ultra thick] (A')--(D);
\draw[ultra thick] (B)--(D);
\draw[ultra thick, red] (B')--(D);
\draw[ultra thick, red] (A)--(D);
\end{tikzpicture}
\end{image}
\end{freeResponse}
\end{problem}

We next want to show why all pyramids with an $r\times r$ square base
and vertical altitude $r$ have the same volume. That is, if we put the
vertex of the pyramid anywhere in a plane parallel to the base and at
distance $r$, the volume is unchanged.

This fact is an example of \textit{Cavalieri's Principle}: Shearing a
figure parallel to a fixed direction does not change the
$n$-dimensional measure of an object in euclidean $n$-space. Think of
a stack of (very thin) books. We'll give a proof in euclidean $3$-space,
and we will use the coordinates $(\hat{x},\hat{y},\hat{z})$ to denote
this space. 






\begin{problem}
Show that Cavalieri's Principle is true for the pyramid using
multivariable calculus.

\begin{hint}
Put the base of the pyramid $P$ so that its vertices are $\left(
0,0,0\right) $, $\left( r,0,0\right) $, $\left( 0,r,0\right) $ and
$\left( r,r,0\right) $ in $3$-dimensional euclidean space. Consider
the transformation%
\[
\mathrm{shear}(x,y,z)=
\begin{bmatrix}
a+ f(z)\\
b+g(z)\\
1
\end{bmatrix}
\]
where $f:\R\to\R$ and $g:\R\to\R$ are functions and use the
3-dimensional change of variables formula,
$$\iiint_{T(P)} f(x,y,z)dxdydz=\iiint_P f(T(x,y,z))\lvert\det DT(x,y,z)\rvert dxdydz,$$
where
$$DT(x,y,z)=\begin{bmatrix}
\frac{\partial T}{\partial x} & \frac{\partial T}{\partial y} & \frac{\partial T}{\partial z}
\end{bmatrix}.$$
(Each partial derivative is itself a vector since $T(x,y,z)$ has three
components.)
\end{hint}



\end{problem}

\subsection{The magnification principle}

\textbf{The magnification principle:} If an object in Euclidean $n$-space is
magnified by factors of $r_{1},\ldots,r_{n}$, its $n$-dimensional
measure is multiplied by $r_{1}\cdots r_{n}$.

\begin{problem}
Use this magnification principle to justify the volume formula%
\[
(1/3)B\cdot h
\]
for any pyramid with rectangular base of area $B$ and vertical altitude $h$.
\end{problem}

\begin{problem}
Prove the magnification principle using multivariable calculus.

\begin{hint}
Consider the transformation%
\[
J(\vec x)=
\begin{bmatrix}
r_{1} & 0   & \dots & 0\\
0     & r_2 & \dots & 0\\
\vdots &\vdots &\ddots & \vdots \\
0     & 0   & \dots & r_{n}%
\end{bmatrix}\vec x.
\]
and use the change of variables formula.
\end{hint}
\end{problem}

Now suppose we have any pyramid%
\begin{image}
\begin{tikzpicture}
%\draw[thick,-stealth] (0,0,0)--(0,0,6); 
%\draw[thick,-stealth] (0,0,0)--(0,6,0);
%\draw[-stealth] (0,0,0)--(6,0,0);
\coordinate (B1) at (0,0,0);
\coordinate (B2) at (0,0,2);
\coordinate (B3) at (2,0,1);
\coordinate (B4) at (4,0,2);
\coordinate (B5) at (3,0,0);
\coordinate (peak) at (1,3,1);

\draw[fill=red,opacity=0.3] (B1)--(B2)--(B3)--(B4)--(B5)--cycle; % base
\draw[fill=orange,opacity=0.3] (B1)--(B2)--(peak);
\draw[fill=yellow,opacity=0.3] (B2)--(B3)--(peak);
\draw[fill=green,opacity=0.3] (B3)--(B4)--(peak);
\draw[fill=blue,opacity=0.3] (B4)--(B5)--(peak);
\draw[fill=purple,opacity=0.3] (B5)--(B1)--(peak);

\draw[ultra thick] (B2)--(B3)--(B4)--(B5)--(peak)--(B4)--(B3)--(peak)--(B2);
\end{tikzpicture}
\end{image}
with any shaped base of area $B$ and any vertical altitude $h$. We can
approximate the base as close as we want ($\varepsilon$-close) by
tiling its interior with rectangles. Taking the limit we find 
\[
\text{volume of a cone} =\left(  1/3\right)  \cdot B\cdot h.
\]
where $B$ is the area of the base of the cone. This formula holds regardless of the shape of the base of the cone.








\subsection{Relation between volume and surface area of a sphere}

Think of a disco-ball. Its surface is approximately a sphere, but that
surface is made up of tiny flat mirrors.

\begin{problem}\hfil
\begin{enumerate}
\item Explain why you can think of the disco-ball as being made up of
  pyramids, with each pyramid having base one of the tiny mirrors and
  vertex at the interior point $O$ at the center of the disco-ball.
\item Argue that the volume of the disco-ball is $\left( 1/3\right) $
  times the distance $h$ from a mirror to $O$ times the sum of the
  areas of all the mirrors.
\end{enumerate}
\end{problem}

\begin{problem}
Argue that, as the mirrors are made to be smaller and smaller,
\begin{enumerate}
\item the sum of the areas of the mirrors approaches the surface area of a sphere,
\item the distance $h$ approaches the radius $R$ of that sphere,
\item the volume of the disco-ball approaches the volume of the sphere.
\end{enumerate}
Conclude that, for a sphere of radius $R$ in euclidean $3$-space, the relation
between the volume $V$ of the sphere and the surface area $S$ of the sphere is
given by the formula%
\[
V=\frac{R\cdot S}{3}.
\]
\end{problem}

\begin{problem}
Use the work above to deduce a formula for the volume of a sphere.
\end{problem}




\begin{problem}
Summarize the results from this section. In particular, indicate which
results follow from the others.
\begin{freeResponse}
\end{freeResponse}
\end{problem}


\end{document}
