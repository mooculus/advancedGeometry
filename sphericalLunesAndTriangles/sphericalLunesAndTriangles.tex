\documentclass[newpage,hints,12pt,handout,nooutcomes,noauthor]{ximera}

%\usepackage{microtype}
%\usepackage{tikz}
\usepackage{tkz-euclide}
%\usetkzobj{all}
\tikzstyle geometryDiagrams=[rounded corners=.5pt,ultra thick,color=blue!50!black]

\usepackage{tikz-cd}

\colorlet{penColor}{blue!50!black} % Color of a curve in a plot

%% \hypersetup{
%%     colorlinks = false,
%%     }


\tikzset{%% partial ellipse
    partial ellipse/.style args={#1:#2:#3}{
        insert path={+ (#1:#3) arc (#1:#2:#3)}
    }
}

\graphicspath{
{./}
{sphericalLunesAndTriangles/}
{hyperbolicLunesAndTriangles/}
{centralProjection/}
{stereographicProjection/}
{linesAnglesAndAreasInCentralProjection/}
{linesAnglesAndAreasInStereographicProjection/}
{stereographicProjection/}
{centralProjectionInHG/}
{stereographicProjectionInHG/}
{linesInSphericalGeometry/}
{linesInHyperbolicGeometry/}
{theArtOfEscher/}
}


\newcommand{\transpose}{\intercal}
\newcommand{\eval}[1]{\bigg[ #1 \bigg]}

\renewcommand{\epsilon}{\varepsilon}
\renewcommand{\l}{\ell}
\renewcommand{\d}{\,d}

\DeclareMathOperator{\arccosh}{arccosh}
\DeclareMathOperator{\arctanh}{arctanh}
\renewcommand{\tilde}{\widetilde}
\newcommand{\R}{\mathbb R}
\newcommand{\dd}[2][]{\frac{d #1}{d #2}}
\newcommand{\pp}[2][]{\frac{\partial #1}{\partial #2}}
\newcommand{\dfn}{\textbf}

\renewcommand{\bar}{\overline}
\renewcommand{\hat}{\widehat}


\ifxake
\NewEnviron{freeResponse}{}
\fi


\title{Spherical lunes and triangles}
\begin{document}
\begin{abstract}
In this activity we explore the areas of lunes and triangles on the sphere.
For now we will assume that the equivalent of ``lines'' on the sphere are
\emph{great circles}: circles that cut the sphere exactly in half.  Later we will
see an explanation of why this is the right notion.
\end{abstract}
\maketitle

\subsection{Rigid motions}


Rigid motions are transformations that preserve lengths. This
means that the distance between two points $A$ and $B$ is the same,
both before and after the transformation is applied. Rigid motions are
central to geometry because \textit{congruences} can be explained in
terms of rigid motions: \textbf{Two objects are congruent if there is a rigid
motion mapping one to the other.} 



\begin{problem}
  Give an intuitive explanation why the transformation
  \[
  \begin{bmatrix}-x\\-y\\-z\end{bmatrix}=
    \begin{bmatrix}
      -1 & 0 & 0\\
      0 & -1 & 0\\
      0 & 0 & -1
    \end{bmatrix}
    \begin{bmatrix}x\\y\\z\end{bmatrix}
  \]
  preserves lengths, angles, and areas (and hence, is a rigid
  motion). Later, we will find a method to make this rigorous and
  general.
\end{problem}




\subsection{Spherical lunes}
While a three-sided triangle is the `simplest' closed convex polygon
in euclidean geometry, this is not the case for spherical geometry. In
this case, we have two-sided lunes.  In the picture we have drawn an
`$\alpha$-lune' on the $R$-sphere in euclidean $3$-space.
%% \begin{image}
%% \includegraphics[width=3in]{W13_3.png}%
%% \end{image}



\begin{center}
  \begin{tikzpicture}[geometryDiagrams,scale=1.7]  
      
    %% sphere
    \draw (1.75,0) arc (0:360:1.75);
    
    %%right line
    \draw[thin,dashed,rotate around={80:(0,0)}] (-1.75,0) arc (180:360:1.75 and .5);
    \draw[thin,rotate around={80:(0,0)}] (1.75,0) arc (0:180:1.75 and .5);     
    \begin{scope}
      \clip[rotate around={120:(0,0)}] (-1.75,0) arc (180:360:1.75 and .5);
      \draw[rotate around={80:(0,0)}] (1.75,0) arc (0:180:1.75 and .5);     
    \end{scope}
    \begin{scope}
      \clip (-1,1) rectangle (1,-2);
      \draw[rotate around={80:(0,0)}] (1.75,0) arc (0:180:1.75 and .5);     
    \end{scope}
    \draw[dashed,rotate around={80:(0,0)}] (-1.75,0) arc (180:233:1.75 and .5);
    

    
    %% left line
    \draw[thin,rotate around={120:(0,0)}] (-1.75,0) arc (180:360:1.75 and .5);
    \draw[thin,dashed,rotate around={120:(0,0)}] (1.75,0) arc (0:128:1.75 and .5);
    \begin{scope}
      \clip[rotate around={80:(0,0)}] (1.75,0) arc (0:180:1.75 and .5);
      \draw[rotate around={120:(0,0)}] (-1.75,0) arc (180:360:1.75 and .5); 
    \end{scope}
    \begin{scope}
      \clip (0,1) rectangle (1,-2);
      \draw[rotate around={120:(0,0)}] (-1.75,0) arc (180:360:1.75 and .5);     
    \end{scope}
    \draw[dashed,rotate around={120:(0,0)}] (-1.75,0) arc (180:127:1.75 and .5);

    \node at (-.15,.85) {$\alpha$};


    \begin{scope}
      \clip[rotate around={80:(0,0)}] (1.75,0) arc (0:180:1.75 and .5);     
      \clip[rotate around={120:(0,0)}] (-1.75,0) arc (180:360:1.75 and .5);

      \draw[thin] (-.15,1) circle (.3cm);
    \end{scope}
\end{tikzpicture}  
  \end{center}



The lune has two vertices. They are at opposite (antipodal) points on the
$R$-sphere, that is, the line in euclidean $3$-space that joins the two
vertices runs through the center of the sphere. The angle at a vertex of the
lune is $\alpha$ radians.







\begin{problem}
  Prove that the other vertex of of the lune is found by the rigid motion%
  \[
  \begin{bmatrix}-x\\-y\\-z\end{bmatrix}=
    \begin{bmatrix}
      -1 & 0 & 0\\
      0 & -1 & 0\\
      0 & 0 & -1
    \end{bmatrix}
    \begin{bmatrix}x\\y\\z\end{bmatrix}.
  \]
  \begin{hint}
    Let $x^2 + y^2 + z^2 = R^2$ be the equation for the
    $R$-sphere. Hence the vertices of our lune are solutions to
    simultaneous equations of the form:
    \begin{align*}
      x^2 + y^2 + z^2 &= R^2\\
      ax + by + cz &= 0\\
      a'x + b'y + c'z &= 0.
    \end{align*}
    Prove that if $(x,y,z)$ is a solution, so is $(-x,-y,-z)$.
  \end{hint}
\end{problem}





\begin{problem}
\label{67} Explain why the area of the $\alpha$-lune is $2\alpha
\cdot R^{2}$.
\end{problem}


\subsection{Spherical triangles}

If a triangle on the sphere of radius $R$ has interior angles with radian
measures $\alpha$, $\beta$, and $\gamma$, it can be covered three times by
lunes as shown in the figure below.%
%% \begin{image}
%% \includegraphics[width=3in]{W13_4.png}%
%% \end{image}





\begin{center}
  \begin{tikzpicture}[geometryDiagrams,scale=1.7]  
      
    %% sphere
    \draw (1.75,0) arc (0:360:1.75);
    

    \draw[thin,rotate around={10:(0,0)}] (-1.75,0) arc (180:360:1.75 and .5);
    \draw[thin,dashed,rotate around={10:(0,0)}] (1.75,0) arc (0:180:1.75 and .5);
    \begin{scope}
      \clip (-.705,-1) rectangle (1,2);
      \draw[rotate around={10:(0,0)}] (-1.75,0) arc (180:292:1.75 and .5);     
    \end{scope}




    
    \draw[thin,dashed,rotate around={70:(0,0)}] (-1.75,0) arc (180:360:1.75 and .5);
    \draw[thin,rotate around={70:(0,0)}] (1.75,0) arc (0:180:1.75 and .5);
    %% \begin{scope}
    %%   \clip[rotate around={120:(0,0)}] (-1.75,0) arc (180:360:1.75 and .5);
    %%   \clip[rotate around={10:(0,0)}] (-1.75,0) arc (180:360:1.75 and .5);
    %%   \draw[blue,rotate around={70:(0,0)}] (1.75,0) arc (0:130:1.75 and .5);     
    %% \end{scope}
    \begin{scope}
      \clip (-1,1.01) rectangle (1,-2);
      \draw[rotate around={70:(0,0)}] (1.75,0) arc (0:117:1.75 and .5);     
    \end{scope}
    



    
    \draw[thin,rotate around={120:(0,0)}] (-1.75,0) arc (180:360:1.75 and .5);
    \draw[thin,dashed,rotate around={120:(0,0)}] (1.75,0) arc (0:180:1.75 and .5);
    \begin{scope}
      \clip (-1,1.01) rectangle (1,-2);
      \draw[rotate around={120:(0,0)}] (1.75,0) arc (0:-112:1.75 and .5);     
    \end{scope}
\end{tikzpicture}  
  \end{center}
Notice that each lune has one vertex at a vertex of the triangle and
angle equal to that interior angle of the triangle.

\begin{problem}
Show that the area of the spherical triangle with angles $\alpha$,
$\beta$, and $\gamma$ is given by the formula%
\[
R^{2}\left(  \left(  \alpha+\beta+\gamma\right)  -\pi\right).
\]
\begin{hint}
From our previous work, we see that the triangle and its opposite have
the same area.  The three lunes cover the triangle three times. The
three opposite lunes cover the opposite triangle three times. If you
take all six lunes together, they cover each of the two triangles
three times and everything else exactly once.
\end{hint}
\end{problem}


\begin{problem}
Find a formula for the area of any spherical quadrilateral.

\end{problem}


\begin{problem}
Summarize the results from this section. In particular, indicate which
results follow from the others.
\begin{freeResponse}
\end{freeResponse}
\end{problem}

\end{document}
