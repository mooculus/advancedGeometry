\documentclass{ximera}

\usepackage{microtype}
\usepackage{tikz}
\usepackage{tkz-euclide}
\usetkzobj{all}
\tikzstyle geometryDiagrams=[ultra thick,color=blue!50!black]

\graphicspath{
{./}
{areasOnSpheresInEuclidean3Space/}
{centralProjection/}
{stereographicProjection/}
{centralProjectionInHG/}
{stereographicProjectionInHG/}
}


\newcommand{\transpose}{\intercal}
\renewcommand{\epsilon}{\varepsilon}
\renewcommand{\l}{\ell}
\renewcommand{\d}{\,d}

\newcommand{\R}{\mathbb R}


\renewcommand{\bar}{\overline}


%% \prerequisites{euclideanGeometry}
%% \outcome{circles}

\title{The cross-ratio and Ptolemy's Theorem}

\begin{document}
\begin{abstract}
In this activity, we do some basic projective geometry and learn
Ptolemy's Theorem.
\end{abstract}
\maketitle

In this section, we will study some basic ideas of \textit{projective
  geometry}. The main topic, one closely related to the notion of
perspective in painting, is called the \textit{cross-ratio}.  In
particular, we will use the cross-ratio to prove a famous mathematical
relationship, \textit{Ptolemy's Theorem}. We will then explore several
corollaries of Ptolemy's Theorem.



\section{The cross-ratio}

\begin{problem}\label{46}
In the diagram
\begin{image}
\begin{tikzpicture}[geometryDiagrams]
\coordinate (O) at ({2*cos(120)},{2*sin(120)});
\coordinate (A) at ({2*cos(190)},{2*sin(190)});
\coordinate (B) at ({2*cos(250)},{2*sin(250)});
\coordinate (C) at ({2*cos(330)},{2*sin(330)});

\draw (0,0) circle (2cm);
\draw (O) -- (C);
\draw (A) -- (B);
\draw (O) -- (B);
\draw (B) -- (C);
\draw (O) -- (A);


\tkzMarkAngle[size=.9cm,thin](A,O,B)
\tkzLabelAngle[pos = 0.7](A,O,B){$\alpha$}
\tkzMarkAngle[arc=ll,size=0.9cm,thin](B,O,C)
\tkzLabelAngle[pos = 0.7](B,O,C){$\beta$}

\tkzLabelPoints[above](O)
\tkzLabelPoints[left](A)
\tkzLabelPoints[below](B)
\tkzLabelPoints[right](C)

\end{tikzpicture}
\end{image}
show that
\[
\frac{\left\vert AB\right\vert }{\left\vert CB\right\vert }=\frac
{\sin\alpha}{\sin\beta}=\frac{\sin\left(  \angle
AOB\right)  }{\sin\left(  \angle COB\right)  }.
\]


%% \begin{hint}
%% Notice that
%% \[
%% m\left(  \angle BAO\right)  +m\left(\angle OCB\right)  =180^{\circ}%
%% \]
%% so that%
%% \[
%% \sin\left(  \angle BAO\right)  =\sin\left(\angle  OCB\right)  .
%% \]
%% Now use the Law of Sines.
%% \end{hint}
\begin{freeResponse}
By the Law of Sines we have that
\[
\frac{|AB|}{\sin\alpha} = \frac{|OB|}{\sin(\angle OAB)}\qquad\text{and}\qquad\frac{|BC|}{\sin\beta} = \frac{|OB|}{\sin(\angle OCB)}.
\]
Write 
\begin{align*}
m\left(  \angle BAO\right)  +m\left(\angle OCB\right)  &=180^{\circ}\\
m\left(  \angle BAO\right)  &=180^{\circ}-m\left(\angle OCB\right)  \\
\sin\left(\angle BAO\right)  &=\sin\left(180^{\circ}-m\left(\angle OCB\right)\right)  \\
\sin\left(\angle BAO\right)  &=\sin\left(180^{\circ}\right)\cos\left(\angle OCB\right) - \cos\left(180^{\circ}\right)\sin\left(\angle OCB\right)\\ 
\sin\left(\angle BAO\right)  &=\sin\left(\angle OCB\right).
\end{align*}
So  now we have that 
\[
\frac{|AB|}{\sin\alpha} = \frac{|OB|}{\sin(\angle OAB)}= \frac{|OB|}{\sin(\angle OCB)}=\frac{|BC|}{\sin\beta}. 
\]
Using algebra we find 
\[
\frac{\left\vert AB\right\vert }{\left\vert CB\right\vert }=\frac
{\sin\alpha}{\sin\beta}=\frac{\sin\left(  \angle
AOB\right)  }{\sin\left(  \angle COB\right)  }.
\]
\end{freeResponse}
\end{problem}

\begin{problem}
Note that if, in the above figure, $B$ moves along the
circle to the other side of $C$, it is still true that%
\[
\frac{\left\vert AB\right\vert }{\left\vert CB\right\vert }=\frac
{\sin\left(  \angle AOB\right)  }{\sin\left(  \angle
COB\right).  }%
\]
Carefully draw the diagram for this situation.
%% \begin{freeResponse}
%% By the Law of Sines we have that
%% \[
%% \frac{|AB|}{\sin(\angle AOB)} = \frac{|OB|}{\sin(\angle OAB)}\qquad\text{and}\qquad\frac{|CB|}{\sin(\angle COB)} = \frac{|OB|}{\sin(\angle OCB)}.
%% \]
%% However, 
%% \[
%% \angle OAB = \angle OCB
%% \]
%% So now we have that
%% \[
%% \frac{|AB|}{\sin(\angle AOB)} =\frac{|CB|}{\sin(\angle COB)}. 
%% \]
%% Using algebra we find 
%% \[
%% \frac{\left\vert AB\right\vert }{\left\vert CB\right\vert }=\frac
%% {\sin\left(  \angle AOB\right)  }{\sin\left(  \angle
%% COB\right)  }.
%% \]
%% \end{freeResponse}
\end{problem}






\begin{problem}\label{48}
In the diagram
\begin{image}
\begin{tikzpicture}[geometryDiagrams]
\coordinate (O) at ({2*cos(120)},{2*sin(120)});
\coordinate (A) at ({2*cos(190)},{2*sin(190)});
\coordinate (B) at ({2*cos(250)},{2*sin(250)});
\coordinate (C) at ({2*cos(330)},{2*sin(330)});

\coordinate (A') at (-3.21,-3);
\coordinate (B') at (-.59,-3);
\coordinate (C') at (3.73,-3);

\draw (0,0) circle (2cm);
\draw (O) -- (C');
\draw (A) -- (B);
\draw (O) -- (B');
\draw (B) -- (C);
\draw (O) -- (A');

\draw (-4,-3) -- (5,-3);


\tkzMarkAngle[size=.9cm,thin](A,O,B)
\tkzLabelAngle[pos = 0.7](A,O,B){$\alpha$}
\tkzMarkAngle[arc=ll,size=0.9cm,thin](B,O,C)
\tkzLabelAngle[pos = 0.7](B,O,C){$\beta$}

\tkzMarkAngle[size=0.7cm,thin](B',A',A)
\tkzLabelAngle[pos = 0.4](B',A',A){$\gamma$}

\tkzMarkAngle[size=0.9cm,thin](C,C',B')
\tkzLabelAngle[pos = 0.7](C,C',B'){$\delta$}

\tkzLabelPoints[above](O)
\tkzLabelPoints[left](A)
\tkzLabelPoints[below left](B)
\tkzLabelPoints[below](A',B',C')
\tkzLabelPoints[right](C)

\end{tikzpicture}
\end{image}
show that
\[
\frac{\left|A'B'\right|}{\left|C'B'\right|}=
\frac{\sin\alpha}{\sin\beta}\cdot\frac{\sin\delta}{\sin\gamma}=
\frac{\sin\left(\angle A'OB'\right)}{\sin\left(\angle C'OB'\right)}\cdot\frac{
\sin\left( \angle B'C'O\right)}{\sin\left(\angle B'A'O\right)}.
\]
%[MJG,266-267]
\begin{freeResponse}
By the Law of Sines we have that
\[
\frac{|A'B'|}{\sin \alpha} = \frac{|OB'|}{\sin \gamma}\qquad\text{and}\qquad\frac{|C'B'|}{\sin \beta} = \frac{|OB'|}{\sin\delta}.
\]
We now see that
\[
|A'B'| = \frac{|OB'|\sin\alpha}{\sin \gamma}\qquad\text{and}\qquad|C'B'| = \frac{|OB'|\sin\beta}{\sin\delta}.
\]
Finally, we see directly that 
\[
\frac{\left|A'B'\right|}{\left|C'B'\right|}=
\frac{\sin\alpha}{\sin\beta}\cdot\frac{\sin\delta}{\sin\gamma}=
\frac{\sin\left(\angle A'OB'\right)}{\sin\left(\angle C'OB'\right)}\cdot\frac{
\sin\left( \angle B'C'O\right)}{\sin\left(\angle B'A'O\right)}.
\]
\end{freeResponse}
\end{problem}


%% \begin{problem}\label{49}
%% Show that if, in the previous figure, $B'$ moves
%% along the line to the other side of $C'$, it is still true that%
%% \[
%% \frac{\left\vert A'B'\right\vert }{\left\vert C^{\prime
%% }B'\right\vert }=\frac{\sin\left(  \angle A'%
%% OB'\right)  }{\sin\left(  \angle C'OB'\right)
%% }\div\frac{\sin\left(  \angle B'A'O\right)
%% }{\sin\left(  \angle B'C'O\right)  }.
%% \]
%% \begin{freeResponse}
%% By the Law of Sines we have that
%% \[
%% \frac{|A'B'|}{\sin(\angle A'OB')} = \frac{|OB'|}{\sin(\angle B'A'O)}\qquad\text{and}\qquad\frac{|C'B'|}{\sin(\angle C'OB')} = \frac{|OB'|}{\sin(\angle B'C'O)}.
%% \]
%% We now see that
%% \[
%% |A'B'| = \frac{|OB'|\sin(\angle A'OB')}{\sin(B'A'O)}\qquad\text{and}\qquad|C'B'| = \frac{|OB'|\sin(\angle C'OB')}{\sin(\angle B'C'O)}.
%% \]
%% Again, we see directly that 
%% \[
%% \frac{\left\vert A'B'\right\vert }{\left\vert C^{\prime
%% }B'\right\vert }=\frac{\sin\left(  \angle A'%
%% OB'\right)  }{\sin\left(  \angle C'OB'\right)
%% }\div\frac{\sin\left(  \angle B'A'O\right)
%% }{\sin\left(  \angle B'C'O\right)  }.
%% \]
%% \end{freeResponse}
%% \end{problem}



\begin{problem}
Note that if, in the above figure, $B$ moves along the
circle to the other side of $C$, it is still true that%
\[
\frac{\left|A'B'\right|}{\left|C'B'\right|}=
\frac{\sin\alpha}{\sin\beta}\cdot\frac{\sin\delta}{\sin\gamma}=
\frac{\sin\left(\angle A'OB'\right)}{\sin\left(\angle C'OB'\right)}\cdot\frac{
\sin\left( \angle B'C'O\right)}{\sin\left(\angle B'A'O\right)}.
\]
Carefully draw the diagram for this situation.
\end{problem}

These problems allow us to define the cross-ratio of four points on a
circle.

\begin{definition}
For a sequence of four (ordered) points $A,B,C,$ and $D$ on a circle,
we define%
\[
\left(A:B:C:D\right)=
\frac{\left| AB\right|}{\left|CB\right|}\cdot
\frac{
\left|CD\right|
}{
\left|AD\right|
}%
\]
which we call the \textbf{cross-ratio of the ordered sequence of four
  points on a circle}.  Similarly for a sequence of four (ordered) points
$A',B',C',$ and $D'$ on a line, we define%
\[
\left(  A':B':C':D'\right)  =
\frac{\left| A'B'\right| }{\left| C'B'\right|
}\cdot
\frac{
\left|C'D'\right|
}{
\left|A'D'\right|
}
\]
which we call the \textbf{cross-ratio of the ordered sequence of the
  four points on a line}.
\end{definition}

%% Notice that the original definition of a \textit{cross-ratio} is just
%% a refinement of the definition of $\left( A':B':C':D'\right)$ just
%% above. In the original definition we are keeping track of the signs of
%% the terms in the quotients whereas $\left( A':B':C':D'\right)$ is
%% always non-negative.

\begin{problem}\label{50}
Show that, in the figure%
\begin{image}
\begin{tikzpicture}[geometryDiagrams]
\coordinate (O) at ({2*cos(120)},{2*sin(120)});
\coordinate (A) at ({2*cos(190)},{2*sin(190)});
\coordinate (B) at ({2*cos(250)},{2*sin(250)});
\coordinate (C) at ({2*cos(310)},{2*sin(310)});
\coordinate (D) at ({2*cos(340)},{2*sin(340)});

\coordinate (A') at (-3.21,-3);
\coordinate (B') at (-.59,-3);
\coordinate (C') at (2.31,-3);
\coordinate (D') at (4.64,-3);

\draw (0,0) circle (2cm);
\draw (O) -- (C');
\draw (A) -- (B);
\draw (A) -- (D);
\draw (C) -- (D);
\draw (O) -- (B');
\draw (B) -- (C);
\draw (O) -- (A');
\draw (O) -- (D');

\draw (-4,-3) -- (6,-3); % line projecting on



\tkzLabelPoints[above](O)
\tkzLabelPoints[left](A)
\tkzLabelPoints[below left](B)
\tkzLabelPoints[below](A',B',C',D')
\tkzLabelPoints[right](C)
\tkzLabelPoints[above right](D)

\end{tikzpicture}
\end{image}
we have the equality%
\[
\left(  A:B:C:D\right)  =\left(A':B':C':D'\right)  .
\]


\begin{hint}
Use the previous problems. 
\end{hint}

%What happens in if we move $B$ to the other side of $C$?
\begin{freeResponse}
By our previous work, 
\begin{align*}
(A':B':C':D') &=\frac{|A'B'|}{|C'B'|}\cdot\frac{|C'D'|}{|A'D'|}\\
&=\left(\frac{\sin(\angle A'OB')}{\sin(\angle C'OB')}\cdot 
\frac{
\sin(\angle B'C'O)
}{
\sin(\angle B'A'O)
}\right) 
\cdot\left(\frac{
\sin(\angle C'OD')
}{
\sin(\angle A'OD')
}\cdot\frac{\sin(\angle D'A'O)}{\sin(\angle D'C'O)}\right).
\end{align*}
Since 
\begin{align*}
m(\angle A'OB') &= m(\angle AOB),\\
m(\angle C'OB') &= m(\angle COB),\\
m(\angle A'OD') &= m(\angle AOD),\\
m(\angle C'OD') &= m(\angle COD),\\
m(\angle B'A'O) &= m(\angle D'A'O),
\end{align*}
and as we have shown, 
\[
\sin(\angle B'C'O) = \sin(\angle D'C'O)
\]
as 
\[
m(\angle B'C'O) + m(\angle D'C'O) = 180^\circ.
\]
With all of these substitutions, our ``large'' expression above becomes
\[
\frac{\sin(\angle AOB)}{\sin(\angle COB)} \cdot\frac{
\sin(\angle COD)
}{
\sin(\angle AOD)
} = (A:B:C:D). 
\]
This completes the proof. 
\end{freeResponse}
\end{problem}


We say that ``Cross-ratio is invariant under stereographic
projection.''







\section{Ptolemy's Theorem}

You can easily convince yourself with a few examples that, given four
non-collinear points $A,B,C$ and $D$ in the plane, it is not always
true that there is a circle that passes through all four. A famous
theorem of classical Euclidean geometry gives a condition when there
is a circle that passes through all four.

\begin{theorem}[Ptolemy] If the ordered sequence of points $A,B,C$ and $D$ lies on a circle,
\begin{image}
\begin{tikzpicture}[geometryDiagrams]
\coordinate (A) at ({2*cos(190)},{2*sin(190)});
\coordinate (B) at ({2*cos(250)},{2*sin(250)});
\coordinate (C) at ({2*cos(310)},{2*sin(310)});
\coordinate (D) at ({2*cos(340)},{2*sin(340)});

\draw (0,0) circle (2cm);
\draw (A) -- (B);
\draw (C) -- (D);
\draw (A) -- (D);
\draw (B) -- (C);

\draw[thin] (A) -- (C);
\draw[thin] (B) -- (D);

\tkzLabelPoints[left](A)
\tkzLabelPoints[below left](B)
\tkzLabelPoints[right](C)
\tkzLabelPoints[above right](D)

\end{tikzpicture}
\end{image}
then%
\[
\left\vert AC\right\vert \cdot\left\vert BD\right\vert
=\left\vert AD\right\vert \cdot\left\vert BC\right\vert
+\left\vert AB\right\vert \cdot\left\vert CD\right\vert
.
\]
That is, the product of the diagonals of the quadrilateral $ABCD$ is the sum
of the products of pairs of opposite sides.
\end{theorem}

\begin{proof}
We need to check that%
\[
\left\vert AC\right\vert \cdot\left\vert BD\right\vert
=\left\vert AD\right\vert \cdot\left\vert BC\right\vert
+\left\vert AB\right\vert \cdot\left\vert CD\right\vert
\]
or, what is the same, we need to check that%
\[
\frac{\left\vert AC\right\vert \cdot\left\vert
BD\right\vert }{\left\vert AD\right\vert \cdot\left\vert
BC\right\vert }=1+\frac{\left\vert AB\right\vert \text{\textperiodcentered
}\left\vert CD\right\vert }{\left\vert AD\right\vert \text{\textperiodcentered
}\left\vert BC\right\vert }.
\]
That is, we need to check that
\[
\left(  A:C:B:D\right)  =1+\left(  A:B:C:D\right)  .
\]
But by a previous problem this is the same as checking that%
\[
\left(  A':C':B':D'\right)  =1+\left(
A':B':C':D'\right)
\]
for the projection of the four points onto a line from a point $O$ on the
circle. But that is the same thing as showing that
\[
\frac{\left\vert A'C'\right\vert \text{\textperiodcentered
}\left\vert B'D'\right\vert }{\left\vert A'D^{\prime
}\right\vert \cdot\left\vert B'C^{\prime
}\right\vert }=1+\frac{\left\vert A'B'\right\vert
\cdot\left\vert C'D'\right\vert
}{\left\vert A'D'\right\vert \cdot \left\vert B'C'\right\vert }%
\]
which is the same thing as showing that%
\[
\left\vert A'C'\right\vert \text{\textperiodcentered
}\left\vert B'D'\right\vert =\left\vert A'D^{\prime
}\right\vert \cdot\left\vert B'C^{\prime
}\right\vert +\left\vert A'B'\right\vert
\cdot\left\vert C'D'\right\vert .
\]
\end{proof}


\begin{problem}
Check the last equality in the proof of Ptolemy's Theorem using high
school-algebra.
\begin{freeResponse}
\end{freeResponse}
\end{problem}

\begin{problem}
Use Ptolemy's Theorem to give another proof of the Pythagorean Theorem.
\begin{hint}
Let the four points in Ptolemy's Theorem form a rectangle.
\end{hint}
\begin{freeResponse}
\end{freeResponse}
\end{problem}


\begin{problem}
Prove the addition formula for sine:
\[
\sin(\alpha + \beta) = \sin(\alpha)\cos(\beta) +
\cos(\alpha)\sin(\beta)
\]
\begin{hint}
Consider this setup, 
\begin{image}
\begin{tikzpicture}[geometryDiagrams]
\coordinate (A) at ({2*cos(80)},{2*sin(80)});
\coordinate (B) at ({2*cos(180)},{2*sin(180)});
\coordinate (C) at ({2*cos(270)},{2*sin(270)});
\coordinate (D) at ({2*cos(0)},{2*sin(0)});

\draw (0,0) circle (2cm);
\draw (A) -- (B);
\draw (C) -- (D);
\draw (A) -- (D);
\draw (B) -- (C);

\draw[thin] (A) -- (C);
\draw[thin] (B) -- (D);

\tkzLabelPoints[above](A)
\tkzLabelPoints[left](B)
\tkzLabelPoints[below](C)
\tkzLabelPoints[right](D)

\tkzMarkAngle[size=0.7cm,thin](D,B,A)
\tkzLabelAngle[pos = 0.5](D,B,A){$\alpha$}

\tkzMarkAngle[arc=ll,size=0.8cm,thin](C,B,D)
\tkzLabelAngle[pos = 0.6](C,B,D){$\beta$}

\end{tikzpicture}
\end{image}
where $\bar{BD}$ is a unit diameter for the circle.
\end{hint}
\end{problem}


\begin{problem}
Prove the subtraction formula for sine:
\[
\sin(\alpha - \beta) = \sin(\alpha)\cos(\beta) -
\cos(\alpha)\sin(\beta)
\]
\begin{hint}
Consider this setup, 
\begin{image}
\begin{tikzpicture}[geometryDiagrams]
\coordinate (A) at ({2*cos(110)},{2*sin(110)});
\coordinate (B) at ({2*cos(180)},{2*sin(180)});
\coordinate (C) at ({2*cos(0)},{2*sin(0)});
\coordinate (D) at ({2*cos(65)},{2*sin(65)});

\draw (0,0) circle (2cm);
\draw (A) -- (B);
\draw (C) -- (D);
\draw (A) -- (D);
\draw (B) -- (C);

\draw[thin] (A) -- (C);
\draw[thin] (B) -- (D);

\tkzLabelPoints[above](A)
\tkzLabelPoints[left](B)
\tkzLabelPoints[right](C)
\tkzLabelPoints[above](D)

\tkzMarkAngle[size=0.7cm,thin](C,B,A)
\tkzLabelAngle[pos = 0.5](C,B,A){$\alpha$}

\tkzMarkAngle[arc=ll,size=1.1cm,thin](C,B,D)
\tkzLabelAngle[pos = 0.9](C,B,D){$\beta$}

\end{tikzpicture}
\end{image}
where $\bar{BC}$ is a unit diameter for the circle.
\end{hint}
\end{problem}

Both of the above formulas were very important to Ptolemy. You see,
Ptolemy wrote a book called the \textit{Almagest}. The
\textit{Almagest} was a book that contained all the current
information about the stars, including information on how a potential
reader could reproduce their observations and conclusions. A key
technical hurtle that needed to be resolved was the computation of
sine and cosine. The two formulas above, were the key in this endeavor.


%% Given any three non-collinear points in the Euclidean plane, there is
%% one and only one circle that passes through the three points. 

%% \begin{problem}
%% Show that, given any three non-collinear points in the Euclidean
%% plane, there is a unique circle passing through the three points.

%% \begin{hint}
%% Show that the center of the circle must be the intersection of
%% the perpendicular bisectors of any two of the sides of the triangle
%% whose vertices are the three given points.
%% \end{hint}

%% \begin{freeResponse}
%% \end{freeResponse}

%% \end{problem}

%% But how about four points in the plane, no three of which are
%% collinear?

%% \begin{problem}
%% \begin{enumerate}\hfil
%% \item Draw four points in the Euclidean plane, no $3$ of which are collinear, that cannot lie on a single circle.
%% \item Draw four points in the Euclidean plane that do lie on a single
%% circle.
%% \end{enumerate}
%% \end{problem}




\begin{problem}
Summarize the results from this section. In particular, indicate which
results follow from the others.
\begin{freeResponse}
\end{freeResponse}
\end{problem}









\end{document}
