\documentclass[newpage,hints,handout]{ximera}

%\usepackage{microtype}
%\usepackage{tikz}
\usepackage{tkz-euclide}
%\usetkzobj{all}
\tikzstyle geometryDiagrams=[rounded corners=.5pt,ultra thick,color=blue!50!black]

\usepackage{tikz-cd}

\colorlet{penColor}{blue!50!black} % Color of a curve in a plot

%% \hypersetup{
%%     colorlinks = false,
%%     }


\tikzset{%% partial ellipse
    partial ellipse/.style args={#1:#2:#3}{
        insert path={+ (#1:#3) arc (#1:#2:#3)}
    }
}

\graphicspath{
{./}
{sphericalLunesAndTriangles/}
{hyperbolicLunesAndTriangles/}
{centralProjection/}
{stereographicProjection/}
{linesAnglesAndAreasInCentralProjection/}
{linesAnglesAndAreasInStereographicProjection/}
{stereographicProjection/}
{centralProjectionInHG/}
{stereographicProjectionInHG/}
{linesInSphericalGeometry/}
{linesInHyperbolicGeometry/}
{theArtOfEscher/}
}


\newcommand{\transpose}{\intercal}
\newcommand{\eval}[1]{\bigg[ #1 \bigg]}

\renewcommand{\epsilon}{\varepsilon}
\renewcommand{\l}{\ell}
\renewcommand{\d}{\,d}

\DeclareMathOperator{\arccosh}{arccosh}
\DeclareMathOperator{\arctanh}{arctanh}
\renewcommand{\tilde}{\widetilde}
\newcommand{\R}{\mathbb R}
\newcommand{\dd}[2][]{\frac{d #1}{d #2}}
\newcommand{\pp}[2][]{\frac{\partial #1}{\partial #2}}
\newcommand{\dfn}{\textbf}

\renewcommand{\bar}{\overline}
\renewcommand{\hat}{\widehat}


\ifxake
\NewEnviron{freeResponse}{}
\fi


\title{Changing coordinates}
\begin{document}
\begin{abstract}
Now we will change coordinates.
\end{abstract}
\maketitle

\begin{listOutcomes}
 \item Define $K$-warped space
 \item Change from Euclidean coordinates to $K$-warped space coordinates
 \item Prove facts about the angle between vectors and the area of a parallelogram in $K$-warped space
\end{listOutcomes}

\section{Bringing the North Pole of the $R$-sphere to $\left(
0,0,1\right)  $}

We are now ready to change coordinates on Euclidean $3$-space so that we can fill
up that space  with plane geometry and all the spherical and hyperbolic
geometries.  We have reserved the notation $(x,y,z)$ for these new coordinates
that we will put on the `same' objects we have been studying in Euclidean
$(\hat{x},\hat{y},\hat{z})$-coordinates. These new coordinates will be chosen to
keep the north and south poles from going to infinity as the radius $R$ of a
sphere increases without bound. In these now $(x,y,z)$-coordinates the sphere of
radius $R$ will be given by the equation
\[
K(x^{2}+y^{2})+z^{2}=1
\]
where $K=1/R^2$. Notice that the above equation has solutions even when $K$ is negative. It is on those solution sets that hyperbolic geometries will live. So this change of
viewpoint will eventually let us go hyperbolic or, in the language of Buzz
Lightyear, will let $R$ go `to infinity and beyond.' The
idea will be like the change from rectangular to polar coordinates for the
plane that you encountered in calculus, only easier. 

We are now ready to introduce this slightly different set of coordinates for
$\mathbb{R}^{3}$, three-dimensional Euclidean space. To understand a bit better why we are doing this,
suppose we are standing at the North Pole%
\[
N=(0,0,R)
\]
of the sphere%
\[
\hat{x}^{2}+\hat{y}^{2}+\hat{z}^{2}=R^{2} %\label{4}%
\]
of radius $R$. As $R$ increases, but we stay our same size, the sphere
around us becomes more and more like a flat, plane surface. However it
can never get completely flat because we are zooming out the positive
$\hat{z}$-axis and we would have to be `at infinity' for our surface
to become exactly flat. We remedy that unfortunate situation by
considering another copy of $\mathbb{R}^{3}$, that we will call
\dfn{$\boldsymbol{K}$-warped space}, whose coordinates we denote as
$\left( x,y,z\right) $.  We make the following rule in order to pass
between the two $\mathbb{R}^{3}$'s:%
\begin{align*}
\hat{x}  &  =x\\
\hat{y}  &  =y\\
\hat{z}  &  =Rz.
\end{align*}
We think of the $\left( x,y,z\right) $-coordinates as simply being a
different set of addresses for the points in Euclidean space. For
example,
\[
\left(x,y,z\right)  =\left(0,0,1\right)
\]
tells me that the point in Euclidean space that I'm referring to is%
\[
\left(\hat{x},\hat{y},\hat{z}\right) =\left( 0,0,R\right)= N.
\]
Continuing with this ``change of addresses'' the sphere of radius $R$
in Euclidean space is given by
\begin{align*}
R^{2} & =\hat{x}^{2}+\hat{y}^{2}+\hat{z}^{2}\\ &
=x^{2}+y^{2}+R^{2}z^{2}
\end{align*}
that is, by the equation
\[
1=\frac{1}{R^{2}}\left(  x^{2}+y^{2}\right)  +z^{2}. %\label{5}%
\]
\begin{definition}
  For the surface defined by
  \[
  1=\frac{1}{R^{2}}\left(  x^{2}+y^{2}\right)  +z^{2}. %\label{5%}
  \]
The quantity $K=\frac{1}{R^{2}}$ is called the \dfn{curvature} of the
$R$-sphere.
\end{definition}

\begin{problem}
  What happens to the surface when $K$ goes to $0$?  How does this relate to the
  colloquial sense of ``curvature''?
  
\begin{freeResponse}
When K goes to zero the surface goes to $z^{2} =1$.  A large sphere is not as curvy as a smaller sphere. As the radius increases, the surface gets flatter so the curvature gets smaller and vice versa.  When the curvature goes to zero, the surface becomes two copies of the plane which makes sense because a plane has zero curvature.  
\end{freeResponse}

\end{problem}

\begin{problem}\hfil
\begin{enumerate}
\item Sketch the solution set in $\left(  x,y,z\right)  $-coordinates
representing the sphere%
\[
R^{2}=\hat{x}^{2}+\hat{y}^{2}+\hat{z}^{2}=1.
\]

\item Sketch the solution set in $\left(  x,y,z\right)  $-coordinates
representing the sphere%
\[
R^{2}=\hat{x}^{2}+\hat{y}^{2}+\hat{z}^{2}=10^{2}.
\]

\item Sketch the solution set in $\left(  x,y,z\right)  $-coordinates
representing the sphere%
\[
R^{2}=\hat{x}^{2}+\hat{y}^{2}+\hat{z}^{2}=10^{-2}.
\]
\end{enumerate}
%
%\begin{hint}
%curvatureOfRsphere Geogebra
%\end{hint}
\end{problem}

\section{Formulas for Euclidean lengths, angles, and areas in terms of $(x,y,z)$-coordinates}

To prepare ourselves to do hyperbolic geometry, which (in some sense)
has no satisfactory model in Euclidean space, we will `practice' by
doing spherical geometry (which \textit{does} have a completely
satisfactory model in Euclidean space) using these `slightly strange'
$\left( x,y,z\right) $-coordinates. Gradually throughout this course
we will discover that the same rules that govern spherical geometry,
expressed in $\left( x,y,z\right) $-coordinates, also govern flat and
hyperbolic geometry! In all three cases, the surface in $\left(
x,y,z\right) $-coordinates that we will study is%
\[
1=K\left(x^{2}+y^{2}\right)+z^{2} .
\]
If $K>0$, the geometry we will be studying is the geometry of the the
Euclidean sphere of radius%
\[
R=\frac{1}{\sqrt{K}}.
\]
If $K=0$ we will be studying flat (plane) geometry. If $K<0$, we will be
studying hyperbolic geometry. 

In short, we want to use $\left(  x,y,z\right)  $-coordinates to compute with,
but we want lengths and angles to be the usual Euclidean ones in $\left(
\hat{x},\hat{y},\hat{z}\right)  $-coordinates.

\begin{problem}
  Suppose we have a curve $\hat\gamma$ in Euclidean space and we think of it as a
  composition of a curve $\gamma$ in $K$-warped space with a transformation.  In
  other words, we're looking at a diagram
  \begin{image}
  \begin{tikzpicture}
    \node{
  \begin{tikzcd}[ampersand replacement=\&]
    t \ar[r,mapsto,"\gamma"] \ar[rr,mapsto,bend right=20,swap,"\hat\gamma"] \&  (x(t),y(t),z(t))\ar[r,mapsto,"{\left[\begin{smallmatrix}
            \hat x(x,y,z)\\
            \hat y(x,y,z)\\
            \hat z(x,y,z)
          \end{smallmatrix}\right]}"
    ] \& \hat\gamma(t)=(\hat x(t),\hat y(t),\hat z(t))
    \end{tikzcd}};
  \end{tikzpicture}
\end{image}

\begin{image}
    \begin{tikzpicture}
    \node{
  \begin{tikzcd}[ampersand replacement=\&, column sep = huge]
    \R \ar[r,"\gamma"] \ar[rr,bend right=20,swap,"\hat\gamma"] \&  K-warped\ space \ar[r,"{\left[\begin{smallmatrix}
            \hat x(x,y,z)\\
            \hat y(x,y,z)\\
            \hat z(x,y,z)
          \end{smallmatrix}\right]}"
    ] \& Euclidean\ geometry
    \end{tikzcd}};
  \end{tikzpicture}
\end{image}

  Use the chain rule to compute
  \[
  \dd[\hat x]{t}, \qquad \dd[\hat y]{t}, \qquad \dd[\hat z]{t},
  \]
  in terms of $\dd[x]{t}$, $\dd[y]{t}$, $\dd[z]{t}$, $\pp[\hat x]{x}$, $\pp[\hat y]{x}$, $\pp[\hat z]{x}$,
  $\pp[\hat x]{y}$, $\pp[\hat y]{y}$, $\pp[\hat z]{y}$, $\pp[\hat x]{z}$, $\pp[\hat y]{z}$, and $\pp[\hat z]{z}$. 
  \begin{hint}
    Recall that if $F$ is a differentiable function of $x$, $y$, and
    $z$; and if $x$, $y$, and $z$ are all differentiable functions of
    $t$, then the chain rule states
    \[
    \dd[F]{t} = \nabla F \cdot
    \begin{bmatrix}
      \dd[x]{t} & \dd[y]{t} & \dd[z]{t}
    \end{bmatrix}^\transpose.
    \]
  \end{hint}
  \begin{freeResponse}
    Write
    \begin{align*}
      \dd[\hat x]{t} &= \begin{bmatrix} \pp[\hat x]{x} & \pp[\hat x]{y} & \pp[\hat x]{z} \end{bmatrix} \cdot \begin{bmatrix} \dd[x]{t} & \dd[y]{t} & \dd[z]{t} \end{bmatrix}^\transpose
      = \pp[\hat x]{x}\cdot\dd[x]{t} + \pp[\hat x]{y}\cdot\dd[y]{t} + \pp[\hat x]{z}\cdot\dd[z]{t}, \\
      \dd[\hat y]{t} &= \begin{bmatrix} \pp[\hat y]{x} & \pp[\hat y]{y} & \pp[\hat y]{z} \end{bmatrix} \cdot \begin{bmatrix} \dd[x]{t} & \dd[y]{t} & \dd[z]{t} \end{bmatrix}^\transpose
      = \pp[\hat y]{x}\cdot\dd[x]{t} + \pp[\hat y]{y}\cdot\dd[y]{t} + \pp[\hat y]{z}\cdot\dd[z]{t}, \\
      \dd[\hat z]{t} &= \begin{bmatrix} \pp[\hat z]{x} & \pp[\hat z]{y} & \pp[\hat z]{z} \end{bmatrix} \cdot \begin{bmatrix} \dd[x]{t} & \dd[y]{t} & \dd[z]{t} \end{bmatrix}^\transpose
      = \pp[\hat z]{x}\cdot\dd[x]{t} + \pp[\hat z]{y}\cdot\dd[y]{t} + \pp[\hat z]{z}\cdot\dd[z]{t}.
    \end{align*}
  \end{freeResponse}
\end{problem}

\begin{problem}
  With the same setting as in the previous problem, rewrite the result
  of your computation in matrix notation to find $D_K$ such that
\[
\begin{bmatrix}
d\hat{x}/dt \\ d\hat{y}/dt \\ d\hat{z}/dt%
\end{bmatrix}
=D_K \cdot \begin{bmatrix}
dx/dt \\ dy/dt \\ dz/dt
\end{bmatrix}
\]
in terms of $\pp[\hat x]{x}$, $\pp[\hat y]{x}$, $\pp[\hat z]{x}$,
$\pp[\hat x]{y}$, $\pp[\hat y]{y}$, $\pp[\hat z]{y}$, $\pp[\hat x]{z}$,
$\pp[\hat y]{z}$, and $\pp[\hat z]{z}$.

\begin{freeResponse}
\[
\begin{bmatrix}
\frac{d\hat{x}}{dt} & \frac{d\hat{y}}{dt} & \frac{d\hat{z}}{dt}%
\end{bmatrix}
=
\begin{bmatrix}
\frac{dx}{dt} & \frac{dy}{dt} & \frac{dz}{dt}%
\end{bmatrix}\cdot
\begin{bmatrix}
\pp[\hat x]{x} & \pp[\hat y]{x} & \pp[\hat z]{x} \\
\pp[\hat x]{y} & \pp[\hat y]{y} & \pp[\hat z]{y} \\
\pp[\hat x]{z} &\pp[\hat y]{z} & \pp[\hat z]{z}
\end{bmatrix}
\]
\[
D_K = 
\begin{bmatrix}
\pp[\hat x]{x} & \pp[\hat y]{x} & \pp[\hat z]{x} \\
\pp[\hat x]{y} & \pp[\hat y]{y} & \pp[\hat z]{y} \\
\pp[\hat x]{z} &\pp[\hat y]{z} & \pp[\hat z]{z}
\end{bmatrix}
\]

\end{freeResponse}
\end{problem}


\begin{problem}
  Use the previous problems and the relationship between Euclidean and
  $(x,y,z)$-coordinates to show that
  \[
  \left[\dd[\hat{\gamma}]{t}\right] =
  \begin{bmatrix}
    1 & 0 & 0\\
    0 & 1 & 0\\
    0 & 0 & R
  \end{bmatrix}\cdot \left[ \dd[\gamma]{t}\right].
  \]
  
\begin{freeResponse}
Since we have 
\begin{align*}
\hat{x}  &  =x\\
\hat{y}  &  =y\\
\hat{z}  &  =Rz
\end{align*}
then, 
\[
D_K = 
 \begin{bmatrix}
    1 & 0 & 0\\
    0 & 1 & 0\\
    0 & 0 & R
  \end{bmatrix}.
  \]
  Therefore, 
  \[
 \left[\dd[\hat{\gamma}]{t}\right] 
= \begin{bmatrix}
\frac{d\hat{x}}{dt} & \frac{d\hat{y}}{dt} & \frac{d\hat{z}}{dt}%
\end{bmatrix}
= \left[ \dd[\gamma]{t}\right] \cdot
  \begin{bmatrix}
    1 & 0 & 0\\
    0 & 1 & 0\\
    0 & 0 & R
  \end{bmatrix}.
\]
\end{freeResponse}

\end{problem}

This last computation shows that if
\[
\hat{\mathbf v}_{1}=\begin{pmatrix}\hat{a}_{1} \\ \hat{b}_{1} \\ \hat{c}_{1}\end{pmatrix}
\qquad\text{and}\qquad
\hat{\mathbf v}_{2} =\begin{pmatrix}\hat{a}_{2} \\ \hat{b}_{2} \\ \hat{c}_{2}\end{pmatrix}
\]
are vectors tangent to a curve in $(\hat{x},\hat{y},\hat{z})$-coordinates and
\[
\mathbf{v}_{1}=\begin{pmatrix}a_{1} \\ b_{1} \\ c_{1}\end{pmatrix}
\qquad\text{and}\qquad
\mathbf{v}_{2} =\begin{pmatrix}a_{2} \\ b_{2} \\ c_{2}\end{pmatrix}
\]
are their transformations into $(x,y,z)$-coordinates, then
\[
\hat{\mathbf v}_{1}=
\begin{bmatrix}
1 & 0 & 0\\
0 & 1 & 0\\
0 & 0 & R
\end{bmatrix}\mathbf{v}_{1}
\qquad\text{and}\qquad
\hat{\mathbf v}_{2}=
\begin{bmatrix}
1 & 0 & 0\\
0 & 1 & 0\\
0 & 0 & R
\end{bmatrix}\mathbf{v}_{2} 
\]
and%
\begin{align*}
\hat{\mathbf v}_{1}\bullet\hat{\mathbf v}_{2}
=\hat{\mathbf v}_{1}^\transpose \cdot \hat{\mathbf v}_{2} 
&= 
\left(\begin{bmatrix}
1 & 0 & 0\\
0 & 1 & 0\\
0 & 0 & R
\end{bmatrix}\mathbf{v}_{1}\right)^\transpose
\left(
\begin{bmatrix}
1 & 0 & 0\\
0 & 1 & 0\\
0 & 0 & R
\end{bmatrix}\mathbf{v}_{2}\right)\\
&=\mathbf{v}_{1}^\transpose
\begin{bmatrix}
1 & 0 & 0\\
0 & 1 & 0\\
0 & 0 & R
\end{bmatrix}
\begin{bmatrix}
1 & 0 & 0\\
0 & 1 & 0\\
0 & 0 & R
\end{bmatrix}
\mathbf{v}_{2}\\
&=\mathbf{v}_{1}^\transpose
\begin{bmatrix}
1 & 0 & 0\\
0 & 1 & 0\\
0 & 0 & K^{-1}%
\end{bmatrix}
\mathbf{v}_{2}.
\end{align*}
Hence:
\begin{center}
\textbf{We can compute the Euclidean dot product without ever referring to Euclidean coordinates!}
\end{center}
We incorporate that fact into the following definition.

\begin{definition}
The \textbf{$\boldsymbol{K}$-dot-product} of vectors:%
\begin{align*}
\mathbf{v}_{1}\bullet_{K}\mathbf{v}_{2}  &=\mathbf{v}_{1}^\transpose
\begin{bmatrix}
1 & 0 & 0\\
0 & 1 & 0\\
0 & 0 & K^{-1}%
\end{bmatrix}
\mathbf{v}_{2}\\
&=
\begin{bmatrix}
a_{1} & b_{1} & c_{1}%
\end{bmatrix}
\begin{bmatrix}
1 & 0 & 0\\
0 & 1 & 0\\
0 & 0 & K^{-1}%
\end{bmatrix}
\begin{bmatrix}
a_{2}\\
b_{2}\\
c_{2}%
\end{bmatrix}.
\end{align*}

\end{definition}

\subsection{Computing length}

\begin{problem}
  Show that if a vector is given to us in $(x,y,z)$-coordinates as%
\[
\mathbf{v}=(a,b,c)^\transpose,
\]
then the length of its image in Euclidean space is given by
\[
|\mathbf{v}|_K=\sqrt{\mathbf{v} \bullet_K \mathbf{v}}.
\]

\begin{freeResponse}
$\hat{\mathbf v} = \left(a,b,Rc \right)$ is the image of $V$ in Euclidean space. Then,
\begin{align*}
|V|_K = |\hat{\mathbf v}| = \sqrt{\hat{\mathbf v}\bullet\hat{\mathbf v}} 
&= \sqrt{a^{2} + b^{2} + R^{2}c^{2}} \\
&= \begin{bmatrix}
a & b & c%
\end{bmatrix}
\begin{bmatrix}
1 & 0 & 0\\
0 & 1 & 0\\
0 & 0 & K^{-1}%
\end{bmatrix}
\begin{bmatrix}
a\\
b\\
c%
\end{bmatrix} \\
&= \sqrt{V\bullet_K V}
\end{align*}
\end{freeResponse} 
\end{problem}

\begin{problem}
  Consider all vectors in $K$-warped space with their tips on the surface
  \[
  1=K(x^2+y^2)+z^2
  \]
  and their tails at the origin. What can you say about the length of these
  vectors? What does this tell you about the surface for all values of $K>0$?
  
\begin{freeResponse}
Let $V = \left(x,y,z\right)$ be a vector with its tail at the origin and tip on the surface above. Now write the equation of the surface as,
\begin{align*}
\frac{1}{K} &= x^{2} + y^{2} + \frac{1}{K} \cdot z^{2} \\
&= \begin{bmatrix}
	x & y & z
	\end{bmatrix}
\begin{bmatrix}
	1 & 0 & 0 \\
	0 & 1 & 0 \\
	0 & 0 & K^{-1}
	\end{bmatrix}
\begin{bmatrix}
	x \\
	y \\
	z
	\end{bmatrix}\\
&= V \bullet_{K} V
\end{align*}
Then $|V|_{K} = \sqrt{V \bullet_{K} V} = \frac{1}{\sqrt{K}} = |R|$. This tells us that for all values of $K$, except $K=0$, the surface is a sphere. 
\end{freeResponse}

\end{problem}

%% So, suppose we have a curve on the $R$-sphere in Euclidean space
%% but it is given to us in $X(t)=(x(t),y(t),z(t))$-coordinates. Then the
%% length of that curve in Euclidean space is
%% \[
%% \int_{b}^{e}\sqrt{\frac{dX}{dt}\bullet_{K}\frac{dX}{dt}}\,dt.
%% \]


\subsection{Computing angles}

\begin{problem}
  Show that when $K>0$ if two vectors are given to us in $(x,y,z)$-coordinates as
  \[
  \mathbf{v}_{1}=\begin{pmatrix}a_{1} \\ b_{1} \\ c_{1}\end{pmatrix}
  \qquad\text{and}\qquad
  \mathbf{v}_{2} =\begin{pmatrix}a_{2} \\ b_{2} \\ c_{2}\end{pmatrix}
  \]
  then the angle between their image in Euclidean space is given by
  \[
  \theta = \arccos\left(\frac{\mathbf{v}_1\bullet_K \mathbf{v}_2}{|\mathbf{v}_1|_K \cdot|\mathbf{v}_2|_K }\right).
  \]

%angleBetweenVectorsKDotProduct Geogebra%

\begin{freeResponse} Using the previous problems,
\[
\arccos\left(\frac{\mathbf{v}_1\bullet_K \mathbf{v}_2}{\left| \mathbf{v}_1\right|_K \cdot\left|\mathbf{v}_2\right|_K }\right) = \arccos\left(\frac{\hat{\mathbf v}_{1}\bullet \hat{\mathbf v}_{2}}{\left| \hat{\mathbf v}_{1}\right| \cdot\left|\hat{\mathbf v}_{2}\right| }\right) = \theta.
\]
\end{freeResponse}
\end{problem}




\subsection{Computing area}

\begin{problem}
%% Show that, if we have any two vectors in Euclidean three-space that
  %% are tangent to the $R$-sphere at some point on it, but the
  
  Show that if two vectors are given to us in $(x,y,z)$-coordinates as
  \[
  \mathbf{v}_{1}=\begin{pmatrix}a_{1} \\ b_{1} \\ c_{1}\end{pmatrix}
  \qquad\text{and}\qquad
  \mathbf{v}_{2} =\begin{pmatrix}a_{2} \\ b_{2} \\ c_{2}\end{pmatrix}
  \]
  then the area of the parallelogram spanned by the image of those two vectors in
  Euclidean space is%
  \[
  \sqrt{\det
    \begin{bmatrix}
      \mathbf{v}_{1}\bullet_{K}\mathbf{v}_{1} & \mathbf{v}_{2}\bullet_{K}\mathbf{v}_{1}\\
      \mathbf{v}_{1}\bullet_{K}\mathbf{v}_{2} & \mathbf{v}_{2}\bullet_{K}\mathbf{v}_{2}%
    \end{bmatrix}}
  =\sqrt{\det\left( 
      \begin{bmatrix}
        \mathbf{v}_{1} \\
        \mathbf{v}_{2}
      \end{bmatrix}
      \begin{bmatrix}
        1 & 0 & 0\\
        0 & 1 & 0\\
        0 & 0 & K^{-1}%
      \end{bmatrix}
      \begin{bmatrix}
        \mathbf{v}_{1}^\transpose & \mathbf{v}_{2}^\transpose%
      \end{bmatrix}
    \right) }.
\]

%kCrossProductParallelogram Geogebra%

\begin{freeResponse} In the previous problems it was shown that the K-dot product agrees with the Euclidean dot product so, 
\[
\sqrt{\det
\begin{bmatrix}
\mathbf{v}_{1}\bullet_{K}\mathbf{v}_{1} & \mathbf{v}_{2}\bullet_{K}\mathbf{v}_{1}\\
\mathbf{v}_{1}\bullet_{K}\mathbf{v}_{2} & \mathbf{v}_{2}\bullet_{K}\mathbf{v}_{2}%
\end{bmatrix}}
=
\sqrt{\det
\begin{bmatrix}
\hat{\mathbf v}_{1}\bullet \hat{\mathbf v}_{1} & \hat{\mathbf v}_{2}\bullet \hat{\mathbf v}_{1}\\
\hat{\mathbf v}_{1}\bullet \hat{\mathbf v}_{2} & \hat{\mathbf v}_{2}\bullet \hat{\mathbf v}_{2}%
\end{bmatrix}}.
\]
Which was previously shown to be the area of the parallelogram spanned by the image of $\mathbf{v}_1$ and $\mathbf{v}_2$ in Euclidean space. To show the equality write,

\begin{align*}
\sqrt{\det
\begin{bmatrix}
\mathbf{v}_{1}\bullet_{K}\mathbf{v}_{1} & \mathbf{v}_{2}\bullet_{K}\mathbf{v}_{1}\\
\mathbf{v}_{1}\bullet_{K}\mathbf{v}_{2} & \mathbf{v}_{2}\bullet_{K}\mathbf{v}_{2}%
\end{bmatrix}}
&=\sqrt{\det
\begin{bmatrix}
a^2_1 + b^2_1 + c^2_1K^{-1} & a_1a_2 + b_1b_2 + c_1c_2K^{-1}\\
a_1a_2 + b_1b_2 + c_1c_2K^{-1} & a^2_2 + b^2_2 + c^2_2K^{-1}%
\end{bmatrix}} \\
&= \sqrt{\det
\begin{bmatrix}
a_{1} & b_{1} & c_{1}\\
a_{2} & b_{2} & c_{2}
\end{bmatrix}
\begin{bmatrix}
1 & 0 & 0\\
0 & 1 & 0\\
0 & 0 & K^{-1}
\end{bmatrix}
\begin{bmatrix}
a_{1} & a_{1}  \\
b_{1} & b_{2}  \\
c_{1} &  c_{2}
\end{bmatrix}} \\
&= \sqrt{\det\left( 
\begin{bmatrix}
\mathbf{v}_{1} \\
\mathbf{v}_{2}
\end{bmatrix}
\begin{bmatrix}
1 & 0 & 0\\
0 & 1 & 0\\
0 & 0 & K^{-1}%
\end{bmatrix}
\begin{bmatrix}
\mathbf{v}_{1}^\transpose & \mathbf{v}_{2}^\transpose%
\end{bmatrix}
\right) }
\end{align*}
\end{freeResponse}

\end{problem}

\textit{Moral of the story:} The dot-product rules! That is, if you
know the dot-product you know everything there is to know about a
geometry, lengths, areas, angles, everything. And the set
\[
1=K\left(x^{2}+y^{2}\right)+z^{2} 
\]
continues to make sense even when $K$ is
negative. And as we will see later on, the definition of the $K$-dot
product also makes sense for tangent vectors to that set when $K$ is
negative. The geometry we get when the constant $K$ is chosen to be
negative is called a hyperbolic geometry. The geometry we get, when
the constant $K$ is just chosen to be non-zero is called a
non-Euclidean geometry.  In fact all the non-Euclidean $2$-dimensional
geometries are either spherical or hyperbolic.

%% \textit{Coming attractions:} In hyperbolic geometry, when $K^{-1}$ in
%% \[
%% \mathbf{v}_{1}\bullet_{K}\mathbf{v}_{2}  =\mathbf{v}_{1} 
%% \begin{bmatrix}
%% 1 & 0 & 0\\
%% 0 & 1 & 0\\
%% 0 & 0 & K^{-1}%
%% \end{bmatrix}
%% \mathbf{v}_{2}^\transpose
%% \]
%% becomes negative, so that the third coordinate of velocity, that is,
%% the $c$-direction, actually \textit{contracts} lengths. It was the
%% understanding of this mysterious fact that allowed Einstein to
%% discover (special) relativity.


\begin{problem}
Summarize the results from this section. In particular, summarize the results in your own words and indicate which
results follow from the others.
\begin{freeResponse}
In this section we started working in $K$-warped space, denoted by $(x, y, z)$-coordinates. We used the following rule to transition between Euclidean and  $(x, y, z)$-coordinates:
\begin{align*}
\hat{x}  &  =x\\
\hat{y}  &  =y\\
\hat{z}  &  =Rz.
\end{align*}
In $K$-warped space the following surface,
\[
K\left(x^{2}+y^{2}\right)+z^{2}=1,
\]
 generates the hyperbolic plane when $K<0$, the Euclidean plane when $K=0$, and the sphere of radius $R$ when $K> 0$. Using a computation involving the Euclidean dot product we defined the $K$-dot-product in $(x, y, z)$-coordinates such that,
\[
\mathbf{v}_{1}\bullet_{K}\mathbf{v}_{2}  = \mathbf{v}_{1} 
\begin{bmatrix}
1 & 0 & 0\\
0 & 1 & 0\\
0 & 0 & K^{-1}%
\end{bmatrix}
\mathbf{v}_{2}^\transpose.
\]
Lastly we calculated formulas for length, area, and the angle between vectors in $(x,y,z)$-coordinates in terms of the $K$-dot-product.
\end{freeResponse}
\end{problem}


\end{document}
