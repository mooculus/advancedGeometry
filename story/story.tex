\documentclass{amsart}
\usepackage{amsmath,amsfonts,amssymb}


\newcommand{\RR}{\mathbb{R}}
\newcommand{\transpose}{\intercal}

\title{The story of 4507}

\begin{document}
\begin{abstract}
  Each course should tell a story; here is the story that 4507 tells.
\end{abstract}
\maketitle

\section{Introduction}

In Math 4507 we guide the students through writing their own geometry
textbook. This textbook takes a quasi-historical approach that not
only presents topics in essentially historical order, but also changes
emphasis of concept es from those of the past, to those of the present
as the course proceeds.

\subsection{A note on rigor}




\section{Historical ordering of the course}

\subsection{Neutral geometry}

The course begins with neutral geometry. While neutral geometry was
first conceptualize by J\'anos Bolyai in 1832, and hence its
conceptualization is quite recent, Euclid seems to have had some
notion of neutral geometries, since he only used his first four axioms
to prove his first 29 theorems!

\subsubsection{Primary result(s)}

In neutral geometry the sum of the interior angles of a triangle sums
to less than $180^\circ$.

\subsubsection{Primary content}

High-school algebra and geometry.


\subsubsection{Current issues}

Each semester I have asked the question, ``Why does this proof fail in
spherical geometry.'' A satisfactory answer has never been given.


\subsection{Euclidean geometry}

Next we add Euclid's fifth axiom.

\subsubsection{Primary result(s)}

Now we show that the sum of the interior angles of a triangle in
euclidean geometry is $180^\circ$.

We give a proof of the Pythagorean Theorem.

\subsubsection{Primary content}

High-school algebra and geometry. 

\subsubsection{Current issues}

The question addressing the bijection between the plane and $\RR^2$
usually falls flat with the students.



\subsection{Similarity}

Now we talk about similar figures.

\subsubsection{Primary result(s)}

We prove basic results (high school) about similar triangles.

We prove basic results about concurrence (centroid, orthocenter of a triangle.)

\subsubsection{Primary content}

High-school algebra and geometry, vector calculus (scaling of
regions), middle grades geometry (another explanation of scaling of
regions).

\subsubsection{Current issues}

Proving dilations are bijective, reduces to proving they are
one-to-one and onto. Two approaches could be taken.
\begin{enumerate}
\item Prove one-to-one and onto separately. The issue here is that the students see these as being similar proofs. This causes some confusion.
\item Take the approach that if $f:A\to A$ and $f^{-1}:A\to A$ then $f$ must be bijective.
\end{enumerate}

The proof that dilations preserve angles is given via vectors.

The proof that if two triangles have perpendicular corresponding sides, then they are similar, is ``tricky'' and not general. 


\subsection{Angles in circles}

We look at (tri)angles inscribed in circles.

\subsubsection{Primary result(s)}

We prove basic results (high school) about central and inscribed angles.

We show that a triangle can be inscribed in a circle.

We prove basic results about concurrence (circumcenter).

We prove the extended law of sines.



\subsubsection{Primary content}

High-school algebra and geometry.

\subsubsection{Current issues}

None.




\subsection{Projective geometry}

We look at (tri)angles the geometric cross ratio.

\subsubsection{Primary result(s)}


We prove basic results about the cross ratio.

We give a second proof of the Pythagorean theorem.

We prove the sine addition and subtraction formula. 


\subsubsection{Primary content}

High-school algebra and geometry.

\subsubsection{Current issues}

None.




\subsection{Quantifying circles and spheres.}

We look at quantities related to circles and spheres.


\subsubsection{Primary result(s)}

We prove that the circumference of the unit circle is $2\pi$.

We give two proofs (middle grades/high school) that the area of the
unit circle is $\pi$

We prove that the surface area of the $R$-sphere is
\[
4\pi R^2.
\]

We prove that the volume of the $R$-sphere is
\[
\frac{4}{3}\pi R^3.
\]


\subsubsection{Primary content}

High-school algebra and geometry.

\subsubsection{Current issues}

None.




\subsection{Spherical triangles}

We look at quantities related to circles and spheres.


\subsubsection{Primary result(s)}

We find formulas for the areas of triangles. 


\subsubsection{Primary content}

A very nice proof.

\subsubsection{Current issues}

We should add the spherical Pythagorean theorem, and spherical law of sines.





\subsection{Euclidean space}

We study euclidean space.


\subsubsection{Primary result(s)}

The dot product computes length/angle/area.

\subsubsection{Primary content}

Vector calculus.

\subsubsection{Current issues}

The formulas for area seem to continue on even after we are done.

Matrices are multiplied on the right.



\subsection{$K$-warped space}

We study $K$-warped space space, $(x,y,z)$-coordinates.


\subsubsection{Primary result(s)}

Formulas for computing length/angle/area.

\subsubsection{Primary content}

Vector calculus.

Linear algebra.

The idea that to produce a new ``agreeable'' dot product is to
\begin{enumerate}
\item Take a curve, $F$, whose tangent vectors are known.
\item Write $F(x(t),y(t),z(t))$ where $x$, $y$, and $z$ determine a curve in the new space.
\item Differentiate via the chain rule.
\item Identify the difference between the tangents.
\item Write the difference as a matrix $M$.
\item Define the new dot product as:
  \[
V\cdot M \cdot V^\transpose.
\]
\end{enumerate}

\subsubsection{Current issues}

The above procedure is used implicitly throughout the notes. It needs
to be explicit.




\subsection{Congruences via rigid motions}

Orthogonal matrices are shown to be rigid motions.

\subsubsection{Primary result(s)}

A matrix is a rigid motion if and only if it is an orthogonal matrix. 

\subsubsection{Primary content}

Linear algebra.

Group theory.


\subsubsection{Current issues}

This now pick up a thread dropped in the first worksheet. However,
might it not be better to state the problem here in terms of rigid motions?

There is an ``apparent'' paradox here, a translation is a rigid
motion; however, a translation is not an orthogonal matrix.

Associativity of the group is not proved.




\subsection{$K$-rigid motions}

$K$-warped rigid motions are studied.

\subsubsection{Primary result(s)}

We derive a necessary and sufficient condition for a matrix to be a
$K$-rigid motion.

\subsubsection{Primary content}

Commutative diagrams.

Linear algebra.

Group theory.

The idea of generalizing the geometry is revealed.


\subsubsection{Current issues}

Associativity of the group is not proved.




\subsection{Spherical lines}

Lines in spherical geometry are studied.

\subsubsection{Primary result(s)}

We prove that great circles are geodesics in spherical geometry.


\subsubsection{Primary content}

Vector calculus.


\subsubsection{Current issues}

None really, in this semester this is when we'll prove the pythagorean
theorem and law of sines.












\begin{itemize}
\item Student centered learning.
\item Quasi-historical list of topics.
\item Quasi-historical emphasis of concepts.
\item Reiteration of topics from calculus.
\item Reiteration of topics from vector calculus.
\item Reiteration of topics from linear algebra.
\item Introduction to topics from abstract algebra.
\item Introduction to topics from applied mathematics.
\item S
\end{itemize}

\end{document}
