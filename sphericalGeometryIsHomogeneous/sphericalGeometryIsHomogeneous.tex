\documentclass{ximera}

\usepackage{microtype}
\usepackage{tikz}
\usepackage{tkz-euclide}
\usetkzobj{all}
\tikzstyle geometryDiagrams=[ultra thick,color=blue!50!black]

\graphicspath{
{./}
{areasOnSpheresInEuclidean3Space/}
{centralProjection/}
{stereographicProjection/}
{centralProjectionInHG/}
{stereographicProjectionInHG/}
}


\newcommand{\transpose}{\intercal}
\renewcommand{\epsilon}{\varepsilon}
\renewcommand{\l}{\ell}
\renewcommand{\d}{\,d}

\newcommand{\R}{\mathbb R}


\renewcommand{\bar}{\overline}


\title{Spherical geometry is homogeneous}
\begin{document}
\begin{abstract}

\end{abstract}
\maketitle



\subsection*{Moving a point to the North Pole by a rigid motion}

As the heading suggests, we are next going to move any point on the
$R$-sphere to the North Pole by a rigid motion. However, we are going
to describe the entire process in $\left( x,y,z\right) $-coordinates,
that is, in $K$-geometry. This will allow us to use all the
computations we did with $K$-geometry since \textbf{SG} is a
$K$-geometry.  Recll that, in $\left( x,y,z\right) $-coordinates, the
equation for the $R$-sphere becomes%
\begin{equation}
K\left(  x^{2}+y^{2}\right)  +z^{2}=1 \label{72}%
\end{equation}
with%
\[
K=\frac{1}{R^{2}}%
\]
and the Euclidean dot product is given by the $K$-dot product. Again, if you
get nervous using these weird coordinates to compute things that are clearer
in $\left(  \hat{x},\hat{y},\hat{z}\right)  $-coordinates, just go through the
constructions in the case $K=1$ first. In that special case
\[
\left(  x,y,z\right)  =\left(  \hat{x},\hat{y},\hat{z}\right)
\]
and your calculations (as well as all those in $K$-geometry above reduce to
the usual ones on the unit sphere in ordinary Euclidean $3$-space.

So, first of all, in $\left(  x,y,z\right)  $-coordinates the North Pole is
the point%
\[
N=\left(  0,0,1\right)  .
\]
Suppose we start with a point%
\[
X_{0}=\left(  x_{0},y_{0},z_{0}\right)
\]
in the geometry, that is, satisfying the equation $\left(  \ref{72}\right)  $.

\begin{exercise}
 Write an explicit $K$-rigid motion%
\[
M_{1}=\left(
\begin{array}
[c]{ccc}%
\cos\theta & \sin \theta & 0\\
-\sin \theta & \cos\theta & 0\\
0 & 0 & 1
\end{array}
\right)
\]
that takes the point $X_{0}$ to a point $X_{1}=\left(  x_{1},0,z_{0}\right)  $.

Hint: Start from the identity%
\[
\frac{-y_{0}}{\sqrt{x_{0}^{2}+y_{0}^{2}}}\cdot x_{0}%
+\frac{x_{0}}{\sqrt{x_{0}^{2}+y_{0}^{2}}}\cdot y_{0}=0
\]%
\[
\sin \theta\cdot x_{0}+\cos\theta
\cdot y_{0}=0
\]
and then show that there is a $\theta$ so that%
\begin{align*}
\cos\theta &  =\frac{x_{0}}{\sqrt{x_{0}^{2}+y_{0}^{2}}}\\
\sin \theta &  =\frac{-y_{0}}{\sqrt{x_{0}^{2}+y_{0}^{2}}}.
\end{align*}

\end{exercise}

\begin{exercise}
 Write an explicit $K$-rigid motion%
\[
M_{2}=\left(
\begin{array}
[c]{ccc}%
\cos\varphi & 0 & R^{-1}\cdot \sin %
\varphi\\
0 & 1 & 0\\
-R\cdot \sin \varphi & 0 & \cos\varphi
\end{array}
\right)
\]
that takes the point $X_{1}=\left(  x_{1},0,z_{0}\right)  $ to $N=\left(
0,0,1\right)  $.
\end{exercise}

Using these last two Exercises we conclude that the transformation%
\[
\left(  \underline{x},\underline{y},\underline{z}\right)  =\left(
x,y,z\right)  \cdot\left(  M_{1}\cdot M_{2}\right)
\]
is a $K$-rigid motion (why?) and that%
\[
N=\left(  x_{0},y_{0},z_{0}\right)  \cdot\left(  M_{1}\cdot M_{2}\right)
\]
(why?).



\subsection*{Moving a (point, direction) to any other (point, direction) by a
rigid motion}

Let
\[
V_{2}=\left(  a_{2},b_{2},0\right)
\]
be a tangent vector to $K$-geometry at the North Pole $N$.

\begin{exercise}
 Write an explicit $K$-rigid motion%
\[
M_{3}=\left(
\begin{array}
[c]{ccc}%
\cos\theta' & \sin \theta' & 0\\
-\sin \theta' & \cos\theta' & 0\\
0 & 0 & 1
\end{array}
\right)
\]
that takes $V_{2}$ to the vector%
\[
\left(  \sqrt{a_{2}^{2}+b_{2}^{2}},0,0\right)  =\left(  \sqrt{V_{2}\bullet
_{K}V_{2}},0,0\right)  .
\]
Why does the transformation given by $M_{3}$ leave the North Pole $N$ fixed?
\end{exercise}

Now suppose we have any point%
\[
X_{0}=\left(  x_{0},y_{0},z_{0}\right)
\]
in $K$-geometry and any $K$-tangent vector%
\[
V_{0}=\left(  a_{0},b_{0},c_{0}\right)
\]
at that point.

\begin{exercise}
 Explain why the $K$-rigid motion%
\[
\left(  \underline{x},\underline{y},\underline{z}\right)  =\left(
x,y,z\right)  \cdot\left(  M_{1}\cdot M_{2}\cdot M_{3}\right)
\]
constructed over the last couple of sections takes the point $X_{0}$ to $N$
and the tangent vector $V_{0}$ to $\left(  \sqrt{V_{0}\bullet_{K}V_{0}%
},0,0\right)  $
\end{exercise}

Now suppose that $\left(  X_{0},V_{0}\right)  $ gives a point $X_{0}$ in
$K$-geometry and a tangent direction $V_{0}$ to $K$-geometry at $X_{0}$.
Suppose that $\left(  X_{0}',V_{0}'\right)  $ gives another
point in $K$-geometry and a tangent direction to $K$-geometry at
$X_{0}'$. Finally suppose that%
\[
V_{0}\bullet_{K}V_{0}=V_{0}'\bullet_{K}V_{0}'.
\]
As above, find a $K$-rigid motion given by%
\[
M=\left(  M_{1}\cdot M_{2}\cdot M_{3}\right)
\]
taking $X_{0}$ to the North Pole and $V_{0}$ to $\left(  \sqrt{V_{0}%
\bullet_{K}V_{0}},0,0\right)  $. Similarly find a $K$-rigid motion given by%
\[
M'=\left(  M_{1}'\cdot M_{2}'\cdot M_{3}^{\prime
}\right)
\]
taking $X_{0}'$ to the North Pole and $V_{0}'$ to $\left(
\sqrt{V_{0}'\bullet_{K}V_{0}'},0,0\right)  .$

\begin{exercise}
\label{64} Explain why the $K$-rigid motion given by%
\[
M\cdot\left(  M'\right)  ^{-1}%
\]
takes $\left(  X_{0},V_{0}\right)  $ to $\left(  X_{0}',\left(
\frac{\left\vert V_{0}\right\vert _{K}}{\left\vert V_{0}'\right\vert
_{K}}\right)  \cdot V_{0}'\right)  $.
\end{exercise}

By completing this Exercise we have shown that $K$-geometry looks the same at
each point and in each direction at that point. That is, we have shown that
each $K$-geometry is homogeneous.


\end{document}
