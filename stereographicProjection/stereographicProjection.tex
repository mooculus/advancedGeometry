\documentclass{ximera}

\preambleinput{../preamble.tex}

\title{Stereographic projection}
\begin{document}
\begin{abstract}
Here we start to develop another model for our geometry.
\end{abstract}
\maketitle


\subsection*{Stereographic projection coordinates}

On the other hand we can project the set
\[
1=K\left(  x^{2}+y^{2}\right)  +z^{2}%
\]
onto the set%
\[
z=1
\]
using the `South Pole'%
\[
S=\left(  0,0,-1\right)
\]
as the center of projection:%

\begin{tabular}
[c]{cc}%
%TCIMACRO{\FRAME{itbpF}{2.789in}{1.9389in}{0in}{}{}{Figure}%
%{\special{ language "Scientific Word";  type "GRAPHIC";
%maintain-aspect-ratio TRUE;  display "USEDEF";  valid_file "T";
%width 2.789in;  height 1.9389in;  depth 0in;  original-width 4.9121in;
%original-height 3.4065in;  cropleft "0";  croptop "1";  cropright "1";
%cropbottom "0";  tempfilename 'MXAJBZ0M.png';tempfile-properties "XPR";}}}%
%BeginExpansion
{\includegraphics[
natheight=3.406500in,
natwidth=4.912100in,
height=1.9389in,
width=2.789in
]%
{MXAJBZ0M.jpg}%
}%
%EndExpansion
&
%TCIMACRO{\FRAME{itbpF}{2.226in}{1.9415in}{0in}{}{}{Figure}%
%{\special{ language "Scientific Word";  type "GRAPHIC";
%maintain-aspect-ratio TRUE;  display "USEDEF";  valid_file "T";
%width 2.226in;  height 1.9415in;  depth 0in;  original-width 4.8896in;
%original-height 4.2635in;  cropleft "0";  croptop "1";  cropright "1";
%cropbottom "0";  tempfilename 'MXAJBZ0N.png';tempfile-properties "XPR";}}}%
%BeginExpansion
{\includegraphics[
natheight=4.263500in,
natwidth=4.889600in,
height=1.9415in,
width=2.226in
]%
{MXAJBZ0N.jpg}%
}%
%EndExpansion
\end{tabular}


That is,%
\[
\rho\text{\textperiodcentered}\left(  x_{s},y_{s},1-\left(  -1\right)
\right)  =\left(  x,y,z-\left(  -1\right)  \right)  .
\]
So%
\begin{align*}
\rho &  =\frac{z+1}{2}\\
z  &  =2\rho-1
\end{align*}
and, from the equation $\left(  \ref{30}\right)  $%
\begin{gather*}
K\left(  \left(  \rho x_{s}\right)  ^{2}+\left(  \rho y_{s}\right)
^{2}\right)  +\left(  2\rho-1\right)  ^{2}=1\\
K\left(  \left(  \rho x_{s}\right)  ^{2}+\left(  \rho y_{s}\right)
^{2}\right)  +4\rho^{2}-4\rho=0\\
\rho K\left(  x_{s}^{2}+y_{s}^{2}\right)  +4\rho=4\\
\rho=\frac{1}{\frac{K}{4}\left(  x_{s}^{2}+y_{s}^{2}\right)  +1}.
\end{gather*}
Notice that, when $K<0$ this last formula only makes sense when%
\begin{gather*}
\frac{K}{4}\left(  x_{s}^{2}+y_{s}^{2}\right)  >-1\\
x_{s}^{2}+y_{s}^{2}<\frac{-4}{K}.
\end{gather*}


\begin{exercise}\label{35}\hfil
\begin{enumerate}
\item For the projection of the set $\left(  \ref{30}\right)  $ onto
the $z=1$ plane with center of projection $S$, write $\left(  x_{s}%
,y_{s}\right)  $ as a function of $\left(  x,y,z\right)  $.
\item For the projection of the set $\left(  \ref{30}\right)  $ onto the $z=1$
plane with center of projection $S$, write $\left(  x,y,z\right)  $ as a
function of $\left(  x_{s},y_{s}\right)  $.
\end{enumerate}
\end{exercise}

\subsection*{Length and angle in stereographic projection coordinates}

\begin{exercise}
\label{37}Suppose we have a path%
\[
X\left(  x_{s}\left(  t\right)  ,y_{s}\left(  t\right)  \right)  =\left(
x\left(  x_{s}\left(  t\right)  ,y_{s}\left(  t\right)  \right)  ,y\left(
x_{s}\left(  t\right)  ,y_{s}\left(  t\right)  \right)  ,z\left(  x_{s}\left(
t\right)  ,y_{s}\left(  t\right)  \right)  \right)
\]
lying on the set $\left(  \ref{30}\right)  $ given in terms of its projection
$\left(  x_{s}\left(  t\right)  ,y_{s}\left(  t\right)  \right)  $ in the
plane $z=1$. Use the formula you derived in Exercise \ref{35}b) and the Chain
Rule from calculus of several variables to find the $2\times3$ matrix%
\[
D_{s}=\left(
\begin{array}
[c]{c}%
\left(  \frac{\partial X}{\partial x_{s}}\right) \\
\left(  \frac{\partial X}{\partial y_{s}}\right)
\end{array}
\right)
\]
such that%
\[
\left(  \frac{dx}{dt},\frac{dy}{dt},\frac{dz}{dt}\right)  =\left(
\frac{dx_{s}\left(  t\right)  }{dt},\frac{dy_{s}\left(  t\right)  }%
{dt}\right)  \cdot D_{s}.
\]
Hint: Use logarithmic differentiation:%
\begin{gather*}
dx=d\left(  \rho x_{s}\right)  =x_{s}d\rho+\rho dx_{s}\\
\rho^{-1}dx=x_{s}d\mathrm{ln}\left(  \rho\right)  +dx_{s}%
\end{gather*}
and similarly for $y$. Also%
\begin{align*}
d\mathrm{ln}\left(  \rho\right)   &  =-d\mathrm{ln}\left(  \frac{K}{4}\left(
x_{s}^{2}+y_{s}^{2}\right)  +1\right) \\
&  =-\frac{1}{\frac{K}{4}\left(  x_{s}^{2}+y_{s}^{2}\right)  +1}d\left(
\frac{K}{4}\left(  x_{s}^{2}+y_{s}^{2}\right)  +1\right) \\
&  =-\rho\frac{K}{4}\left(  2x_{s}dx_{s}+2y_{s}dy_{s}\right)  .
\end{align*}

\end{exercise}

This last Exercise allows us to do something very nice. Namely now, not only
can we use the coordinates $\left(  x_{s},y_{s}\right)  $ for our geometry but
we can also compute the $K$-dot product in terms of these coordinates:%
\[
\left(  \frac{dx}{dt},\frac{dy}{dt},\frac{dz}{dt}\right)  =\left(
\frac{dx_{s}}{dt},\frac{dy_{s}}{dt}\right)  \cdot D_{s}%
\]
so that%
\begin{align*}
\left(  \frac{dx}{dt},\frac{dy}{dt},\frac{dz}{dt}\right)  \bullet_{K}\left(
\frac{dx}{dt},\frac{dy}{dt},\frac{dz}{dt}\right)   &  =\left(
\begin{array}
[c]{ccc}%
\frac{dx}{dt} & \frac{dy}{dt} & \frac{dz}{dt}%
\end{array}
\right)  \left(
\begin{array}
[c]{ccc}%
1 & 0 & 0\\
0 & 1 & 0\\
0 & 0 & K^{-1}%
\end{array}
\right)  \left(
\begin{array}
[c]{c}%
\frac{dx}{dt}\\
\frac{dy}{dt}\\
\frac{dz}{dt}%
\end{array}
\right) \\
&  =\left(
\begin{array}
[c]{cc}%
\frac{dx_{s}}{dt} & \frac{dy_{s}}{dt}%
\end{array}
\right)  \cdot D_{s}\cdot\left(
\begin{array}
[c]{ccc}%
1 & 0 & 0\\
0 & 1 & 0\\
0 & 0 & K^{-1}%
\end{array}
\right)  \cdot D_{s}^{t}\cdot\left(
\begin{array}
[c]{c}%
\frac{dx_{s}}{dt}\\
\frac{dy_{s}}{dt}%
\end{array}
\right)  ,
\end{align*}


\begin{exercise}
\label{36}Use matrix multiplication to compute the $2\times2$ matrix%
\[
P_{s}=D_{s}\cdot\left(
\begin{array}
[c]{ccc}%
1 & 0 & 0\\
0 & 1 & 0\\
0 & 0 & K^{-1}%
\end{array}
\right)  \cdot D_{s}^{t},
\]
that is, to compute the $K$-dot product in $\left(  x_{s},y_{s}\right)
$-coordinates. (You may be surprised at the answer! It is quite simple and
only involves the quantity $\rho$.)
\end{exercise}

So, if, if $K>0$ and you have a path on the sphere of radius $R=K^{-1/2}$ in
Euclidean $3$-space given in $\left(  x_{s},y_{s}\right)  $-coordinates as
$\left(  x_{s}\left(  t\right)  ,y_{s}\left(  t\right)  \right)  $ for
$t\in\left[  b,e\right]  $, you can trace back everything we have done with
coordinate changes to see that the length of the path on the sphere of radius
$R=K^{-1/2}$ in Euclidean $3$-space is given by%
\[%
%TCIMACRO{\dint \nolimits_{b}^{e}}%
%BeginExpansion
{\displaystyle\int\nolimits_{b}^{e}}
%EndExpansion
l\left(  t\right)  dt
\]
where%
\begin{align*}
l\left(  t\right)  ^{2}  &  =\left(  \frac{dx_{s}}{dt},\frac{dy_{s}}%
{dt}\right)  \bullet_{s}\left(  \frac{dx_{s}}{dt},\frac{dy_{s}}{dt}\right) \\
&  =\left(
\begin{array}
[c]{cc}%
\frac{dx_{s}}{dt} & \frac{dy_{s}}{dt}%
\end{array}
\right)  \cdot P_{s}\cdot\left(
\begin{array}
[c]{c}%
\frac{dx_{s}}{dt}\\
\frac{dy_{s}}{dt}%
\end{array}
\right)
\end{align*}
and that the measure $\theta$ of an angle between vectors $\hat{V}_{1}$ and
$\hat{V}_{2}$ on the $R$-sphere is computed by%
\[
\mathrm{arccos}\left(  \frac{V_{1}^{s}\cdot P_{s}\cdot\left(  V_{2}%
^{s}\right)  ^{t}}{\left\vert V_{1}^{s}\right\vert _{s}\left\vert V_{2}%
^{s}\right\vert _{s}}\right)  .
\]


Notice that the matrix $P_{s}$ still makes sense when $K=0$ and when $K$
becomes negative.

\begin{exercise}
Write the formula for the $K$-dot product $\left(  x_{s},y_{s}\right)
$-coordinates when $K=0$. Does it look familiar?
\end{exercise}

So we do have%
\[%
{\renewcommand{\arraystretch}{2}
\begin{tabular}
[c]{c|c|c}%
Spherical ($K>0$) & Euclidean ($K=0$) & Hyperbolic ($K<0$)\\\hline
$\hat{x}^{2}+\hat{y}^{2}+\hat{z}^{2}=R^{2}$ &  & \\\hline
$\hat{V}\bullet\hat{V}$ &  & \\\hline
$1=K\left(  x^{2}+y^{2}\right)  +z^{2}$ & $1=K\left(  x^{2}+y^{2}\right)
+z^{2}$ & $1=K\left(  x^{2}+y^{2}\right)  +z^{2}$\\\hline
$V_{1}\bullet_{K}V_{2}$ &  & $V_{1}\bullet_{K}V_{2}$\\\hline
$V_{1}^{c}\bullet_{c}V_{2}^{c}$ & $V_{1}^{c}\bullet_{c}V_{2}^{c}$ & $V_{1}%
^{c}\bullet_{c}V_{2}^{c}$\\\hline
$V_{1}^{s}\bullet_{s}V_{2}^{s}$ & $V_{1}^{s}\bullet_{s}V_{2}^{s}$ & $V_{1}%
^{s}\bullet_{s}V_{2}^{s}$%
\end{tabular}}
\]
where%
\[
V_{1}^{s}\bullet_{s}V_{2}^{s}=\left(  V_{1}^{s}\right)  \cdot P_{s}%
\cdot\left(  V_{2}^{s}\right)  ^{t}.
\]
Of course if $K>0$, we again have Euclidean angles $\theta$ between vectors
$\hat{V}_{1}$ and $\hat{V}_{2}$ tangent to the $R$-sphere at some point
computed by%
\begin{align*}
\hat{V}_{1}\bullet\hat{V}_{2}  &  =\left\vert \hat{V}_{1}\right\vert
\text{\textperiodcentered}\left\vert \hat{V}_{2}\right\vert
\text{\textperiodcentered\textrm{cos}}\left(  \theta\right) \\
&  =V_{1}^{s}\bullet_{s}V_{2}^{s}.
\end{align*}


\subsection*{Area in stereographic projection coordinates}

Suppose you were given a region $G_{s}$ in the $\left(  x_{s},y_{s}\right)
$-coordinate plane. Also suppose that $K>0$. If you trace back everything we
have done with coordinate changes, you can see how $G_{s}$ gives you a region
$\hat{G}$ on the sphere of radius $R=K^{-1/2}$ in Euclidean $3$-space via the
formulas%
\begin{align*}
\left(  \hat{x},\hat{y},\hat{z}\right)   &  =\left(  x,y,Rz\right) \\
&  =\rho\text{\textperiodcentered}\left(  x_{s},y_{s},R\left(  2\rho-1\right)
\right) \\
&  =\left(  \frac{x_{s}}{\frac{K}{4}\left(  x_{s}^{2}+y_{s}^{2}\right)
+1},\frac{y_{s}}{\frac{K}{4}\left(  x_{s}^{2}+y_{s}^{2}\right)  +1}%
,\frac{R\left(  1-\frac{K}{4}\left(  x_{s}^{2}+y_{s}^{2}\right)  \right)
}{1+\frac{K}{4}\left(  x_{s}^{2}+y_{s}^{2}\right)  }\right)  .
\end{align*}
Now there is a formula in several variable calculus for computing the area of
the region $\hat{G}$ on the sphere of radius $R$ in Euclidean $3$-space in
terms of the parameters $\left(  x_{s},y_{s}\right)  $. [DS,49,231]. It is
\[%
%TCIMACRO{\dint \nolimits_{G_{c}}}%
%BeginExpansion
{\displaystyle\int\nolimits_{G_{c}}}
%EndExpansion
\hat{a}\left(  \frac{d\hat{X}}{dx_{s}},\frac{d\hat{X}}{dy_{s}}\right)
dx_{s}dy_{s}%
\]
where $\hat{a}\left(  \frac{d\hat{X}}{dx_{s}},\frac{d\hat{X}}{dy_{s}}\right)
$ is the (Euclidean) area of the parallelogram spanned by the two vectors
$\frac{d\hat{X}}{dx_{s}}$ and $\frac{d\hat{X}}{dy_{s}}$ in Euclidean
$3$-space. That is%
\[
\hat{a}\left(  \frac{d\hat{X}}{dx_{s}},\frac{d\hat{X}}{dy_{s}}\right)
=\left\vert \frac{d\hat{X}}{dx_{s}}\right\vert \text{\textperiodcentered
}\left\vert \frac{d\hat{X}}{dy_{s}}\right\vert \text{\textperiodcentered
}\mathrm{sin}\left(  \theta\right)
\]
where $\theta$ is the angle between the two vectors $\frac{d\hat{X}}{dx_{s}}$
and $\frac{d\hat{X}}{dy_{s}}$.

\begin{exercise}
As in Exercise \ref{9} show that%
\begin{align*}
\hat{a}\left(  \frac{d\hat{X}}{dx_{s}},\frac{d\hat{X}}{dy_{s}}\right)  ^{2}
&  =\left\vert
\begin{array}
[c]{cc}%
\frac{d\hat{X}}{dx_{s}}\bullet\frac{d\hat{X}}{dx_{s}} & \frac{d\hat{X}}%
{dy_{s}}\bullet\frac{d\hat{X}}{dx_{s}}\\
\frac{d\hat{X}}{dx_{s}}\bullet\frac{d\hat{X}}{dy_{s}} & \frac{d\hat{X}}%
{dy_{s}}\bullet\frac{d\hat{X}}{dy_{s}}%
\end{array}
\right\vert \\
&  =\left\vert
\begin{array}
[c]{cc}%
\frac{dX}{dx_{s}}\bullet_{K}\frac{dX}{dx_{s}} & \frac{dX}{dy_{s}}\bullet
_{K}\frac{dX}{dx_{s}}\\
\frac{dX}{dx_{s}}\bullet_{K}\frac{dX}{dy_{s}} & \frac{dX}{dy_{s}}\bullet
_{K}\frac{dX}{dy_{s}}%
\end{array}
\right\vert
\end{align*}

\end{exercise}

Now notice the matrix $D_{s}$ in Exercise \ref{37} is simply the $2\times3$
matrix whose rows are the vectors $\frac{dX}{dx_{s}}$ and $\frac{dX}{dy_{s}}$.

\begin{exercise}
Use Exercise \ref{36} to show that%
\[
\hat{a}\left(  \frac{d\hat{X}}{dx_{s}},\frac{d\hat{X}}{dy_{s}}\right)
^{2}=\rho^{4}=\frac{1}{\left(  \frac{K}{4}\left(  x_{s}^{2}+y_{s}^{2}\right)
+1\right)  ^{4}}.
\]

\end{exercise}

\end{document}
