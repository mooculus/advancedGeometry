\documentclass{ximera}

\title{Lines in spherical geometry}
\begin{document}
\begin{abstract}
Here we examine ``lines'' in spherical geometry.
\end{abstract}
\maketitle


\subsection*{Spherical coordinates, a shortest path from the North Pole}

We next will figure out what is the shortest path you can take between two
points on the Euclidean $R$-sphere. Again we will do our calculation using
only $\left(  x,y,z\right)  $-coordinates (since, as we have seen in $\left(
\ref{66}\right)  $ we won't have $\left(  \hat{x},\hat{y},\hat{z}\right)
$-coordinates when we get to Hyperbolic Geometry. For our purposes, it will be
convenient to use yet another set of coordinates for $K$-geometry, namely what
are commonly known as spherical coordinates:%
\begin{align}
x\left(  \sigma,\tau\right)   &  =R\cdot \sin%
\sigma\cdot \cos \tau\nonumber\\
y\left(  \sigma,\tau\right)   &  =R\cdot \sin%
\sigma\cdot \sin\tau\label{133}\\
z\left(  \sigma,\tau\right)   &  =\cos \sigma\nonumber
\end{align}


\begin{exercise}
Show that these spherical coordinates do actually parametrize
the $R$-sphere, that is, that%
\[
K\left(  x\left(  \sigma,\tau\right)  ^{2}+y\left(  \sigma,\tau\right)
^{2}\right)  +z\left(  \sigma,\tau\right)  ^{2}\equiv1
\]
for all $\left(  \sigma,\tau\right)  $.
\end{exercise}

Notice that you can write a path on the $R$-sphere by giving a path $\left(
\sigma\left(  t\right)  ,\tau\left(  t\right)  \right)  $ in the $\left(
\sigma,\tau\right)  $-plane. In fact, you can use $\sigma$ as the parameter
$t$ and just write
\begin{equation}
\left(  \sigma,\tau\left(  \sigma\right)  \right)  \label{62}%
\end{equation}
where $\tau$ is a function of $\sigma$. To write a path that starts at the
North Pole, just write%
\[
\left(  \sigma,\tau\left(  \sigma\right)  \right)  ,\;0\leq\sigma
\leq\varepsilon
\]
and demand that%
\[
\tau\left(  0\right)  =0.
\]
If you want the path to end on the plane $y=\hat{y}=0$, demand additionally
that%
\[
\tau\left(  \varepsilon\right)  =0.
\]
But if we are going to describe paths on the $R$-sphere by paths in the
$\left(  \sigma,\tau\right)  $-plane we are going to need to figure out the
$K$-dot product in $\left(  \sigma,\tau\right)  $-coordinates so that we can
compute the lengths of paths in these coordinates.

\begin{exercise}
\label{3333} \begin{enumerate}
\item Referring to $\left(  \ref{133}\right)  $compute
the $2\times3$ matrix
\[
D_{sph}=\left(
\begin{array}
[c]{ccc}%
\frac{dx}{d\sigma} & \frac{dy}{d\sigma} & \frac{dz}{d\sigma}\\
\frac{dx}{d\tau} & \frac{dy}{d\tau} & \frac{dz}{d\tau}%
\end{array}
\right)  .
\]


\item  Show that, if a path in $K$-geometry is given by a path $\left(
\sigma\left(  t\right)  ,\tau\left(  t\right)  \right)  $ in the $\left(
\sigma,\tau\right)  $-plane,%
\[
\left(  \frac{dx}{dt},\frac{dy}{dt},\frac{dz}{dt}\right)  =\left(
\frac{d\sigma}{dt},\frac{d\tau}{dt}\right)  \cdot D_{sph}.
\]


\item For two paths in $K$-geometry given by paths $\left(  \sigma_{1}\left(
t\right)  ,\tau_{1}\left(  t\right)  \right)  $ and $\left(  \sigma_{2}\left(
t\right)  ,\tau_{2}\left(  t\right)  \right)  $ in the $\left(  \sigma
,\tau\right)  $-plane, use a) and b) to show that
\begin{gather*}
\left(  \frac{d\hat{x}_{1}}{dt},\frac{d\hat{y}_{1}}{dt},\frac{d\hat{z}_{1}%
}{dt}\right)  \cdot\left(  \frac{dx_{2}}{dt},\frac{d\hat{y}_{2}}{dt}%
,\frac{d\hat{z}_{2}}{dt}\right)  ^{t}=\\
\left(  \frac{dx_{1}}{dt},\frac{dy_{1}}{dt},\frac{dz_{1}}{dt}\right)
\cdot\left(
\begin{array}
[c]{ccc}%
1 & 0 & 0\\
0 & 1 & 0\\
0 & 0 & K^{-1}%
\end{array}
\right)  \cdot\left(  \frac{dx_{2}}{dt},\frac{dy_{2}}{dt},\frac{dz_{2}}%
{dt}\right)  ^{t}=\\
\left(  \frac{d\sigma_{1}}{dt},\frac{d\tau_{1}}{dt}\right)  \cdot\left(
\begin{array}
[c]{cc}%
K^{-1} & 0\\
0 & K^{-1}\sin^{2}\sigma
\end{array}
\right)  \cdot\left(  \frac{d\sigma_{2}}{dt},\frac{d\tau_{2}}{dt}\right)  ^{t}%
\end{gather*}

\item Explain why the definition%
\[
\left(  \frac{d\sigma_{1}}{dt},\frac{d\tau_{1}}{dt}\right)  \bullet
_{sph}\left(  \frac{d\sigma_{2}}{dt},\frac{d\tau_{2}}{dt}\right)  =\left(
\frac{d\sigma_{1}}{dt},\frac{d\tau_{1}}{dt}\right)  \cdot\left(
\begin{array}
[c]{cc}%
K^{-1} & 0\\
0 & K^{-1}\sin^{2}\sigma
\end{array}
\right)  \cdot\left(  \frac{d\sigma_{2}}{dt},\frac{d\tau_{2}}{dt}\right)  ^{t}%
\]
allows us to compute the dot product of two tangent vectors to the $R$-sphere
in Euclidean space if we just know the values of the two corresponding vectors
in the $\left(  \sigma,\tau\right)  $-plane.
\end{enumerate}
\end{exercise}

\begin{exercise}
 Show that the length $L$ of any path on the $R$-sphere given by%
\[
\left(  \sigma,\tau\left(  \sigma\right)  \right)  ,\;0\leq\sigma
\leq\varepsilon
\]
with%
\[
\tau\left(  0\right)  =0.
\]
and%
\[
\tau\left(  \varepsilon\right)  =0
\]
is given by the formula%
\[
L=R%
%TCIMACRO{\dint \nolimits_{0}^{\varepsilon}}%
%BeginExpansion
{\displaystyle\int\nolimits_{0}^{\varepsilon}}
%EndExpansion
\sqrt{\left(  1,\frac{d\tau}{d\sigma}\right)  \cdot\left(
\begin{array}
[c]{cc}%
1 & 0\\
0 & \sin^{2}\sigma
\end{array}
\right)  \cdot\left(  1,\frac{d\tau}{d\sigma}\right)  ^{t}}d\sigma.
\]


Hint: Use Exercise \ref{3333} with $t=\sigma$.
\end{exercise}

This last formula for $L$ lets us figure out the shortest path from $N=\left(
R\cdot \sin 0\cdot %
\cos 0,R\cdot \sin %
0\cdot \sin 0,\cos 0\right)  $ to $\left(
R\cdot \sin \varepsilon,0,\cos %
\varepsilon\right)  =\left(  R\cdot \mathrm{sin\varepsilon
}\cdot \cos 0,R\text{\textperiodcentered
}\sin 0\cdot \sin 0,\cos %
\varepsilon\right)  $. Since%
\[
L=R%
%TCIMACRO{\dint \nolimits_{0}^{\varepsilon}}%
%BeginExpansion
{\displaystyle\int\nolimits_{0}^{\varepsilon}}
%EndExpansion
\sqrt{1+\sin ^{2}\sigma\cdot \left(  \frac{d\tau
}{d\sigma}\right)  ^{2}}d\sigma
\]
and $\sin ^{2}\sigma$ is is positive for almost all $\sigma\in\left[
0,\varepsilon\right]  $, $L$ is minimal only when $\frac{d\tau}{d\sigma}$ is
identically $0$. But this means that $\tau\left(  \sigma\right)  $ is a
constant function. Since $\tau\left(  0\right)  =0$, this means that
$\tau\left(  \sigma\right)  $ is identically $0$. So we have the shown the
following result.

\begin{theorem}
The shortest path on the $R$-sphere from the North Pole to a
point $\left(  x,y,z\right)  =\left(  R\cdot %
\sin \varepsilon,0,\cos \varepsilon\right)  $ is the path lying
in the plane $y=0$.
\end{theorem}



\subsection*{Shortest path between any two points}

We next prove the theorem that shows that shortest path on the surface of the
Earth from Rio de Janeiro to Los Angeles is the one cut on the surface of the
Earth by the plane that passes through the center of the Earth and through Rio
and through Los Angeles. That is usually the route an airplane would take when
flying between the two cities.

\begin{theorem}
Given any two points $X_{1}=\left(  x_{1},y_{1},z_{1}\right)  $
and $X_{2}=\left(  x_{2},y_{2},z_{_{2}}\right)  $ in $K$-geometry, the
shortest path between the two points is the path cut out by the two equations%
\[
K\left(  x^{2}+y^{2}\right)  +z^{2}=1
\]%
\begin{equation}
\left\vert \left(
\begin{array}
[c]{ccc}%
x & y & z\\
x_{1} & y_{1} & z_{1}\\
x_{2} & y_{2} & z_{2}%
\end{array}
\right)  \right\vert =0, \label{63}%
\end{equation}
that is, the plane containing $\left(  0,0,0\right)  $ and $X_{1}$ and $X_{2}$.
\end{theorem}

\begin{proof}
By Exercise \ref{64} there is a $K$-rigid motion $M$ that takes $X_{1}$ to the
North Pole $N$ and $X_{2}$ to $\left(  K^{-1/2}sin\varepsilon,0,cos\varepsilon
\right)  $ for some $\varepsilon$. That is%
\[
X_{2}\cdot M=\left(  R\cdot \sin \varepsilon
,0,\cos \varepsilon\right)
\]
for some $\varepsilon$ since all points in $K$-geometry with \underline{$y$%
}$=0$ can be written as $\left(  R\cdot \sin %
\varepsilon,0,\cos \varepsilon\right)  $ for some $\varepsilon$. But%
\begin{align*}
\left\vert \left(
\begin{array}
[c]{ccc}%
\underline{x} & \underline{y} & \underline{z}\\
0 & 0 & 1\\
K^{-1/2}sin\varepsilon & 0 & cos\varepsilon
\end{array}
\right)  \right\vert  &  =\left\vert \left(
\begin{array}
[c]{ccc}%
x & y & z\\
x_{1} & y_{1} & z_{1}\\
x_{2} & y_{2} & z_{2}%
\end{array}
\right)  \cdot M\right\vert \\
&  =\left\vert \left(
\begin{array}
[c]{ccc}%
x & y & z\\
x_{1} & y_{1} & z_{1}\\
x_{2} & y_{2} & z_{2}%
\end{array}
\right)  \right\vert \cdot\left\vert M\right\vert =0,
\end{align*}
Since $\left\vert M\right\vert \neq0$ and $K^{-1/2}sin\varepsilon\neq0$ if
$\varepsilon<\pi$, $\left(  x,y,z\right)  $ lies in the plane $\left(
\ref{63}\right)  $ if and only if
\[
\underline{y}=0.
\]
Since $M$ is a $K$-rigid motion it must take the shortest path from $X_{1}$ to
$X_{2}$ to the shortest path from $X_{1}\cdot M=N$ to $X_{2}\cdot M=\left(
R\cdot \sin \varepsilon,0,\cos %
\varepsilon\right)  $. But we already know that the shortest path from
$X_{1}\cdot M$ to $X_{2}\cdot M$ is the one cut out by the plane $y=0$. But
that path comes from the path cut out by the plane given by equation $\left(
\ref{63}\right)  $. This path is called the \textit{great circular arc}
between $X_{1}$ and $X_{2}$.
\end{proof}

\begin{definition}
A \textbf{line} in \textbf{SG} will be a curve that extends infinitely in each
direction and has the property that, given any two points $X_{1}$ and $X_{2}$
on the path, the shortest path between $X_{1}$ and $X_{2}$ lies along that
curve. Lines in \textbf{SG} are usually called great circles on the
$R$-sphere. They are the intersections of the $R$-sphere with planes through
$\left(  0,0,0\right)  $.
\end{definition}

Letting $X_{2}$ approach $X_{1}$ along the great circular arc joining $X_{1}$
and $X_{2}$ we see that the solution set to equation $\left(  \ref{63}\right)
$ does not change. Taking a limit as $X_{2}$ approaches $X_{1}$, this set can
also be expressed as the solution set of the equation%
\[
\left\vert \left(
\begin{array}
[c]{ccc}%
x & y & z\\
x_{1} & y_{1} & z_{1}\\
a_{1} & b_{1} & c_{1}%
\end{array}
\right)  \right\vert =0
\]
where $\left(  a_{1},b_{1},c_{1}\right)  $ is a tangent vector at the point
$X_{1}$ pointing in the direction of $X_{2}.$

\end{document}
