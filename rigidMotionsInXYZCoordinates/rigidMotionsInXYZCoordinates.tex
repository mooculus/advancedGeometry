\documentclass[newpage,hints,handout,noauthor,nooutcomes,12pt]{ximera}

%\usepackage{microtype}
%\usepackage{tikz}
\usepackage{tkz-euclide}
%\usetkzobj{all}
\tikzstyle geometryDiagrams=[rounded corners=.5pt,ultra thick,color=blue!50!black]

\usepackage{tikz-cd}

\colorlet{penColor}{blue!50!black} % Color of a curve in a plot

%% \hypersetup{
%%     colorlinks = false,
%%     }


\tikzset{%% partial ellipse
    partial ellipse/.style args={#1:#2:#3}{
        insert path={+ (#1:#3) arc (#1:#2:#3)}
    }
}

\graphicspath{
{./}
{sphericalLunesAndTriangles/}
{hyperbolicLunesAndTriangles/}
{centralProjection/}
{stereographicProjection/}
{linesAnglesAndAreasInCentralProjection/}
{linesAnglesAndAreasInStereographicProjection/}
{stereographicProjection/}
{centralProjectionInHG/}
{stereographicProjectionInHG/}
{linesInSphericalGeometry/}
{linesInHyperbolicGeometry/}
{theArtOfEscher/}
}


\newcommand{\transpose}{\intercal}
\newcommand{\eval}[1]{\bigg[ #1 \bigg]}

\renewcommand{\epsilon}{\varepsilon}
\renewcommand{\l}{\ell}
\renewcommand{\d}{\,d}

\DeclareMathOperator{\arccosh}{arccosh}
\DeclareMathOperator{\arctanh}{arctanh}
\renewcommand{\tilde}{\widetilde}
\newcommand{\R}{\mathbb R}
\newcommand{\dd}[2][]{\frac{d #1}{d #2}}
\newcommand{\pp}[2][]{\frac{\partial #1}{\partial #2}}
\newcommand{\dfn}{\textbf}

\renewcommand{\bar}{\overline}
\renewcommand{\hat}{\widehat}


\ifxake
\NewEnviron{freeResponse}{}
\fi


\title{Rigid motions in new coordinates}
\begin{document}
\begin{abstract}
  Here we dig deeper to understand our different coordinates.
\end{abstract}
\maketitle

We now wish to figure out how to convert a rigid motion $\hat M$ from
euclidean space to $(x,y,z)$-coordinates. Recall to convert a point
from $(x,y,z)$-coordinates to euclidean coordinates, we write
\[
E_R(x,y,z)=
\begin{bmatrix}
\hat{x} \\ \hat{y} \\ \hat{z}%
\end{bmatrix}
=\begin{bmatrix}
  x \\ y \\ Rz
\end{bmatrix}.
\]
We seek a rigid motion $M$ in $(x,y,z)$-coordinates that completes the diagram below:
\begin{center}
  \begin{tikzcd}[column sep=large,row sep = huge,ampersand replacement=\&]
    (x,y,z)\in\R^3\ar[r,|->,"M",dashed] \ar[d,|->,"{E_R}"] \&  (\underline{x},\underline{y},\underline{z})\in\R^3\ar[d,|->,"{E_R}"] \\
    (\hat x,\hat y,\hat z)\in\R^3\ar[r,|->,"\hat M"] \&  (\underline{\hat x},\underline{\hat y},\underline{\hat z})\in\R^3
  \end{tikzcd}
\end{center}

\begin{problem}
Using the diagram above, explain why%
\[
M
=\begin{bmatrix}
1 & 0 & 0\\
0 & 1 & 0\\
0 & 0 & R^{-1}
\end{bmatrix}
  \cdot\hat{M}\cdot\begin{bmatrix}
%
1 & 0 & 0\\
0 & 1 & 0\\
0 & 0 & R
\end{bmatrix}
\]
\begin{hint}
  Express $E_R$ in terms of a matrix. Note, this might not always be
  possible, but in this case it is!
\end{hint}

\begin{freeResponse}
We want to find $M$ in terms of $\hat{M}$ such that 
\[
\begin{bmatrix}
\underline{x} & \underline{y} & \underline{z}%
\end{bmatrix}
=\begin{bmatrix}
x & y & z
\end{bmatrix}
\cdot M.
\]
Begin in the upper-right corner of the diagram with a vector
$\begin{bmatrix}
x & y & z
\end{bmatrix}
$ in K-geometry. Multiplying by 
\[
\begin{bmatrix}
1 & 0 & 0\\
0 & 1 & 0\\
0 & 0 & R
\end{bmatrix}
\]
moves us to the left corner of the diagram and gives the image of the
vector in euclidean space. Then, multiplying by $\hat{M}$ moves us to
the bottom-left corner of the diagram and gives the transformation in
euclidean space. Now, to determine $M$ in terms of $\hat{M}$, we map
the current vector
\[
%\begin{bmatrix}
%\underline{\hat{x}} & \underline{\hat{y}} & \underline{\hat{z}}
%\end{bmatrix}=
\begin{bmatrix}
x & y & z
\end{bmatrix}
  \cdot\begin{bmatrix}
%
1 & 0 & 0\\
0 & 1 & 0\\
0 & 0 & R
\end{bmatrix}
  \cdot\hat{M}
  \]
back to K-geometry by multiplying by 
\[
\begin{bmatrix}
1 & 0 & 0\\
0 & 1 & 0\\
0 & 0 & R
\end{bmatrix}^{-1}.
\]
Then, we end up in the bottom-right corner of the diagram and have
\[
\begin{bmatrix}
\underline{x} & \underline{y} & \underline{z}%
\end{bmatrix}
=\begin{bmatrix}
x & y & z
\end{bmatrix}
  \cdot \begin{bmatrix}
%
1 & 0 & 0\\
0 & 1 & 0\\
0 & 0 & R
\end{bmatrix}
  \cdot\hat{M}\cdot\begin{bmatrix}
%
1 & 0 & 0\\
0 & 1 & 0\\
0 & 0 & R^{-1}
\end{bmatrix}.
\]

\end{freeResponse}
\end{problem}

%% So, if we let%
%% \[
%% M=\begin{bmatrix}
%% %
%% 1 & 0 & 0\\
%% 0 & 1 & 0\\
%% 0 & 0 & R
%% \end{bmatrix}
%%   \cdot\hat{M}\cdot\begin{bmatrix}
%% %
%% 1 & 0 & 0\\
%% 0 & 1 & 0\\
%% 0 & 0 & R^{-1}%
%% \end{bmatrix}
%%   ,
%% \]
%% then%
%% \begin{equation}
%% \begin{bmatrix}
%% %
%% \underline{x} & \underline{y} & \underline{z}%
%% \end{bmatrix}
%%   =\begin{bmatrix}
%% %
%% x & y & z
%% \end{bmatrix}
%%   \cdot M, \label{15}%
%% \end{equation}
%% that is $M$ is the matrix that gives the transformation $\left(
%% \ref{12}\right)  $ in $\left(  x,y,z\right)  $-coordinates.

%% So how would we check whether a transformation given in $\left(  x,y,z\right)
%% $-coordinates by a matrix $M$ preserves distances in euclidean space?
%% Again, starting from $\left(  \ref{16}\right)  $ this is just a substitution
%% problem:%
%% \begin{gather*}
%% \hat{M}\cdot\hat{M}^\transpose=I\\
%% M=\begin{bmatrix}
%% %
%% 1 & 0 & 0\\
%% 0 & 1 & 0\\
%% 0 & 0 & R
%% \end{bmatrix}
%%   \cdot\hat{M}\cdot\begin{bmatrix}
%% %
%% 1 & 0 & 0\\
%% 0 & 1 & 0\\
%% 0 & 0 & R^{-1}%
%% \end{bmatrix}
%%  \\
%% \begin{bmatrix}
%% %
%% 1 & 0 & 0\\
%% 0 & 1 & 0\\
%% 0 & 0 & R^{-1}%
%% \end{bmatrix}
%%   \cdot M\cdot\begin{bmatrix}
%% %
%% 1 & 0 & 0\\
%% 0 & 1 & 0\\
%% 0 & 0 & R
%% \end{bmatrix}
%%   =\hat{M}%
%% \end{gather*}


\begin{problem}
  Show that a matrix $M$ in $(x,y,z)$-coordinates preserves distances in
  $K$-warped space if and only if%
  \[
    M^\transpose \cdot\begin{bmatrix}
    1 & 0 & 0\\
    0 & 1 & 0\\
    0 & 0 & K^{-1}
    \end{bmatrix}
    \cdot M=\begin{bmatrix}
    1 & 0 & 0\\
    0 & 1 & 0\\
    0 & 0 & K^{-1}
    \end{bmatrix}
  \]
  where $K = 1/R^2$.
\begin{hint} 
  Explain how the previous problem gives a correspondence between a
  matrix in $K$-warped space and a matrix in euclidean space.  Then
  show that our equation above is equivalent to the corresponding
  euclidean matrix $\hat{M}$ being orthogonal.
\end{hint}
  

\begin{freeResponse}
First we assume that a transformation $M$ in K-warped space preserves
distances. In the previous problem we showed that there is a a
bijection between transformations defined by a matrix in K-warped
space and transformations defined by a matrix in euclidean
space. Therefore, if a transformation $M$ preserves distances in
K-warped space then, then $\hat{M}$ preserves distances in euclidean
space and thus is orthogonal. We have
\[
M = \begin{bmatrix}
1 & 0 & 0\\
0 & 1 & 0\\
0 & 0 & R
\end{bmatrix}
  \cdot\hat{M}\cdot \begin{bmatrix}
1 & 0 & 0\\
0 & 1 & 0\\
0 & 0 & R^{-1}
\end{bmatrix}. 
\]
Therefore, $M\cdot \begin{bmatrix}
1 & 0 & 0\\
0 & 1 & 0\\
0 & 0 & K^{-1}
\end{bmatrix}
  \cdot M^\transpose=$
\begin{align*}
 &= \begin{bmatrix}
1 & 0 & 0\\
0 & 1 & 0\\
0 & 0 & R
\end{bmatrix} \cdot \hat{M} \cdot
\begin{bmatrix}
1 & 0 & 0\\
0 & 1 & 0\\
0 & 0 & R^{-1}
\end{bmatrix} \cdot 
\begin{bmatrix}
1 & 0 & 0\\
0 & 1 & 0\\
0 & 0 & K^{-1}
\end{bmatrix} \cdot
\begin{bmatrix}
1 & 0 & 0\\
0 & 1 & 0\\
0 & 0 & R^{-1}
\end{bmatrix} \cdot \hat{M}^\transpose \cdot
\begin{bmatrix}
1 & 0 & 0\\
0 & 1 & 0\\
0 & 0 & R
\end{bmatrix}\\
&= \begin{bmatrix}
1 & 0 & 0\\
0 & 1 & 0\\
0 & 0 & R
\end{bmatrix} \cdot \hat{M} \cdot \hat{M}^\transpose \cdot
\begin{bmatrix}
1 & 0 & 0\\
0 & 1 & 0\\
0 & 0 & R
\end{bmatrix} \\
&= \begin{bmatrix}
1 & 0 & 0\\
0 & 1 & 0\\
0 & 0 & R
\end{bmatrix} \cdot
\begin{bmatrix}
1 & 0 & 0\\
0 & 1 & 0\\
0 & 0 & R
\end{bmatrix} \\
&= \begin{bmatrix}
1 & 0 & 0\\
0 & 1 & 0\\
0 & 0 & K^{-1}
\end{bmatrix}
\end{align*}

Now for the other direction we assume 
\[
M\cdot\begin{bmatrix}
1 & 0 & 0\\
0 & 1 & 0\\
0 & 0 & K^{-1}
\end{bmatrix}
  \cdot M^\transpose=\begin{bmatrix}
1 & 0 & 0\\
0 & 1 & 0\\
0 & 0 & K^{-1}
\end{bmatrix}. 
  \]
Then,
\begin{align*}
\left(\begin{bmatrix}
x & y & z
\end{bmatrix} \cdot M \right) \bullet_{K} \left( \begin{bmatrix}
x & y & z
\end{bmatrix} \cdot M \right) 
&= \begin{bmatrix}
x & y & z
\end{bmatrix} \cdot M \cdot 
\begin{bmatrix}
1 & 0 & 0\\
0 & 1 & 0\\
0 & 0 & K^{-1}
\end{bmatrix} \cdot M^\transpose \cdot
\begin{bmatrix}
x \\
y \\ 
z
\end{bmatrix} \\
&= \begin{bmatrix}
x & y & z
\end{bmatrix} \cdot 
\begin{bmatrix}
1 & 0 & 0\\
0 & 1 & 0\\
0 & 0 & K^{-1}
\end{bmatrix} \cdot 
\begin{bmatrix}
x \\ 
y \\ 
z
\end{bmatrix} \\
&= \begin{bmatrix}
x & y & z
\end{bmatrix} \bullet_{K} \begin{bmatrix}
x & y & z
\end{bmatrix}
\end{align*}
Therefore, the transformation $M$ preserves distances in $K$-warped space. 
\end{freeResponse}

\end{problem}

The equation
\[
    M^\transpose \cdot\begin{bmatrix}
    1 & 0 & 0\\
    0 & 1 & 0\\
    0 & 0 & K^{-1}
    \end{bmatrix}
    \cdot M=\begin{bmatrix}
    1 & 0 & 0\\
    0 & 1 & 0\\
    0 & 0 & K^{-1}
    \end{bmatrix}
    \]
is the condition (in $(x,y,z)$-coordinates) which affirms that
\[
\left|\frac{d\gamma(t)}{dt}\right|_K =\left|\frac{d\gamma_M(t)}{dt}\right|_K
\]
where
\[
\gamma_M(t) = M \cdot
\begin{bmatrix}
  x(t) \\ y(t) \\ z(t)
\end{bmatrix}
\]
Therefore, the (total) length of the curve $\gamma_M$ is the same as
the total length of the curve $\gamma$.

%% \begin{problem}
%% Verify that this is the correct condition by showing that any $3\times3$ matrix
%% $M$ satisfying \eqref{Korth} also satisfies%
%% \[
%% (M \cdot \mathbf v) \bullet_{K} (M \cdot \mathbf v)=
%% \mathbf v \bullet_K \mathbf v,
%% \]
%% where%
%% \[
%% \mathbf v=X_{2}-X_{1}.
%% \]
%% That is, the transformation given in $(x,y,z)$-coordinates by a matrix $M$ that
%% satisfies your condition preserves the $K$-dot product.

%% \begin{freeResponse} 
%% Let $M$ be a matrix satisfying the previous condition. 
%% \begin{align*}
%% \left(   V  \cdot M\right)  \bullet_{K}\left(   V
%% \cdot M\right)  
%% &= \left(   V  \cdot M\right) \cdot
%%  \begin{bmatrix}
%% 1 & 0 & 0\\
%% 0 & 1 & 0\\
%% 0 & 0 & K^{-1}
%% \end{bmatrix} \cdot \left(   V
%% \cdot M\right)^\transpose \\
%% &= V  \cdot M \cdot 
%%  \begin{bmatrix}
%% 1 & 0 & 0\\
%% 0 & 1 & 0\\
%% 0 & 0 & K^{-1}
%% \end{bmatrix} \cdot M^\transpose \cdot V^\transpose \\
%% &= V \cdot \begin{bmatrix}
%% 1 & 0 & 0\\
%% 0 & 1 & 0\\
%% 0 & 0 & K^{-1}
%% \end{bmatrix} \cdot V^\transpose = V\bullet_{K}V
%% \end{align*}
%% \end{freeResponse}

%% \end{problem}





\begin{definition}
A $K$-distance-preserving matrix in $K$-geometry is called a
$K$\textbf{-rigid motion} or a $K$\textbf{-congruence}.
\end{definition}

With this definition, and our work above, we make a new definition:


\begin{definition}
A $3\times3$ matrix $M$ is called \dfn{$\boldsymbol{K}$-orthogonal} if
\[
M^\transpose\cdot\begin{bmatrix}
1 & 0 & 0\\
0 & 1 & 0\\
0 & 0 & K^{-1}%
\end{bmatrix} \cdot M=\begin{bmatrix}
1 & 0 & 0\\
0 & 1 & 0\\
0 & 0 & K^{-1}%
\end{bmatrix}.
\]
\end{definition}



\begin{problem}
  For $K\ne 0$, show that if $M$ is $K$-orthogonal, then the $K$-rigid motion
  \[
  M \cdot
  \begin{bmatrix}
    x \\ y \\ z
  \end{bmatrix}=
  \begin{bmatrix}
    \underline{x} \\ \underline{y} \\ \underline{z}
  \end{bmatrix}
  \]
  takes the set of points $(x,y,z)$ such that
  \[
  1 = K\left(x^2 + y^2\right) +z^2
  \]
  to the set of points $(\underline{x},\underline{y},\underline{z})$ such that
  \[
  1=K\left(\underline{x}^2 + \underline{y}^{2}\right) + \underline{z}^{2}.
  \]
  That is, $M$ gives a one-to-one and onto mapping of $K$-sphere to
  itself.
\begin{hint}
  Explain how you can write the equation
  \[
  1=K\left(\underline{x}^2 + \underline{y}^{2}\right) + \underline{z}^{2}.
  \]
  as
\[
\begin{bmatrix}
\underline{x} & \underline{y} & \underline{z}%
\end{bmatrix}  \cdot\begin{bmatrix}
1 & 0 & 0\\
0 & 1 & 0\\
0 & 0 & K^{-1}%
\end{bmatrix}  \cdot
\begin{bmatrix}
\underline{x}\\
\underline{y}\\
\underline{z}%
\end{bmatrix}  =\frac{1}{K}.
\]
\end{hint}

\begin{freeResponse}
Beginning with the equation of the surface in $K$-geometry, 
\[
1 = K\left(x^2 + y^2\right) +z^2.
\]
We perform some algebra:
\begin{align*}
\frac{1}{K} &= \left(x^2 +y^2 \right) + \frac{z^2}{K}\\
&=
\begin{bmatrix}
x & y & z%
\end{bmatrix}  \cdot\begin{bmatrix}
1 & 0 & 0\\
0 & 1 & 0\\
0 & 0 & K^{-1}%
\end{bmatrix}  \cdot
\begin{bmatrix}
x\\
y\\
z%
\end{bmatrix} \\
&= \begin{bmatrix}
    x & y & z
  \end{bmatrix} \cdot M \cdot
\begin{bmatrix}
1 & 0 & 0\\
0 & 1 & 0\\
0 & 0 & K^{-1}%
\end{bmatrix}  \cdot M^\transpose \cdot
\begin{bmatrix}
    x \\ y \\  z
  \end{bmatrix} \\
 &= \begin{bmatrix}
    \underline{x} & \underline{y} & \underline{z}
  \end{bmatrix} \cdot 
\begin{bmatrix}
1 & 0 & 0\\
0 & 1 & 0\\
0 & 0 & K^{-1}%
\end{bmatrix}  \cdot 
\begin{bmatrix}
    \underline{x} \\ \underline{y} \\  \underline{z}
  \end{bmatrix}. \\
 &= \left(\underline{x}^2 +\underline{y}^2 \right) + \frac{\underline{z}^2}{K}
\end{align*}
Therefore,
\[
1 = K\left(\underline{x}^2 + \underline{y}^2\right) +\underline{z}^2.
\] 
\end{freeResponse}
\end{problem}




\begin{problem}
For $K\neq0$, show that the set of $K$-orthogonal matrices $M$ forms a
group.  That is, show that
\begin{enumerate}
\item multiplication of $K$-orthogonal matrices is associative, 
	\begin{hint}
	Recall that function composition is always associative.
	\end{hint}
\item the product of two $K$-orthogonal matrices is $K$-orthogonal,
\item the identity matrix is $K$-orthogonal,
\item the inverse matrix $M^{-1}$ of a $K$-orthogonal matrix $M$ is $K$-orthogonal.
\end{enumerate}

\begin{freeResponse}
\begin{enumerate}
\item Matrices are functions and function composition is associative. Therefore, matrix multiplication is associative.

\item Let $A$ and $B$ be K-orthogonal matrices. 
\begin{align*}
AB \cdot \begin{bmatrix}
1 & 0 & 0\\
0 & 1 & 0\\
0 & 0 & K^{-1}%
\end{bmatrix} \cdot
\left(AB\right)^\transpose
&= AB \cdot \begin{bmatrix}
1 & 0 & 0\\
0 & 1 & 0\\
0 & 0 & K^{-1}%
\end{bmatrix} \cdot
B^\transpose A^\transpose \\
&=A \cdot \begin{bmatrix}
1 & 0 & 0\\
0 & 1 & 0\\
0 & 0 & K^{-1}%
\end{bmatrix} \cdot
A^\transpose \\
&= \begin{bmatrix}
1 & 0 & 0\\
0 & 1 & 0\\
0 & 0 & K^{-1}%
\end{bmatrix}
\end{align*}

\item Since $I^\transpose = I $, then 
\[
I \cdot \begin{bmatrix}
1 & 0 & 0\\
0 & 1 & 0\\
0 & 0 & K^{-1}%
\end{bmatrix} \cdot
I^\transpose 
= \begin{bmatrix}
1 & 0 & 0\\
0 & 1 & 0\\
0 & 0 & K^{-1}%
\end{bmatrix}.
\]
\item
Since $M^{-1}\cdot M = I$, then $I = \left( M^{-1}\cdot M\right)^\transpose = M^\transpose \cdot \left(M^{-1}\right)^\transpose$. We know
\[
\begin{bmatrix}
1 & 0 & 0\\
0 & 1 & 0\\
0 & 0 & K^{-1}%
\end{bmatrix} =M \cdot \begin{bmatrix}
1 & 0 & 0\\
0 & 1 & 0\\
0 & 0 & K^{-1}%
\end{bmatrix} \cdot M^\transpose.
\]
Then, multiplying by $\left(M^{-1}\right)^\transpose$ on the right and $M^{-1}$ on the left we have 
\[
M^{-1} \cdot \begin{bmatrix}
1 & 0 & 0\\
0 & 1 & 0\\
0 & 0 & K^{-1}%
\end{bmatrix} \cdot  \left(M^{-1}\right)^\transpose
= M^{-1} \cdot M \cdot 
\begin{bmatrix}
1 & 0 & 0\\
0 & 1 & 0\\
0 & 0 & K^{-1}%
\end{bmatrix} \cdot M^\transpose \cdot \left(M^{-1}\right)^\transpose.
\]
Therefore, 
\[
M^{-1} \cdot \begin{bmatrix}
1 & 0 & 0\\
0 & 1 & 0\\
0 & 0 & K^{-1}%
\end{bmatrix} \cdot  \left(M^{-1}\right)^\transpose
= \begin{bmatrix}
1 & 0 & 0\\
0 & 1 & 0\\
0 & 0 & K^{-1}%
\end{bmatrix}.
\]
\end{enumerate}
\end{freeResponse}

\end{problem}

\begin{problem}
  Convert the orthogonal matrix
  \[
  \hat{M}_\theta=\begin{bmatrix}
  \cos\theta & -\sin\theta & 0\\
  \sin\theta & \cos\theta & 0\\
  0 & 0 & 1
  \end{bmatrix}
  \]
  into its $K$-orthogonal counterpart $M_\theta$. Are you surprised? Why or why
  not?
  
\begin{freeResponse}
\begin{align*}
M_\theta &= \begin{bmatrix}
1 & 0 & 0\\
0 & 1 & 0\\
0 & 0 & R%
\end{bmatrix} \cdot \hat{M}_\theta \cdot
\begin{bmatrix}
1 & 0 & 0\\
0 & 1 & 0\\
0 & 0 & R^{-1}%
\end{bmatrix} \\
&= \begin{bmatrix}
1 & 0 & 0\\
0 & 1 & 0\\
0 & 0 & R%
\end{bmatrix} \cdot 
\begin{bmatrix}
  \cos\theta & \sin\theta & 0\\
  -\sin\theta & \cos\theta & 0\\
  0 & 0 & 1
  \end{bmatrix} \cdot
\begin{bmatrix}
1 & 0 & 0\\
0 & 1 & 0\\
0 & 0 & R^{-1}%
\end{bmatrix}
= \begin{bmatrix}
  \cos\theta & \sin\theta & 0\\
  -\sin\theta & \cos\theta & 0\\
  0 & 0 & 1
  \end{bmatrix} 
\end{align*} 
It is not surprising that the matrix stays the same because $M_\theta$ rotates points around the $z$-axis in euclidean space. This is not affected in $K$-warped space because the $R$-sphere is just pressed between $z=1$ and $z=-1$.
\end{freeResponse}

\end{problem}

\begin{problem}
  Convert the orthogonal matrix
  \[
  \hat{M}_\psi=\begin{bmatrix}
  \cos\psi & 0 & -\sin\psi\\
  0 & 1 & 0\\
  \sin\psi & 0 & \cos\psi
  \end{bmatrix}
  \]
  into its $K$-orthogonal counterpart $M_\psi$. Are you surprised? Why or why not?
  
\begin{freeResponse}
\begin{align*}
M_\psi &= \begin{bmatrix}
1 & 0 & 0\\
0 & 1 & 0\\
0 & 0 & R%
\end{bmatrix} \cdot \hat{M}_\psi \cdot
\begin{bmatrix}
1 & 0 & 0\\
0 & 1 & 0\\
0 & 0 & R^{-1}%
\end{bmatrix} \\
&= \begin{bmatrix}
1 & 0 & 0\\
0 & 1 & 0\\
0 & 0 & R%
\end{bmatrix} \cdot 
\begin{bmatrix}
  \cos\psi & 0 & \sin\psi\\
  0 & 1 & 0\\
  -\sin\psi & 0 & \cos\psi
  \end{bmatrix} \cdot
\begin{bmatrix}
1 & 0 & 0\\
0 & 1 & 0\\
0 & 0 & R^{-1}%
\end{bmatrix}
=\begin{bmatrix}
  \cos\psi & 0 & R^{-1} \sin\psi\\
  0 & 1 & 0\\
  -R\sin\psi & 0 & \cos\psi
  \end{bmatrix} 
\end{align*} 
$M_\psi$ does change when converted into a $K$-orthogonal matrix because $M_\psi$ Rotates points around the $y$-axis, and this is affected since the  the $R$-sphere is pressed between $z=1$ and $z=-1$ in $K$-warped space. 
\end{freeResponse}

\end{problem}

















\subsection{Why use new coordinates?}

We have seen that we could measure the usual euclidean lengths of
curves $\hat{\gamma}$ in terms of the formulas of curves $\gamma$ in $K$-warped
space using the $K$-dot product. The short reason for this is that
\[
\dd[\hat{\gamma}]{t}\bullet\dd[\hat{\gamma}]{t}=\dd[\gamma]{t}\bullet_{K}\dd[\gamma]{t}
\]
where
\[
\dd[\gamma]{t}\bullet_{K}\dd[\gamma]{t}
=\left[\dd[\gamma]{t}\right]\cdot\begin{bmatrix}
1 & 0 & 0\\
0 & 1 & 0\\
0 & 0 & K^{-1}%
\end{bmatrix}
\cdot\left[\dd[\gamma]{t}\right]^\transpose.
\]
In other words, the usual geometry of the sphere of radius $R$ is
simply the geometry of the set
\[
\{(x,y,z)\in\R^3:1 = K\left(x^2 + y^2\right) +z^2\}
\]
with $K=1/R^{2}$ and with lengths (and areas) given by the $K$-dot
product. Said another way, we can do all of spherical geometry in
$(x,y,z)$-coordinates. All we need is the set defined by the relation
\[
1 = K\left(x^2 + y^2\right) +z^2
\]
and the $K$-dot product. But the set defined by the equation above
\textbf{continues to exist even if $\boldsymbol{K=0}$ or
  $\boldsymbol{K<0}$,} and the $K$-dot product formula continues to
make sense even if $K<0$. In short we have the following table:
\[
  {\renewcommand{\arraystretch}{2.7}
  \begin{array}{|c||c|c|c|}\hline
    & \text{Spherical (}K>0) & \text{Euclidean (}K=0) & \text{Hyperbolic (}K<0)\\
    \hline\hline
    \text{Surface in euclidean space}
    & \hat{x}^{2}+\hat{y}^{2}+\hat{z}^{2}=R^{2} & \text{DNE}  & \text{DNE} \\\hline
    \text{Euclidean dot product} & \hat{\mathbf v}^\transpose\cdot \hat{\mathbf w}
                             & \text{DNE}  & \text{DNE}\\\hline
    \text{Surface in $K$-warped space}
    & {1=K(x^{2}+y^{2})+z^{2}} & 1=K(x^{2}+y^{2})+z^{2} & 1=K(x^{2}+y^{2})+z^{2}\\\hline
    K\text{-dot product}
    & \mathbf v^\transpose \left[\begin{smallmatrix}
        1 & 0 & 0\\
        0 & 1 & 0\\
        0 & 0 & K^{-1}
      \end{smallmatrix}\right] \mathbf w &  \text{DNE}
    & \mathbf{v}^\transpose \left[\begin{smallmatrix}1 & 0 & 0\\ 0 & 1 & 0\\ 0 & 0 & K^{-1}\end{smallmatrix}\right]\mathbf w \\\hline
    \text{Rigid motions} &  \left[\begin{smallmatrix}1 & 0 & 0\\ 0 & 1 & 0\\ 0 & 0 & R^{-1}\end{smallmatrix}\right] \hat M \left[\begin{smallmatrix}1 & 0 & 0\\ 0 & 1 & 0\\ 0 & 0 & R\end{smallmatrix}\right] & \hat M & \left[\begin{smallmatrix}1 & 0 & 0\\ 0 & 1 & 0\\ 0 & 0 & \sqrt{K}\end{smallmatrix}\right] \hat M \left[\begin{smallmatrix}1 & 0 & 0\\ 0 & 1 & 0\\ 0 & 0 & \sqrt{K^{-1}}\end{smallmatrix}\right] \\\hline
\end{array}}
\]



This table tells us that `there is something else out there,' that is,
some other type of two-dimensional geometry beyond plane geometry and
spherical geometry. But the gap in the bottom row of the table is a
bit disturbing. If we can't express the usual dot-product in plane
geometry as the $K$-dot product for $K=0$, we can't pass smoothly from
spherical through plane geometry to hyperbolic geometry using
$(x,y,z)$-coordinates. Later, we will examine two ways to produce coordinates
uniformly for spherical, plane and hyperbolic geometry that overcome
this difficulty. 



\begin{problem}
Summarize the results from this section. In particular, indicate which
results follow from the others.
\begin{freeResponse}
In this section we learned the definition of a \textit{$K$-rigid
  motion}, a $K$-distance-preserving matrix in $K$-geometry. Next we
gave a necessary and sufficient condition for a matrix to be a
$K$-rigid motion: A matrix $M$ is a $K$-rigid motion if and only if
\[
M\cdot\begin{bmatrix}
1 & 0 & 0\\
0 & 1 & 0\\
0 & 0 & K^{-1}%
\end{bmatrix}  \cdot M^\transpose=\begin{bmatrix}
1 & 0 & 0\\
0 & 1 & 0\\
0 & 0 & K^{-1}%
\end{bmatrix}.
\]
In other words, a matrix $M$ is a $K$-rigid motion if and only if it is $K$-orthogonal.
We proved that the set of $K$-orthogonal matrices (and hence the $K$-rigid motions defined by matrices) form a group. 
We also showed that the $K$-dot product is preserved by $K$-rigid motions.
\end{freeResponse}
\end{problem}


\end{document}
