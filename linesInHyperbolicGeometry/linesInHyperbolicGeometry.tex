\documentclass[handout,newpage,hints,,12pt,noauthor,nooutcomes]{ximera}

%\usepackage{microtype}
%\usepackage{tikz}
\usepackage{tkz-euclide}
%\usetkzobj{all}
\tikzstyle geometryDiagrams=[rounded corners=.5pt,ultra thick,color=blue!50!black]

\usepackage{tikz-cd}

\colorlet{penColor}{blue!50!black} % Color of a curve in a plot

%% \hypersetup{
%%     colorlinks = false,
%%     }


\tikzset{%% partial ellipse
    partial ellipse/.style args={#1:#2:#3}{
        insert path={+ (#1:#3) arc (#1:#2:#3)}
    }
}

\graphicspath{
{./}
{sphericalLunesAndTriangles/}
{hyperbolicLunesAndTriangles/}
{centralProjection/}
{stereographicProjection/}
{linesAnglesAndAreasInCentralProjection/}
{linesAnglesAndAreasInStereographicProjection/}
{stereographicProjection/}
{centralProjectionInHG/}
{stereographicProjectionInHG/}
{linesInSphericalGeometry/}
{linesInHyperbolicGeometry/}
{theArtOfEscher/}
}


\newcommand{\transpose}{\intercal}
\newcommand{\eval}[1]{\bigg[ #1 \bigg]}

\renewcommand{\epsilon}{\varepsilon}
\renewcommand{\l}{\ell}
\renewcommand{\d}{\,d}

\DeclareMathOperator{\arccosh}{arccosh}
\DeclareMathOperator{\arctanh}{arctanh}
\renewcommand{\tilde}{\widetilde}
\newcommand{\R}{\mathbb R}
\newcommand{\dd}[2][]{\frac{d #1}{d #2}}
\newcommand{\pp}[2][]{\frac{\partial #1}{\partial #2}}
\newcommand{\dfn}{\textbf}

\renewcommand{\bar}{\overline}
\renewcommand{\hat}{\widehat}


\ifxake
\NewEnviron{freeResponse}{}
\fi


\title{Lines in hyperbolic geometry}

\begin{document}
\begin{abstract}
Here we examine ``lines'' in hyperbolic geometry and prove a
hyperbolic version of the Pythagorean Theorem.
\end{abstract}
\maketitle



We next will figure out what is the shortest path you can take between
two points on the surface defined by $K(x^2+y^2)+z^2=1$ when $K$ is
negative. This surface is called a \textbf{two-sheeted
  hyperboloid}. Since $K$ is negative, we must do our calculation
using only $(x,y,z)$-coordinates. However, this will allow us to see
the full power of working in $K$-warped space, since our work will be
essentially the same as when $K$ was positive.




\subsection{A refresher on hyperbolic trigonometric functions}


Just as $(\cos\tau,\sin\tau)$ parametrize the unit circle,
hyperbolic functions parametrize the `unit' hyperbola.
\begin{center}
\begin{tikzpicture}[geometryDiagrams,scale=2]  
      %% x-axis
      \draw[thin,->] (-1.2,0)--(1.2,0); 
      \node[right] at (1.2,0) {$x$};

      %% y-axis
      \draw[thin,->] (0,-1.2)--(0,1.2); 
      \node[above] at (0,1.2) {$y$};
     
      
      \draw[domain=0:360,smooth] plot ({cos(\x)},{sin(\x)});
      \draw[fill=black] ({cos(57)},{sin(57)}) circle (.03);
      \node[above right] at ({cos(57)},{sin(57)}) {$(\cos\tau,\sin\tau)$};


      \draw[thin,dashed]  ({cos(57)},{sin(57)})--({cos(57)},0);
      \draw[thin,dashed]  ({cos(57)},{sin(57)})--(0,{sin(57)});

      
      %% x-axis
      \draw[thin,->] (2.8,0)--(5.2,0); 
      \node[right] at (5.2,0) {$x$};

      %% y-axis
      \draw[thin,->] (3,-1.2)--(3,1.2); 
      \node[above] at (3,1.2) {$y$};
     
      
      %% hyperbola
      \draw[domain=-1.2:1.2,smooth] plot ({3+cosh(\x)},{sinh(\x)});
      \draw[fill=black] ({3+cosh(.8)},{sinh(.8)}) circle (.03);
      \node[below right] at ({3+cosh(.8)},{sinh(.8)}) {$(\cosh\sigma,\sinh\sigma)$};

      \draw[thin,dashed]  ({3+cosh(.8)},{sinh(.8)})--({3+cosh(.8)},0);
      \draw[thin,dashed]  ({3+cosh(.8)},{sinh(.8)})--(3,{sinh(.8)});
\end{tikzpicture}  
\end{center}
where
\begin{align*}
  \cos\tau &= 1-\frac{\tau^2}{2!}+\frac{\tau^4}{4!}-\frac{\tau^6}{6!}+ \cdots,  & \cosh\sigma &= 1+\frac{\sigma^2}{2!}+\frac{\sigma^4}{4!}+\frac{\sigma^6}{6!}+ \cdots,\\
  \sin\tau &= \tau-\frac{\tau^3}{3!}+\frac{\tau^5}{5!}-\frac{\tau^7}{7!}+ \cdots,  & 
  \sinh\sigma &= \sigma+\frac{\sigma^3}{3!}+\frac{\sigma^5}{5!}+\frac{\sigma^7}{7!}+ \cdots.
\end{align*}

\begin{problem}
  Look-up  geometrical meaings of $\sigma$ and $\tau$ and explain
  them here, using words and pictures as necessary.
\end{problem}




\subsection{Hyperbolic coordinates, a shortest path from the North Pole}

To parametrize the surface $K(x^2+y^2) + z^2 = 1$ we define
\begin{align*}
  x(\sigma,\tau) &=|K|^{-1/2}\cdot \sinh \sigma\cdot\cos \tau,\\
  y(\sigma,\tau) &=|K|^{-1/2}\cdot\sinh\sigma\cdot\sin \tau,\\
  z(\sigma,\tau) &=\cosh \sigma,
\end{align*}
where $0\le \sigma< \infty$ and $0\le \tau<2\pi$.


\begin{center}
  \begin{tikzpicture}[geometryDiagrams,scale=1]  
      %% x-axis
      \draw [thin,->] (2.2,1.1)--(-2.2,-1.1); 
      \node[below left] at (-2.2,-1.1) {$x$};

      %% y-axis
      \draw [thin,->] (-2.2,0)--(2.5,0); 
      \node[right] at (2.5,0) {$y$};

       %% z-axis
      \draw[thin,->] (0,-1)--(0,3.5); 
      \node[above] at (0,3.5) {$z$};

      
      
      %% 2-sheeted hyperboloid
      \draw[domain=-3.5:3.5] plot (\x,{sqrt(1.75^2+\x*\x)});
      %%%% Equator
      \draw (3.5,4) arc (0:360:3.5 and .8);
      \node[right] at (3.5,4) {$K(x^2+y^2) + z^2 = 1$};
      \node[right] at (3.5,3.5) {$K<0$};

      

      
      
      %%%% line to point
      %\draw[thin,gray] (0,0)--(.88,1.21);
      \draw[thin,gray] (0,0)--(.88,-.2)--(.88,2.3);

      
      %%%% point on sphere
      \draw[fill=black] (.88,2.3) circle (.05);
      \node at (1.7,1.6) {$(x,y,z)$};

      %% angles:
      \draw[thin,->] (-.5,-.25) arc (206:342:.6 and .3);
      \node at (.15,-.25) {$\tau$};
            
      %% \draw[thin,->] (0,.6) arc (90:60:.6);
      %% \node at (.1,.3) {$\sigma$}; 
  \end{tikzpicture}  
\end{center}



\begin{problem}
Show that these hyperbolic coordinates do actually parametrize the
$K$-sphere when $K$ is negative.
\begin{hint}
  Remember,
  \[
  -\sinh^2\sigma + \cosh^2\sigma =1.
  \]
\end{hint}
\begin{hint}
  This is an exercise in ``double-containment.'' To show the one
  direction, show
\[
K\left(x(\sigma,\tau)^{2}+y(\sigma,\tau)^{2}\right)+z(\sigma,\tau)^{2}=1
\]
for all $(\sigma,\tau)$. To show the other direction, appeal to the
diagram above.
\end{hint}
\begin{freeResponse}
  We will first show that the set determined by
  \[
  \left(x(\sigma,\tau), y(\sigma,\tau), z(\sigma,\tau)\right)
  \]
  lies on the surface
  \[
  K\left(x(\sigma,\tau) ^{2}+y(\sigma,\tau) ^{2}\right) +z(\sigma,\tau)^{2} = 1
  \]
  where $K<0$. Substituting we have
  $K\left(x(\sigma,\tau) ^{2}+y(\sigma,\tau) ^{2}\right)
  +z(\sigma,\tau)^{2}$
  \begin{align*}
    &=K\left((|K|^{-1/2}\cdot\sinh\sigma\cdot\cos\tau)^{2}+(|K|^{-1/2}\cdot\sinh\sigma\cdot\sin\tau)^{2}\right) +(\cosh\sigma)^{2} \\
    &= -\sinh^2\sigma\cdot\cos^2\tau-\sinh^2\sigma\sin^2\tau + \cosh^2\sigma \\
    &= -\sinh^2\sigma(\cos^2\tau+\sin^2\tau) + \cosh^2\sigma \\
    &= -\sinh^2\sigma + \cosh^2\sigma \\
    &=1.
  \end{align*}
  Now we must show that every point on the surface can be obtained from
  \[
  \left(x(\sigma,\tau), y(\sigma,\tau), z(\sigma,\tau)\right).
  \]
  We see this from the diagram above. 
\end{freeResponse}
\end{problem}

Just as we did on the $R$-sphere, we can write a path on the
$K$-sphere by giving a path in the $(\sigma,\tau)$-plane. Again, we
will need to figure out the $K$-dot product in
$(\sigma,\tau)$-coordinates so that we can compute the lengths of
paths in these coordinates. Our map from the $(\sigma,\tau)$ plane to $K$-warped space will be
\begin{center}
\begin{tikzcd}[column sep=large,row sep = huge,ampersand replacement = \&]
  (\sigma,\tau)\in \R^2 \ar[r,|->,"H_K"] \& (x, y , z) \in \R^3
\end{tikzcd}
\end{center}
where
\[
H_K(\sigma,\tau) = 
\begin{bmatrix}
x(\sigma,\tau)  \\
y(\sigma,\tau)  \\
z(\sigma,\tau)  
\end{bmatrix}
=
\begin{bmatrix}
  |K|^{-1/2}\cdot \sinh\sigma\cdot \cos \tau  \\
  |K|^{-1/2}\cdot \sinh\sigma\cdot \sin\tau\\
  |K|^{-1/2}\cdot \cosh \sigma \\
\end{bmatrix}.
\]




\begin{problem}
  Suppose we have a curve $\gamma$ in $K$-warped space which we can decompose as
    \begin{center}
    \begin{tikzcd}[column sep=large,row sep = huge,ampersand replacement=\&]
      t\in\R \ar[r,|->,"\gamma_\mathrm{hyp}"] \ar[rd,|->,swap,"\gamma=H_K\circ \gamma_\mathrm{hyp}"] \&  (\sigma(t),\tau(t))\in\R^2\ar[d,|->,"{H_K}"] \\
      \& \gamma(t)=( x(t), y(t), z(t))\in\R^3 
    \end{tikzcd}
    \end{center}
    Now use your previous work to:
    \begin{enumerate}
    \item Find $D_{\mathrm{hyp}}$ such
      that
      \[
      \begin{bmatrix}
        dx/dt\\ dy/dt \\ dz/dt
      \end{bmatrix}
      = D_{\mathrm{hyp}} \cdot
      \begin{bmatrix}
        d\sigma/dt \\ d\tau/dt
      \end{bmatrix}
      \]
    \item Find $P_\mathrm{hyp}$ such that
      \[
      \left(\dd[x]{t}, \dd[y]{t}, \dd[z]{t}\right)\bullet_K
      \left(\dd[x]{t}, \dd[y]{t}, \dd[z]{t}\right)
      =
      \begin{bmatrix}
        \dd[\sigma]{t} &  \dd[\tau]{t}
      \end{bmatrix}
      \cdot P_\mathrm{hyp}
      \cdot
      \begin{bmatrix}
        d\sigma/dt \\  d\tau/dt
      \end{bmatrix}.
      \]
    \end{enumerate}
    
\begin{freeResponse}
  \[
  D_{\mathrm{hyp}} =
  \begin{bmatrix}
    \pp[x]{\sigma} & \pp[y]{\sigma} & \pp[z]{\sigma} \\
    \pp[x]{\tau}   & \pp[y]{\tau}   & \pp[z]{\tau}
  \end{bmatrix}.
  \]
  And write
     \[
     P_\mathrm{hyp} =
  \begin{bmatrix}
    |K|^{-1} & 0 \\
    0 & |K|^{-1}\cdot\sinh^2 \sigma
  \end{bmatrix}.
  \]
  \end{freeResponse}
\end{problem}



\begin{definition}
  Let $\mathbf{v}_\mathrm{hyp}$ and $\mathbf{w}_\mathrm{hyp}$ be a vectors in
  $(\sigma,\tau)$-coordinates originating at the same
  $(\sigma,\tau)$-coordinate. Define
  \[
  \mathbf{v}_\mathrm{hyp} \bullet_\mathrm{hyp} \mathbf{w}_\mathrm{hyp} = \mathbf{v}_\mathrm{hyp}^\transpose \cdot P_\mathrm{hyp} \cdot \mathbf{w}_\mathrm{hyp}
  \]
  where
  \[
     P_\mathrm{hyp} =
  \begin{bmatrix}
    |K|^{-1} & 0 \\
    0 & |K|^{-1}\cdot\sinh^2 \sigma
  \end{bmatrix}
  \]
  and $\sigma$ is determined by the coordinate that the vectors
  originate from.
\end{definition}




Now notice that you can write a path on the $R$-sphere by giving a
path
\[
\gamma_\mathrm{hyp}(t) = \left( \sigma(t),\tau(t)\right)
\]
in the $(\sigma,\tau)$-plane.  To write a path that starts at the
North Pole, $(0,0,1)$, demand that $\sigma(0)=0$.  If you want the path
to end on the plane $y=\hat{y}=0$, demand additionally that
$\tau(\varepsilon)=0$. Given such a path, its length is given by the formula%
\[
L=\int_{0}^\varepsilon \sqrt{\left(\frac{d\sigma}{dt},\frac{d\tau}{dt}\right)
  \bullet_\mathrm{hyp} \left(\frac{d\sigma}{dt},\frac{d\tau}{dt}\right)}\, dt
 \]
 since $\frac{d\gamma_\mathrm{hyp}}{d t} \dot_\mathrm{hyp} \frac{d\gamma_\mathrm{hyp}}{d t}
 = \frac{d\gamma}{d t} \dot_K \frac{d\gamma}{d t} = \frac{d\hat \gamma}{d t} \dot \frac{d\hat\gamma}{d t}$.













\begin{problem}
  Prove that the shortest path on the $K$-surface from the North Pole
  \[
  N=\left( |K|^{-1/2}\cdot \sinh 0\cdot  \cos 0,|K|^{-1/2}\cdot \sinh  0\cdot \sin 0,\cosh
  0\right)
  \]
  to a point
  \[
  (x,y,z)=\left(|K|^{-1/2}\sinh \varepsilon,0,\cosh \varepsilon\right)
  \]
  is the path lying in the plane $y=0$.
  \begin{hint}
    Use the same steps you did in the case of the sphere.
  \end{hint}

  \begin{freeResponse}
    Write
    \begin{align*}
    L  &=\int_{0}^{\varepsilon} \sqrt{\left(1,\dd[\tau]{\sigma}\right)\bullet_\mathrm{hyp} \left(1,\dd[\tau]{\sigma}\right)}\d\sigma\\
    &= \int_{0}^{\varepsilon} \sqrt{
      \begin{bmatrix} 1 & \dd[\tau]{\sigma}
      \end{bmatrix} \cdot P_\mathrm{hyp}\cdot
      \begin{bmatrix} 1 & \dd[\tau]{\sigma}
      \end{bmatrix}^\transpose}\d\sigma \\
    &= \int_{0}^{\varepsilon} \sqrt{
      \begin{bmatrix} 1 & \dd[\tau]{\sigma}
      \end{bmatrix}
      \begin{bmatrix}
        |K|^{-1} & 0 \\
        0 & |K|^{-1}\cdot\sinh^2 \sigma
      \end{bmatrix}
      \begin{bmatrix} 1 \\ \dd[\tau]{\sigma}
    \end{bmatrix}}\d\sigma \\
    &= \int_{0}^{\varepsilon} \sqrt{
      |K|^{-1}+|K|^{-1}\sinh^{2}\sigma\cdot \left(\frac{d\tau }{d\sigma}\right)^{2}
    }\d\sigma \\
    &= |K|^{-1/2}\int_{0}^{\varepsilon} \sqrt{
      1+\sinh^{2}\sigma\cdot \left(\frac{d\tau }{d\sigma}\right)^{2}
    }\d\sigma.
    \end{align*}
   Since $\sinh^{2}\sigma$ is is positive for almost all $\sigma\in[
     0,\varepsilon] $, $L$ is minimal only when
   $\frac{d\tau}{d\sigma}$ is identically $0$. But this means that
   $\tau\left( \sigma\right) $ is a constant function. Since
   $\tau\left( 0\right) =0$, this means that $\tau\left( \sigma\right)
   $ is identically $0$. Hence our path is determined by
   \begin{align*}
     x(\sigma,\tau) &=|K|^{-1/2}\cdot \sinh\sigma,\\
     y(\sigma,\tau) &=0,\\
     z(\sigma,\tau) &=\cosh \sigma,
   \end{align*}
   and this corresponds to a hyperbolic path embedded in the plane
   $y=\hat{y}=0$.
  \end{freeResponse}

\end{problem}








\subsection{Shortest path between any two points}

Just as we proved in spherical geometry that the shortest path is the
path cut out by
\[
K(x^{2}+y^{2})+z^{2}=1
\]
and the plane containing the origin and the two points in question, we
will see that a completely analogous result is true in hyperbolic
geometry. Before we start, we will need one more class of rigid
motions to add to our collection.


\begin{problem}
  Assuming $K$ is negative, show
  \[
  \begin{bmatrix}
    1 & 0 & 0 \\
    0 & 1 & 0 \\
    0 & 0 & \sqrt{K}
  \end{bmatrix}
  \cdot 
  \hat N_{-i\psi}
  \begin{bmatrix}
    1 & 0 & 0 \\
    0 & 1 & 0\\
    0 & 0 & \sqrt{K^{-1}}
  \end{bmatrix}
=
  \begin{bmatrix}
    \cosh\psi & 0 & |K|^{-1/2}\cdot\sinh\psi\\
    0 & 1 & 0\\
    |K|^{1/2}\cdot\sinh\psi & 0 & \cosh\psi
  \end{bmatrix}
  \]
\end{problem}




\begin{problem}
  Assuming $K$ is negative, show
  \[
   N_\psi=
  \begin{bmatrix}
    \cosh\psi & 0 & |K|^{-1/2}\cdot\sinh\psi\\
    0 & 1 & 0\\
    |K|^{1/2}\cdot\sinh\psi & 0 & \cosh\psi
  \end{bmatrix}
  \]
  is a $K$-rigid motion.
  \begin{freeResponse}
    We will show that $N_\psi$ is $K$ orthogonal. Write
    \[
    N_\psi
      \begin{bmatrix}
        1 & 0 & 0 \\
        0 & 1 & 0 \\
        0 & 0 & K^{-1}
      \end{bmatrix}
      N_\psi^\transpose
    \]
    \begin{align*}
  &=
      \begin{bmatrix}
    \cosh\psi & 0 & |K|^{1/2}\cdot\sinh\psi\\
    0 & 1 & 0\\
    |K|^{-1/2}\cdot\sinh\psi & 0 & \cosh\psi
      \end{bmatrix}
      \begin{bmatrix}
        1 & 0 & 0 \\
        0 & 1 & 0 \\
        0 & 0 & K^{-1}
      \end{bmatrix}
 \begin{bmatrix}
    \cosh\psi & 0 & |K|^{-1/2}\cdot\sinh\psi\\
    0 & 1 & 0\\
    |K|^{1/2}\cdot\sinh\psi & 0 & \cosh\psi
 \end{bmatrix} \\
&=\begin{bmatrix}
    \cosh\psi & 0 & -|K|^{-1/2}\cdot\sinh\psi\\
    0 & 1 & 0\\
    |K|^{-1/2}\cdot\sinh\psi & 0 &  K^{-1}\cosh\psi
      \end{bmatrix}
 \begin{bmatrix}
    \cosh\psi & 0 & |K|^{-1/2}\cdot\sinh\psi\\
    0 & 1 & 0\\
    |K|^{1/2}\cdot\sinh\psi & 0 & \cosh\psi
 \end{bmatrix}\\
&=\begin{bmatrix}
    \cosh^2\psi -\sinh^2\psi & 0 & |K|^{-1/2}\cdot\cosh\psi\cdot\sinh\psi- |K|^{-1/2}\cdot\cosh\psi\cdot\sinh\psi\\
    0 & 1 & 0\\
    |K|^{-1/2}\cdot\cosh\psi\cdot\sinh\psi- |K|^{-1/2}\cdot\cosh\psi\cdot\sinh\psi & 0 &  K^{-1}(\cosh^2\psi -\sinh^2\psi )
 \end{bmatrix}\\
 &=\begin{bmatrix}
        1 & 0 & 0 \\
        0 & 1 & 0 \\
        0 & 0 & K^{-1}
      \end{bmatrix}.
    \end{align*}
    Hence $N_\psi$ is $K$-orthogonal.
  \end{freeResponse}
\end{problem}







\begin{problem}
  Assuming $K$ is negative, consider
  \[
  N_\psi=
  \begin{bmatrix}
    \cosh\psi & 0 & |K|^{-1/2}\cdot\sinh\psi\\
    0 & 1 & 0\\
    |K|^{1/2}\cdot\sinh\psi & 0 & \cosh\psi
  \end{bmatrix}.
  \]
  Can you describe geometrically what this mapping is doing to the
  points in $K$-warped space?
  \begin{hint}
    First look at the image of the point $(0,0,1)$.
  \end{hint}
  \begin{freeResponse}
    This rigid motion ``rotates'' along the $K$-surface around the $y$-axis:
    \begin{image}
      \includegraphics[width=3in]{hyperRigidMotion.jpg}
    \end{image}
  \end{freeResponse}
\end{problem}


\begin{theorem}
Given any two points $X_{1}=\left(x_{1},y_{1},z_{1}\right) $ and
$X_{2}=\left(x_{2},y_{2},z_{_{2}}\right) $ in $K$-geometry, the
shortest path between the two points is the path cut out by the set
\[
K\left(  x^{2}+y^{2}\right)  +z^{2}=1
\]
and the plane containing $(0,0,0)$, $X_{1}$, and $X_{2}$.
\end{theorem}


\begin{problem}
  Explain in words, with pictures as needed, how to prove this theorem
  by using the $K$-rigid motions
  \[
  M_\theta=
  \begin{bmatrix}
    \cos\theta & -\sin\theta & 0\\
    \sin\theta & \cos\theta & 0\\
    0 & 0 & 1
  \end{bmatrix}
  \qquad\text{and}\qquad
  N_\psi=
  \begin{bmatrix}
    \cosh\psi & 0 & |K|^{-1/2}\cdot\sinh\psi\\
    0 & 1 & 0\\
    |K|^{1/2}\cdot\sinh\psi & 0 & \cosh\psi
  \end{bmatrix}.
  \]
  \begin{hint}
    $M_\theta$ is a $K$-rigid motion that rotates around the $z$-axis and $N_\psi$
    is a $K$-rigid motion that ``slides'' the $K$-surface past the $y$-axis.
  \end{hint}
  \begin{hint}
    You should apply \textit{two} $K$-rigid motions of the form
    $M_\theta$ (for different angles) and one $K$-rigid motion of the
    form $N_\psi$---though not necessarily in that order!
  \end{hint}
  \begin{freeResponse}
    Given $X_1$ and $X_2$, we may instead consider $X_1\cdot M_\theta$
    and $X_2\cdot M_\theta$ where $\theta$ is chosen so that $X_1\cdot
    M_\theta$ is in the plane $y=0$.

    Next consider $X_1\cdot M_\theta\cdot N_\psi$ and $X_2\cdot
    M_\theta\cdot N_\psi$ where $\psi$ is chosen so that $X_1\cdot
    M_\theta\cdot N_\psi$ is at the North Pole.

    Finally consider $X_1\cdot M_\theta\cdot N_\psi\cdot M_\varphi$
    and $X_2\cdot M_\theta\cdot N_\psi\cdot M_\varphi$ where $\varphi$
    is chosen so that $X_1\cdot M_\theta\cdot N_\psi\cdot M_\varphi$
    is at the North Pole and $X_2\cdot M_\theta\cdot N_\psi\cdot
    M_\varphi$ is in the plane $y=0$.

    Now by our previous work, the shortest path between $X_1\cdot
    M_\theta\cdot N_\psi\cdot M_\varphi$ and $X_2\cdot M_\theta\cdot
    N_\psi\cdot M_\varphi$ is the curve formed by intersecting
    \[
    K(x^2+y^2)+z^2=1
    \]
    and the plane $y=0$. Applying $M_\varphi^{-1}\cdot
    N_\psi^{-1}\cdot M_\theta^{-1}$ to our ``moved'' points will move
    them back again, and the shortest path is the transformed curve.
  \end{freeResponse}
\end{problem}




\begin{definition}
A \dfn{line} in hyperbolic geometry will be a curve that extends
infinitely in each direction and has the property that, given any two
points $X_{1}$ and $X_{2}$ on the path, the shortest path between
$X_{1}$ and $X_{2}$ lies along that curve. Lines in hyperbolic
geometry are the intersections of the $K$-geometry with planes through
$(0,0,0)$. The length of the shortest path between two points in
$K$-geometry will be called the $K$-distance.
\end{definition}


\subsection{The hyperbolic Pythagorean Theorem}

To start we need some basic facts about lengths of lines in hyperbolic
geometry.

\begin{problem}
  Given a line in hyperbolic geometry lying entire in the plane
  $y=0$,
  \begin{align*}
    x(t) &= |K|^{-1/2}\sinh t,\\
    y(t) &= 0,\\
    z(t) &= \cosh(t),
  \end{align*}
  show that the length of the segment on the interval $0 \leq t \leq
  \varepsilon$ is exactly $|K|^{-1/2}\varepsilon$.
  \begin{hint}
    Use a previous problem.
  \end{hint}
  \begin{freeResponse}
    By a previous problem the length of this line is given by
    \[
    L=\int_{0}^{\varepsilon} 1 \d\sigma = |K|^{-1/2}e.
    \]
  \end{freeResponse}
\end{problem}

\begin{problem}
  Explain in words how to prove that given two points on the surface
  \[
  K(x^2 + y^2) + z^2 =1,
  \]
  say $A$ and $B$, the length of the hyperbolic line connecting them
  is given by
  \[
  |K|^{-1/2}\cdot \varepsilon = |K|^{-1/2} \cdot \mathrm{arcosh}
  \left(\frac{A\bullet_K B}{|A|_K\cdot |B|_K}\right).
  \]
  by using the $K$-rigid motions
  \[
   M_\theta=
  \begin{bmatrix}
    \cos\theta & -\sin\theta & 0\\
    \sin\theta & \cos\theta & 0\\
    0 & 0 & 1
  \end{bmatrix}
  \qquad\text{and}\qquad
 N_\psi=
  \begin{bmatrix}
    \cosh\psi & 0 & |K|^{-1/2}\cdot\sinh\psi\\
    0 & 1 & 0\\
    |K|^{1/2}\cdot\sinh\psi & 0 & \cos\psi
  \end{bmatrix}.
  \]
  \begin{freeResponse}
    The problem is essentially the same as a previous problem. Hence one must consider
    \[
    A\cdot M_\theta\cdot N_\psi\cdot M_\varphi \qquad\text{and}\qquad
    B\cdot M_\theta\cdot N_\psi\cdot M_\varphi
    \]
    where $\theta$, $\psi$, and $\varphi$ are chosen to place $A$ on
    the North Pole and $B$ in the plane $y=0$.

    Now we know the distance between the points to be
    $|K|^{-1/2}\cdot\varepsilon$. Since $K$-rigid motions preserve
    distance and angle, this completes the proof.
  \end{freeResponse}
\end{problem}


We will now give the hyperbolic analogue of the Pythagorean Theorem.

\begin{theorem}[Hyperbolic Pythagorean Theorem]
  If $\triangle ABC$ is a right triangle on the surface
  \[
  K(x^2+y^2)+z^2\qquad\text{where}\qquad K<0
  \]
  with right angle $\angle ACB$, and side $a$ opposite $A$,
  $b$ opposite $B$, and $c$ opposite $C$, then
  \[
  \cosh\left(|K|^{1/2}\cdot c\right)=\cosh\left(|K|^{1/2}\cdot a\right)\cosh\left(|K|^{1/2}\cdot b\right).
  \]
\end{theorem}

Let's see why this theorem is true.  We may via $K$-rigid motions
place the triangle so that $C$ is at the North Pole, $A$ is in the
plane $y=0$, and $B$ is in the plane $x=0$ (note $A$ and $B$ may
be switched---if this is the case, simply rename them). In this case,
\begin{align*}
  A &= (|K|^{-1/2}\cdot \sinh\alpha, 0, \cosh\alpha),\\
  B &= (0, |K|^{-1/2}\cdot \sinh \beta, \cosh\beta).
\end{align*}
\begin{image}
  \includegraphics[width=4in]{hypPythag.png}
\end{image}
Hence the length of side $b$ is $|K|^{-1/2}\cdot\alpha$. Using a rigid motion of the form
\[
M_\theta=
\begin{bmatrix}
  \cos\theta & -\sin\theta & 0\\
  \sin\theta & \cos\theta & 0\\
  0 & 0 & 1
\end{bmatrix}
\]
when $\theta = \pi/2$ we see that the length of side $a$ is $|K|^{-1/2}\cdot
\beta$. Set
\[
\varepsilon = \mathrm{arcosh}\left(\frac{A\bullet_K B}{|A|_K\cdot |B|_K}\right).
\]
Since we are working on the $K$-surface,
\begin{align*}
  K^{-1}\cdot \cosh \varepsilon &= A\bullet_K B\\
  &=
  \begin{bmatrix}
    |K|^{-1/2}\cdot \sinh\alpha &  0 & \cosh\alpha
  \end{bmatrix}
    \begin{bmatrix}
      1 & 0 & 0\\
      0 & 1 & 0\\
      0 & 0 & K^{-1}
    \end{bmatrix}
    \begin{bmatrix}
      0\\
      |K|^{-1/2}\cdot\sinh\beta\\
      \cosh\beta
    \end{bmatrix}\\
   &=K^{-1} \cdot \cosh\alpha \cdot \cosh\beta.
\end{align*}


\begin{problem}
  Explain how to progress from the fact that
  \[
  K^{-1}\cdot \cosh \varepsilon = K^{-1} \cdot \cosh\alpha \cdot \cosh\beta.
  \]
  to the conclusion of the theorem
  \[
  \cosh(|K|^{1/2}\cdot c)=\cosh(|K|^{1/2} \cdot a)\cosh(|K|^{1/2}\cdot b).
  \]
  \begin{freeResponse}
    As we have seen
    \[
    \begin{split}
      |K|^{-1/2}\cdot \alpha &= a,\\
      |K|^{-1/2}\cdot \beta  &= b,\\
      |K|^{-1/2}\cdot \varepsilon &= c,
    \end{split}
    \qquad\text{so}\qquad
    \begin{split}
      \alpha &= |K|^{1/2} \cdot a,\\
      \beta  &= |K|^{1/2} \cdot b,\\
      \varepsilon &= |K|^{1/2} \cdot c,\\
    \end{split}
    \]
    and hence
    \[
      \cosh\left(|K|^{1/2}\cdot c\right)=\cosh\left(|K|^{1/2} \cdot a\right)\cosh\left(|K|^{1/2}\cdot b\right).
    \]
  \end{freeResponse}
\end{problem}


\begin{problem}
  Use the Taylor series expansion of $\cosh(x)$ centered around $x=0$,
  \[
  \cosh(x) = 1 + \frac{x^2}{2!} + \frac{x^4}{4!} + \frac{x^6}{6!} + \cdots
  \]
to show that for ``small'' triangles, the hyperbolic Pythagorean
Theorem reduces to the euclidean Pythagorean Theorem, meaning
\[
c^2 \approx a^2+b^2.
\]
\begin{freeResponse}
  If
  \[
  \cosh\left(|K|^{1/2}\cdot c\right)=\cosh\left(|K|^{1/2}\cdot a\right)\cosh\left(|K|^{1/2}\cdot b\right),
  \]
  then we may replace each cosine with its Taylor series expansion
  \[
  1 + \frac{\left(|K|^{1/2}\cdot c\right)^2}{2!} + \frac{\left(|K|^{1/2}\cdot c\right)^4}{4!} +  \cdots
  \]
  \[
  =\left(
  1 + \frac{\left(|K|^{1/2}\cdot a\right)^2}{2!} + \frac{\left(|K|^{1/2}\cdot a\right)^4}{4!} +  \cdots
  \right)
  \left(
  1 + \frac{\left(|K|^{1/2}\cdot b\right)^2}{2!} + \frac{\left(|K|^{1/2}\cdot b\right)^4}{4!} + \cdots
  \right).
  \]
  Expanding this out, and discarding higher-order terms (as they will
  go to zero when $a$, $b$, and $c$ are small) we find
  \[
  \frac{-1}{2}\left(|K|^{1/2}\cdot c\right)^2 = \frac{-1}{2}\left(|K|^{1/2}\cdot a\right)^2+\frac{-1}{2}\left(|K|^{1/2}\cdot b\right)^2.
  \]
  Multiplying both sides of the equation by $-2\cdot |K|$, we see that for a
  ``small'' right triangle on the surface of the hyperboloid,
  \[
  c^2 \approx a^2 +b^2.
  \]
\end{freeResponse}
\end{problem}


\begin{problem}
Summarize the results from this section. In particular, indicate which
results follow from the others.
\begin{freeResponse}
\end{freeResponse}
\end{problem}







\end{document}
