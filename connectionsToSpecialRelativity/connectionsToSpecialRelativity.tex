\documentclass[12pt,noauthor,nooutcomes,handout,newpage]{ximera}

\author{Bart Snapp}

%\usepackage{microtype}
%\usepackage{tikz}
\usepackage{tkz-euclide}
%\usetkzobj{all}
\tikzstyle geometryDiagrams=[rounded corners=.5pt,ultra thick,color=blue!50!black]

\usepackage{tikz-cd}

\colorlet{penColor}{blue!50!black} % Color of a curve in a plot

%% \hypersetup{
%%     colorlinks = false,
%%     }


\tikzset{%% partial ellipse
    partial ellipse/.style args={#1:#2:#3}{
        insert path={+ (#1:#3) arc (#1:#2:#3)}
    }
}

\graphicspath{
{./}
{sphericalLunesAndTriangles/}
{hyperbolicLunesAndTriangles/}
{centralProjection/}
{stereographicProjection/}
{linesAnglesAndAreasInCentralProjection/}
{linesAnglesAndAreasInStereographicProjection/}
{stereographicProjection/}
{centralProjectionInHG/}
{stereographicProjectionInHG/}
{linesInSphericalGeometry/}
{linesInHyperbolicGeometry/}
{theArtOfEscher/}
}


\newcommand{\transpose}{\intercal}
\newcommand{\eval}[1]{\bigg[ #1 \bigg]}

\renewcommand{\epsilon}{\varepsilon}
\renewcommand{\l}{\ell}
\renewcommand{\d}{\,d}

\DeclareMathOperator{\arccosh}{arccosh}
\DeclareMathOperator{\arctanh}{arctanh}
\renewcommand{\tilde}{\widetilde}
\newcommand{\R}{\mathbb R}
\newcommand{\dd}[2][]{\frac{d #1}{d #2}}
\newcommand{\pp}[2][]{\frac{\partial #1}{\partial #2}}
\newcommand{\dfn}{\textbf}

\renewcommand{\bar}{\overline}
\renewcommand{\hat}{\widehat}


\ifxake
\NewEnviron{freeResponse}{}
\fi


\title{Connections to special relativity}

\begin{document}
\begin{abstract}
Here we see a connection to special relativity.
\end{abstract}
\maketitle

In this section we will show a connection to special
relativity. Underlying all of this work is the notion of a
\textit{group action}.

\begin{definition}
  If $G$ is a group and $X$ is a set, then a (left) \dfn{group
    action} is a map
  \begin{align*}
    \_\cdot\_: G\times X &\to X\\
    (g,x) &\mapsto g\cdot x
  \end{align*}
  written as $g\cdot x \in X$ such that
  \begin{enumerate}
  \item $(g_2\cdot g_1)\cdot x = g_2\cdot (g_1\cdot x)$, for all $g_1,
    g_2\in G$ and $x\in X$.
  \item $I\cdot x = x$ for all $x\in X$ where $I$ is the identity
    element of the group.
  \end{enumerate}
\end{definition}
In essence, a group action tells you how to ``multiply'' elements of a
group by elements of a set, and obtain an element of the set as a
result of the multiplication. As we work, you should try to see where
we are using group actions.

\subsection{Galilean transformations of velocity}



In this case we will let points in central projection coordinates
$(x_c,y_c)$ represent ``observed velocities.'' We have shown that
orthogonal matrices in $K$-warped space form a group. That is we have
shown that
\begin{enumerate}
\item multiplication of orthogonal matrices is associative, 
\item the product of two orthogonal matrices is orthogonal,
\item the identity matrix is orthogonal,
\item the inverse matrix $M^{-1}$ of a orthogonal matrix $M$ is orthogonal.
\end{enumerate}
The group of orthogonal $3\times 3$ matrices is denoted as
$O(3)$. We'll denote the group of $K$-orthogonal matrices as
$O_K(3)$. When $K<0$, we already know three (classes of) elements of this group:
\[
I =
\begin{bmatrix}
  1 & 0 & 0\\
  0 & 1 & 0\\
  0 & 0 & 1
\end{bmatrix},
\qquad
M_\theta=
\begin{bmatrix}
  \cos\theta & -\sin\theta & 0\\
  \sin\theta & \cos\theta & 0\\
  0 & 0 & 1
\end{bmatrix},
\]
\[
N_\psi=\begin{bmatrix}
\cosh\psi & 0 & |K|^{-1/2}\cdot\sinh\psi\\
0 & 1 & 0\\
|K|^{1/2}\cdot\sinh\psi & 0 & \cosh\psi
\end{bmatrix}
\]
If we have an observer at point $O = (0,0)$ in central projection
coordinates, we can denote any other observed velocity in the plane as
a point $(x_c,y_c)$.

\begin{problem}
  With the setting described above in mind, what does the point
  $(.3,.3)$ represent?
\end{problem}

Now if we have a configuration of points in this velocity space, we
would like to answer the following question:

\begin{quote}
  If a new observer $O'$ views the original observer $O$ as moving at
  a velocity of $(u,0)$, how do we transform all other velocities
  measured by the original observer?
\end{quote}



\begin{problem}
  Can you give a naive solution to this problem?
\end{problem}


Even if you can give a naive solution to this problem, we will think
of this problem in central projection coordinates. Eventually we want
to solve a much harder problem where the naive solution will fail,
however the solution using central projection coordinates will
prevail. 

We'll start by working when $K<0$, then we will take the limit as $K$
goes to zero, and see what we find.

\begin{problem}
  Convert the $K$-rigid motion
  \[
  N_\psi=\begin{bmatrix}
  \cosh\psi & 0 & |K|^{-1/2}\cdot\sinh\psi\\
  0 & 1 & 0\\
  |K|^{1/2}\cdot\sinh\psi & 0 & \cosh\psi
\end{bmatrix}
  \]
  to a rigid motion of the $(x_c,y_c)$-plane. Call your answer $\nu_\psi$.
  \begin{hint}
    Either write down the answer from a previous exercise or recall
    that a rigid motion in $(x,y,z)$-coordinates
    \[
    M=\begin{bmatrix}
    m_{11} & m_{12} & m_{13}\\
    m_{21} & m_{22} & m_{23}\\
    m_{31} & m_{32} & m_{33}
    \end{bmatrix}
    \]
    converts to
    \[
    \mu(x_c,y_c) = \left(
    \frac{x_c\cdot m_{11} + y_c\cdot m_{21} + m_{31}}{x_c\cdot m_{13} + y_c\cdot m_{23} + m_{33}},
    \frac{x_c\cdot m_{12} + y_c\cdot m_{22} + m_{32}}{x_c\cdot m_{13} + y_c\cdot m_{23} + m_{33}}
    \right).
    \]
  \end{hint}
  \begin{freeResponse}
    Write
    \[
    \nu_\psi(x_c,y_c) = \left(\frac{x_c\cdot\cosh\psi+|K|^{-1/2}\cdot\sinh\psi}{x_c\cdot|K|^{1/2}\cdot\sinh\psi+\cosh\psi},\frac{y_c}{x_c\cdot|K|^{1/2}\cdot\sinh\psi+\cosh\psi}\right).
    \]
  \end{freeResponse}
\end{problem}

We need to be translating velocities by a constant amount in central
projection. To do this, we must set $\psi =\arctanh\left(u\cdot|K|^{1/2}\right)$.

\begin{problem}
  Substitute $\arctanh\left(u\cdot|K|^{1/2}\right)$ for $\psi$ in your formula above.
  \begin{hint}
    \begin{align*}
    \cosh(\arctanh(x)) &= \frac{1}{\sqrt{1-x^2}},\\
    \sinh(\arctanh(x)) &= \frac{x}{\sqrt{1-x^2}}.
    \end{align*}
  \end{hint}
  \begin{freeResponse}
    Write
    \begin{align*}
      \nu_\psi(x_c,y_c) &= \left(\frac{x_c\cdot\cosh\psi+|K|^{-1/2}\cdot\sinh\psi}{x_c\cdot|K|^{1/2}\cdot\sinh\psi+\cosh\psi},\frac{y_c}{x_c\cdot|K|^{1/2}\cdot\sinh\psi+\cosh\psi}\right)\\
      &= \left(\frac{\frac{x_c}{\sqrt{1-\left(u\cdot|K|^{1/2}\right)^2}} + \frac{|K|^{-1/2}\cdot u\cdot|K|^{1/2}}{\sqrt{1-\left(u\cdot|K|^{1/2}\right)^2}}}
         {\frac{x_c\cdot |K|^{1/2}\cdot u \cdot |K|^{1/2}}{\sqrt{1-\left(u\cdot|K|^{1/2}\right)^2}}+\frac{1}{\sqrt{1-\left(u\cdot |K|^{1/2}\right)^2}}},
         \frac{y_c}{\frac{x_c\cdot |K|^{1/2}\cdot u \cdot |K|^{1/2}}{\sqrt{1-\left(u\cdot|K|^{1/2}\right)^2}}+\frac{1}{\sqrt{1-\left(u\cdot |K|^{1/2}\right)^2}}}\right)\\
         &= \left(\frac{\frac{x_c}{\sqrt{1-u^2\cdot|K|}} + \frac{u}{\sqrt{1-u^2\cdot|K|}}}
         {\frac{x_c\cdot u \cdot |K|}{\sqrt{1-u^2\cdot|K|}}+\frac{1}{\sqrt{1-u^2\cdot |K|}}},
         \frac{y_c}{\frac{x_c\cdot u \cdot |K|}{\sqrt{1-u^2\cdot|K|}}+\frac{1}{\sqrt{1-u^2\cdot |K|}}}\right).
         %&= \left(\frac{x_c + u}{x_c\cdot u \cdot |K|+1},\frac{y_c\cdot \sqrt{1-u^2\cdot|K|}}{x_c\cdot u \cdot |K|+1}\right).
    \end{align*}
  \end{freeResponse}
\end{problem}

You have just written down a transformation that will work for any
negative value of $K$. We need this to work for $K=0$.

\begin{problem}
  Take the limit as $K$ goes to zero in your answer above.
  \begin{freeResponse}
    Write
    \[
    \lim_{K\to 0} \left(\frac{\frac{x_c}{\sqrt{1-u^2\cdot|K|}} + \frac{u}{\sqrt{1-u^2\cdot|K|}}}
         {\frac{x_c\cdot u \cdot |K|}{\sqrt{1-u^2\cdot|K|}}+\frac{1}{\sqrt{1-u^2\cdot |K|}}},
         \frac{y_c}{\frac{x_c\cdot u \cdot |K|}{\sqrt{1-u^2\cdot|K|}}+\frac{1}{\sqrt{1-u^2\cdot |K|}}}\right)
         \]
         \begin{align*}
           &= \lim_{K\to 0}\left(\frac{x_c + u}{x_c\cdot u \cdot |K|+1},\frac{y_c\cdot \sqrt{1-u^2\cdot|K|}}{x_c\cdot u \cdot |K|+1}\right)\\
           &=(x_c+u,y_c).
         \end{align*}
  \end{freeResponse}
\end{problem}


At this point we are working in the \textit{euclidean plane}, that is,
central projection when $K=0$.
The answer to our question:
\begin{quote}
  If a new observer $O'$ views the original observer $O$ as moving at
  a velocity of $(u,0)$, how do we transform all other velocities
  measured by the original observer?
\end{quote}
is almost too easy, as it is that we apply the transformation
\[
\nu_u(x_c,y_c) = (x_c+u,y_c).
\]
Before you object that we have done nothing but ``made mathematics
difficult'' we preemptively retort:
\begin{center}
  \textbf{We are building a foundation for future work.}
\end{center}

\begin{problem}
  Can you rephrase what we are doing in terms of group actions?
\end{problem}




\subsection{Hyperbolic transformations of velocity}

Armed with experience from above, let's now suppose we are again attacking the question:
\begin{quote}
  If a new observer $O'$ views the original observer $O$ as moving at
  a velocity of $(u,0)$, how do we transform all other velocities
  measured by the original observer?
\end{quote}
however, this time we are working not on the euclidean plane, but on
the Klein disk, when $K=-1$. The beauty of our work above is that it
can be directly applied in this case:

\begin{problem}
  Show that when $u\in [0,1)$ and $K=-1$,
  \[
  \nu_u(x_c,y_c) = \left(\frac{x_c + u}{x_c\cdot u + 1},\frac{y_c\cdot \sqrt{1-u^2}}{x_c\cdot u+1}\right).
  \]
  \begin{freeResponse}
    Write
    \begin{align*}
    \nu_u(x_c,y_c)&=\left(\frac{\frac{x_c}{\sqrt{1-u^2\cdot|K|}} + \frac{u}{\sqrt{1-u^2\cdot|K|}}}
         {\frac{x_c\cdot u \cdot |K|}{\sqrt{1-u^2\cdot|K|}}+\frac{1}{\sqrt{1-u^2\cdot |K|}}},
         \frac{y_c}{\frac{x_c\cdot u \cdot |K|}{\sqrt{1-u^2\cdot|K|}}+\frac{1}{\sqrt{1-u^2\cdot |K|}}}\right)\\
         &=\left(\frac{x_c + u}{x_c\cdot u \cdot |K|+1},\frac{y_c\cdot \sqrt{1-u^2\cdot|K|}}{x_c\cdot u \cdot |K|+1}\right)
    \end{align*}
    and set $K=-1$ to find:
    \[
    \nu_u(x_c,y_c)=\left(\frac{x_c + u}{x_c\cdot u + 1},\frac{y_c\cdot \sqrt{1-u^2}}{x_c\cdot u+1}\right).
    \]
  \end{freeResponse}
\end{problem}

For those who are in-the-know, a minor miracle has just occurred. The transformation
\[
  \nu_u(x_c,y_c) = \left(\frac{x_c + u}{x_c\cdot u + 1},\frac{y_c\cdot \sqrt{1-u^2}}{x_c\cdot u+1}\right),
\]
is in fact the velocity transformation for Einstein's special theory
of relativity when one takes the speed of light to be $c=1$!
%% \begin{problem}
%% Show that the transformation of the Klein disk $\nu_u$ is
%% given by the transformation of $K$-warped space
%% \[
%% N_u = 
%% \begin{bmatrix}
%%   \frac{1}{\sqrt{1-u^2\cdot|K|}} & 0 & \frac{u\cdot|K|}{\sqrt{1-u^2\cdot|K|}}\\
%%   0 & 1 & 0\\
%%   \frac{u}{\sqrt{1-u^2\cdot|K|}} & 0 & \frac{1}{\sqrt{1-u^2\cdot|K|}}
%% \end{bmatrix}
%% \]
%% when $K=-1$.
%% \end{problem}

%% \begin{problem}
%%   Show that
%%   \[
%%   N_u = 
%%   \begin{bmatrix}
%%     \frac{1}{\sqrt{1-u^2\cdot|K|}} & 0 & \frac{u\cdot|K|}{\sqrt{1-u^2\cdot|K|}}\\
%%     0 & 1 & 0\\
%%     \frac{u}{\sqrt{1-u^2\cdot|K|}} & 0 & \frac{1}{\sqrt{1-u^2\cdot|K|}}
%%   \end{bmatrix}
%%   \]
%%   is a $K$-rigid motion.
%% \end{problem}
The upshot to all of these computations is that transformations of the form:
\[
\nu_u(x_c,y_c) = \left(\frac{x_c + u}{x_c\cdot u + 1},\frac{y_c\cdot \sqrt{1-u^2}}{x_c\cdot u+1}\right),
\]
correspond exactly to a rigid motion of the surface
\[
-(x^2+y^2) + z^2 =1
\]
with the $K$-dot product.


One imaginative interpretation of these computations is that we are in
fact living on a hyperbolic surface, and that we experience our world
through the lens of a central projection.
%% In our experience, the ``visual'' velocity of an object should be a
%% straight vector. It's magnitude coincides with the euclidean force
%% required to accelerate the object to that speed.  This applied force
%% can be viewed as an applied rigid motion that transforms velocity as
%% we have seen above. We find special relativity confusing because our
%% world looks euclidean to us, yet we find velcity transforms in a
%% strange way.
%% Perhaps we are living on the hyperbolic surface, experiencing velocity as
%% Moreover, if we imagine the path of an object in space, the velocity
%% vectors look like euclidean vectors, but they are not measured using a
%% euclidean metric. When when we measure velocity vectors, we must use
%% the $K$-dot product, even though we are witnessing a projection. Let's
%% repeat the last bit, this means that tangent vectors are measured
%% using the $K$-dot product. And as we know by now,
%% \begin{center}
%%   \textbf{The dot product RULES!}
%% \end{center}
%% The conclusion is as inescapble as the Klein disk, we live in
%% hyprbolic geometry.
What an amazing thought!


%% \subsection{Hyperbolic transformations of position}


%% THIS IS A DRAFT


%% Here we must set
%% \[
%% K = \frac{-1}{T^2}\qquad \text{or}\qquad K = \frac{-1}{H+T^2}.
%% \]
%% Where $T$ is another time parameter different from $t$.
%% Computationally $T=t$ and $K= \frac{-1}{t^2}$ seems to work out, but
%% it makes very little physical sense to me what this means, or why this
%% should be the case.

%% Write
%% \[
%% N_{u\cdot t} = 
%% \begin{bmatrix}
%%   \frac{1}{\sqrt{1-u^2\cdot t^2\cdot|K|}} & 0 & \frac{u\cdot t^2\cdot|K|}{\sqrt{1-u^2\cdot t^2\cdot|K|}}\\
%%   0 & 1 & 0\\
%%   \frac{u\cdot t^2}{\sqrt{1-u^2\cdot t^2\cdot|K|}} & 0 & \frac{1}{\sqrt{1-u^2\cdot t^2\cdot|K|}}
%% \end{bmatrix}
%% \]
%% If $K=\frac{-1}{t^2}$, then 
%% \[
%% N_{u\cdot t} = 
%% \begin{bmatrix}
%%   \frac{1}{\sqrt{1-u^2}} & 0 & \frac{u}{t\cdot\sqrt{1-u^2}}\\
%%   0 & 1 & 0\\
%%   \frac{u\cdot t}{\sqrt{1-u^2}} & 0 & \frac{1}{\sqrt{1-u^2}}
%% \end{bmatrix}
%% \]
%% and this is a $K$-rigid motion in this case. Moreover, here
%% \[
%% \nu_{u\cdot t}(v\cdot t,0) = \left(\frac{(v+u)\cdot t}{\sqrt{1+v\cdot u}},0\right)  
%% \]









\begin{problem}
Summarize the results from this section. In particular, indicate which
results follow from the others.
\begin{freeResponse}
\end{freeResponse}
\end{problem}




\end{document}
