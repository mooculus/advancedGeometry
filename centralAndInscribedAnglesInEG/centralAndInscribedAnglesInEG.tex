\documentclass[newpage,hints,handout]{ximera}

%\usepackage{microtype}
%\usepackage{tikz}
\usepackage{tkz-euclide}
%\usetkzobj{all}
\tikzstyle geometryDiagrams=[rounded corners=.5pt,ultra thick,color=blue!50!black]

\usepackage{tikz-cd}

\colorlet{penColor}{blue!50!black} % Color of a curve in a plot

%% \hypersetup{
%%     colorlinks = false,
%%     }


\tikzset{%% partial ellipse
    partial ellipse/.style args={#1:#2:#3}{
        insert path={+ (#1:#3) arc (#1:#2:#3)}
    }
}

\graphicspath{
{./}
{sphericalLunesAndTriangles/}
{hyperbolicLunesAndTriangles/}
{centralProjection/}
{stereographicProjection/}
{linesAnglesAndAreasInCentralProjection/}
{linesAnglesAndAreasInStereographicProjection/}
{stereographicProjection/}
{centralProjectionInHG/}
{stereographicProjectionInHG/}
{linesInSphericalGeometry/}
{linesInHyperbolicGeometry/}
{theArtOfEscher/}
}


\newcommand{\transpose}{\intercal}
\newcommand{\eval}[1]{\bigg[ #1 \bigg]}

\renewcommand{\epsilon}{\varepsilon}
\renewcommand{\l}{\ell}
\renewcommand{\d}{\,d}

\DeclareMathOperator{\arccosh}{arccosh}
\DeclareMathOperator{\arctanh}{arctanh}
\renewcommand{\tilde}{\widetilde}
\newcommand{\R}{\mathbb R}
\newcommand{\dd}[2][]{\frac{d #1}{d #2}}
\newcommand{\pp}[2][]{\frac{\partial #1}{\partial #2}}
\newcommand{\dfn}{\textbf}

\renewcommand{\bar}{\overline}
\renewcommand{\hat}{\widehat}


\ifxake
\NewEnviron{freeResponse}{}
\fi


%% \prerequisites{EuclideanGeometry}
%% \outcome{circles}

\title{Central and inscribed angles in Euclidean geometry}

\begin{document}
\begin{abstract}
In this activity, we study central and inscribed angles in Euclidean
geometry.
\end{abstract}
\maketitle

\begin{listOutcomes}
 \item Prove that the  measure of any angle inscribed in a circle is one-half of the measure of the corresponding central angle.
 \item State and prove the extended Law of Sines for Euclidean geometry.
\end{listOutcomes}

\section{A basic fact}
Now we prove a basic fact about isosceles triangles.


\begin{problem}
Prove that given an isosceles triangle, angles opposite the congruent
sides are also congruent.
\begin{hint}
Use a congruence theorem.
\end{hint}
\begin{freeResponse}
Consider an isosceles triangle and construct a median to the
``nonequal'' side.
\begin{image}
\begin{tikzpicture}[geometryDiagrams]
\coordinate (A) at (0,0);
\coordinate (B) at (2,0);
\coordinate (C) at (1,3);
\coordinate (D) at (1,0);
\draw (A)--(B)--(C)--cycle;
\draw[thin] (C)--(D);
\tkzMarkSegments[mark=|](A,C B,C)
\tkzMarkSegments[mark=||](A,D B,D)
\tkzLabelPoints[above](C)
\tkzLabelPoints[below](A,D,B)
%\draw[step=.5cm] (0,0) grid (10,5);
\end{tikzpicture}
\end{image}
By SSS, we see that triangle $\triangle ADC \cong \triangle
BDC$. Hence $\angle DAC \cong \angle DBC$.
\end{freeResponse}
\end{problem}


\section{Central and inscribed angles}

The next topic in Euclidean geometry concerns central and inscribed
angles in circles. We include this partly for its own interest, and
partly because the properties we visit here will be useful later on.

\begin{problem}
On the circle with center $O$ below,
\begin{image}
\begin{tikzpicture}[geometryDiagrams]
\coordinate (O) at (0,0);
\coordinate (A) at ({2*cos(40)},{2*sin(40)});
\coordinate (B) at ({2*cos(350)},{2*sin(350)});
\coordinate (X) at ({2*cos(170)},{2*sin(170)});

\draw[thin] (A)--(O);
\draw[thin] (A)--(X);
\draw[thin] (X)--(B);
\draw (0,0) circle (2cm);

\tkzLabelPoints[above](O)
\tkzLabelPoints[left](X)
\tkzLabelPoints[right](A,B)

\end{tikzpicture}
\end{image}
show that%
\[
\angle BXA=(1/2)(\angle BOA).
\]
\begin{hint}
$\triangle OAX$ is isosceles.
\end{hint}
\begin{freeResponse}
Since all radii of a circle are equal in length, we have the following
diagram with an isosceles triangle. 
\begin{image}
\begin{tikzpicture}[geometryDiagrams]
\coordinate (O) at (0,0);
\coordinate (A) at ({2*cos(40)},{2*sin(40)});
\coordinate (B) at ({2*cos(350)},{2*sin(350)});
\coordinate (X) at ({2*cos(170)},{2*sin(170)});

\draw[thin] (A)--(O);
\draw[thin] (A)--(X);
\draw[thin] (X)--(B);
\draw (0,0) circle (2cm);

\tkzLabelPoints[above](O)
\tkzLabelPoints[left](X)
\tkzLabelPoints[right](A,B)
\tkzMarkSegments[mark=|](X,O O,A)
\end{tikzpicture}
\end{image}
Hence by our previous result, $\angle AXO \cong \angle OAX$. From this we see that 
\begin{align*}
\angle OXA + \angle XAO + \angle AOX &= \angle BOA + \angle AOX,\\
2\cdot \angle OXA &= \angle BOA.
\end{align*}
Since $\angle OXA = \angle BXA$, we have shown $\angle BXA =
(1/2)\angle BOA$.
\end{freeResponse}
\end{problem}



\begin{problem}
On the circle with center $O$ below,
\begin{image}
\begin{tikzpicture}[geometryDiagrams]
\coordinate (O) at (0,0);
\coordinate (A) at ({2*cos(40)},{2*sin(40)});
\coordinate (B) at ({2*cos(290)},{2*sin(290)});
\coordinate (X) at ({2*cos(170)},{2*sin(170)});

\draw[thin] (A)--(O);
\draw[thin] (A)--(X);
\draw[thin] (X)--(B);
\draw[thin] (O)--(B);
\draw (0,0) circle (2cm);

\tkzLabelPoints[above](O)
\tkzLabelPoints[left](X)
\tkzLabelPoints[right](A)
\tkzLabelPoints[below](B)

\end{tikzpicture}
\end{image}
show that%
\[
\angle BXA=(1/2)(\angle BOA).
\]

\begin{hint}
Draw the diameter through $O$ and $X$ and add.
\end{hint}

\begin{freeResponse}
Construct the diameter through $O$ and $X$:
\begin{image}
\begin{tikzpicture}[geometryDiagrams]
\coordinate (O) at (0,0);
\coordinate (A) at ({2*cos(40)},{2*sin(40)});
\coordinate (B) at ({2*cos(290)},{2*sin(290)});
\coordinate (X) at ({2*cos(170)},{2*sin(170)});
\coordinate (Y) at ({2*cos(350)},{2*sin(350)});

\draw[thin] (A)--(O);
\draw[thin] (A)--(X);
\draw[thin] (X)--(B);
\draw[thin] (O)--(B);
\draw[thin] (X)--(Y);
\draw (0,0) circle (2cm);

\tkzLabelPoints[above](O)
\tkzLabelPoints[left](X)
\tkzLabelPoints[right](Y)
\tkzLabelPoints[right](A)
\tkzLabelPoints[below](B)

\end{tikzpicture}
\end{image}
Now by our previous problem
\[
\angle OXA = (1/2)(\angle YOA) \qquad and \qquad \angle OXB = (1/2)(\angle BOY).
\]
However, 
\[
\angle OXA + \angle OXB = \angle BXA
\]
and 
\begin{align*}
(1/2)(\angle YOA) + (1/2)(\angle BOY) &= (1/2)(\angle YOA + \angle BOY)\\ 
&= (1/2)(\angle BOA).
\end{align*}
Hence $\angle BXA=(1/2)(\angle BOA)$.
\end{freeResponse}
\end{problem}



\begin{problem}
On the circle with center $O$ below,
\begin{image}
\begin{tikzpicture}[geometryDiagrams]
\coordinate (O) at (0,0);
\coordinate (A) at ({2*cos(60)},{2*sin(60)});
\coordinate (B) at ({2*cos(20)},{2*sin(20)});
\coordinate (X) at ({2*cos(170)},{2*sin(170)});

\draw[thin] (A)--(O);
\draw[thin] (A)--(X);
\draw[thin] (X)--(B);
\draw[thin] (O)--(B);
\draw (0,0) circle (2cm);

\tkzLabelPoints[below](O)
\tkzLabelPoints[left](X)
\tkzLabelPoints[above](A)
\tkzLabelPoints[right](B)

\end{tikzpicture}
\end{image}
show that
\[
\angle BXA=(1/2)(\angle BOA).
\]

\begin{hint}
Draw the diameter through $O$ and $X$ and subtract.
\end{hint}
\begin{freeResponse}
Construct the diameter through $O$ and $X$:
\begin{image}
\begin{tikzpicture}[geometryDiagrams]
\coordinate (O) at (0,0);
\coordinate (A) at ({2*cos(60)},{2*sin(60)});
\coordinate (B) at ({2*cos(20)},{2*sin(20)});
\coordinate (X) at ({2*cos(170)},{2*sin(170)});
\coordinate (Y) at ({2*cos(350)},{2*sin(350)});

\draw[thin] (A)--(O);
\draw[thin] (A)--(X);
\draw[thin] (X)--(B);
\draw[thin] (O)--(B);
\draw[thin] (X)--(Y);
\draw (0,0) circle (2cm);

\tkzLabelPoints[below](O)
\tkzLabelPoints[left](X)
\tkzLabelPoints[above](A)
\tkzLabelPoints[right](B,Y)
\end{tikzpicture}
\end{image}

Now by our previous problem
\[
\angle OXA = (1/2)(\angle YOA) \qquad and \qquad \angle OXB = (1/2)(\angle YOB).
\]
However, 
\[
\angle OXA - \angle OXB = \angle BXA
\]
and 
\begin{align*}
(1/2)(\angle YOA) - (1/2)(\angle YOB) &= (1/2)(\angle YOA + \angle YOB)\\ 
&= (1/2)(\angle BOA).
\end{align*}
Hence $\angle BXA=(1/2)(\angle BOA)$.
\end{freeResponse}
\end{problem}



We can summarize the results of the last three problems into the following theorem.

\begin{theorem}
\label{43}The measure of any angle inscribed in a circle is one-half of the
measure of the corresponding central angle.
\end{theorem}

\begin{definition}
 An \emph{inscribed angle} is formed by two chords of a circle, while a \emph{central angle} is formed by two radii.
\end{definition}


\section{The Law of Sines}


Recall the (extended) Law of Sines:

\begin{theorem}[Law of Sines]
Given a triangle with angles $\alpha$, $\beta$, and $\gamma$, with
side $a$ opposite $\alpha$, side $b$ opposite $\beta$, and side $c$
opposite $\gamma$, we have
\[
\frac{a}{\sin(\alpha)} = \frac{b}{\sin(\beta)} = \frac{c}{\sin(\gamma)} = 2R  
\]
where $R$ is the radius of a circle circumscribing the triangle.
\end{theorem}

To prove this theorem, we must check several things. 

\begin{definition}
The \textbf{midset} of two given points is the set of points
equidistant from the given points.
\end{definition}

\begin{problem}
Show that in Euclidean geometry, the midset of two points is given by the
perpendicular bisector of the segment connecting the two given points.

\begin{hint}
 Find the midpoint of the segment connecting the two given points. Draw two congruent triangles, each with one side on the segment connecting the two points and one vertex on their midpoint.
\end{hint}
\begin{freeResponse}
Consider the perpendicular bisector of a segment $\bar{AB}$.
\begin{image}
IMAGE HERE
\end{image}
Constructing a triangle to any point (say $C$) on the perpendicular
bisector, we see by SAS that $|AC|=|BC|$.
\end{freeResponse}
\end{problem}


\begin{problem}
Explain why every triangle can be circumscribed by a circle.
\begin{hint}
Consider the intersection of the perpendicular bisectors of the sides.
\end{hint}
\end{problem}


\begin{problem}
Prove the Law of Sines.
\begin{hint}
Start by circumscribing the triangle. 
\end{hint}
\begin{hint}
Next, construct two (congruent) right triangles using the center of circle and two vertices of the triangle.
\end{hint}
\begin{hint}
Finally, use the definition of sine.
\end{hint}
\end{problem}




%% \begin{problem}
%% Use similar triangles and the previous problems to show that
%% \[
%% | AX| \cdot |XB| =| A'X|\cdot |XB'|
%% \]
%% in the figure below.%
%% \begin{image}
%% \begin{tikzpicture}[geometryDiagrams]
%% \coordinate (X) at (-.02,1.02);
%% \coordinate (B') at ({2*cos(60)},{2*sin(60)});
%% \coordinate (B) at ({2*cos(20)},{2*sin(20)});
%% \coordinate (A) at ({2*cos(140)},{2*sin(140)});
%% \coordinate (A') at ({2*cos(190)},{2*sin(190)});

%% \draw[thin] (A)--(B);
%% \draw[thin] (A')--(B');
%% \draw (0,0) circle (2cm);

%% \tkzLabelPoints[left](A,A')
%% \tkzLabelPoints[right](B)
%% \tkzLabelPoints[above](B')
%% \tkzLabelPoints[above](X)

%% \end{tikzpicture}
%% \end{image}
%% \begin{hint}
%% Draw $\overline{AB'}$ and $\overline{A'B}$.
%% \end{hint}

%% \begin{hint}
%% Draw the center of the circle and use the previous theorem.
%% \end{hint}

%% \begin{freeResponse}
%% Construct $\overline{AB'}$ and $\overline{A'B}$, and add the center of
%% the circle.
%% \begin{image}
%% \begin{tikzpicture}[geometryDiagrams]
%% \coordinate (O) at (0,0);
%% \coordinate (X) at (-.02,1.02);
%% \coordinate (B') at ({2*cos(60)},{2*sin(60)});
%% \coordinate (B) at ({2*cos(20)},{2*sin(20)});
%% \coordinate (A) at ({2*cos(140)},{2*sin(140)});
%% \coordinate (A') at ({2*cos(190)},{2*sin(190)});

%% \draw[thin] (A)--(B);
%% \draw[thin] (A')--(B');
%% \draw[thin] (A)--(B');
%% \draw[thin] (A')--(B);
%% \draw[thin,dashed] (O)--(B);
%% \draw[thin,dashed] (O)--(B');
%% \draw (0,0) circle (2cm);

%% \tkzLabelPoints[left](A,A')
%% \tkzLabelPoints[right](B)
%% \tkzLabelPoints[above](B')
%% \tkzLabelPoints[above](X)
%% \tkzLabelPoints[below](O)
%% \end{tikzpicture}
%% \end{image}
%% Since $\angle BAB'$ and $\angle BA'B'$ share the same central angle,
%% they are congruent. Moreover, decorating our diagram slightly differently, 
%% \begin{image}
%% \begin{tikzpicture}[geometryDiagrams]
%% \coordinate (O) at (0,0);
%% \coordinate (X) at (-.02,1.02);
%% \coordinate (B') at ({2*cos(60)},{2*sin(60)});
%% \coordinate (B) at ({2*cos(20)},{2*sin(20)});
%% \coordinate (A) at ({2*cos(140)},{2*sin(140)});
%% \coordinate (A') at ({2*cos(190)},{2*sin(190)});

%% \draw[thin] (A)--(B);
%% \draw[thin] (A')--(B');
%% \draw[thin] (A)--(B');
%% \draw[thin] (A')--(B);
%% \draw[thin,dashed] (O)--(A);
%% \draw[thin,dashed] (O)--(A');
%% \draw (0,0) circle (2cm);

%% \tkzLabelPoints[left](A,A')
%% \tkzLabelPoints[right](B)
%% \tkzLabelPoints[above](B')
%% \tkzLabelPoints[above](X)
%% \tkzLabelPoints[below](O)
%% \end{tikzpicture}
%% \end{image}
%% since $\angle ABA'$ and $\angle AB'A'$ share the same central angle,
%% they are congruent. Hence $\triangle AXB'\cong \triangle A'XB'$ are similar. 
%% This tells us that 
%% \[
%% \frac{|AX|}{|XB'|} = \frac{|A'X|}{|XB|}
%% \]
%% so we see that $|AX|\cdot|XB|=|A'X|\cdot|XB'|$.
%% \end{freeResponse}
%% \end{problem}


%% \begin{problem}
%% Use similar triangles and the previous problems to show that
%% \[
%% |AX|\cdot|XB|=|A'X| \cdot| XB'|
%% \]
%% in the figure below.%
%% \begin{image}
%% \begin{tikzpicture}[geometryDiagrams]
%% \coordinate (X) at (4.51,.22);
%% \coordinate (B') at ({2*cos(0)},{2*sin(0)});
%% \coordinate (B) at ({2*cos(20)},{2*sin(20)});
%% \coordinate (A) at ({2*cos(140)},{2*sin(140)});
%% \coordinate (A') at ({2*cos(190)},{2*sin(190)});

%% \draw[thin] (A)--(X);
%% \draw[thin] (A')--(X);
%% \draw (0,0) circle (2cm);

%% \tkzLabelPoints[left](A,A')
%% \tkzLabelPoints[above right](B)
%% \tkzLabelPoints[below right](B')
%% \tkzLabelPoints[right](X)

%% \end{tikzpicture}
%% \end{image}

%% \begin{hint}
%% Draw $\overline{AB^{\prime}}$ and $\overline{A^{\prime}B}$.
%% \end{hint}

%% \begin{hint}
%% Draw the center of the circle and use the previous theorem.
%% \end{hint}

%% \begin{freeResponse}
%% Construct $\overline{AB'}$ and $\overline{A'B}$, and add the center of
%% the circle.
%% \begin{image}
%% \begin{tikzpicture}[geometryDiagrams]
%% \coordinate (O) at (0,0);
%% \coordinate (X) at (4.51,.22);
%% \coordinate (B') at ({2*cos(0)},{2*sin(0)});
%% \coordinate (B) at ({2*cos(20)},{2*sin(20)});
%% \coordinate (A) at ({2*cos(140)},{2*sin(140)});
%% \coordinate (A') at ({2*cos(190)},{2*sin(190)});

%% \draw[thin] (A)--(X);
%% \draw[thin] (A')--(X);
%% \draw[thin] (A)--(B');
%% \draw[thin] (A')--(B);

%% \draw[thin,dashed] (O)--(B);
%% \draw[thin,dashed] (O)--(B');

%% \draw (0,0) circle (2cm);

%% \tkzLabelPoints[left](A,A')
%% \tkzLabelPoints[above right](B)
%% \tkzLabelPoints[below right](B')
%% \tkzLabelPoints[right](X)
%% \tkzLabelPoints[left](O)
%% \end{tikzpicture}
%% \end{image}
%% Since $\angle BAB'$ and $\angle BA'B'$ share the same central angle,
%% they are congruent. Moreover, decorating our diagram slightly differently, 
%% \begin{image}
%% \begin{tikzpicture}[geometryDiagrams]
%% \coordinate (O) at (0,0);
%% \coordinate (X) at (4.51,.22);
%% \coordinate (B') at ({2*cos(0)},{2*sin(0)});
%% \coordinate (B) at ({2*cos(20)},{2*sin(20)});
%% \coordinate (A) at ({2*cos(140)},{2*sin(140)});
%% \coordinate (A') at ({2*cos(190)},{2*sin(190)});

%% \draw[thin] (A)--(X);
%% \draw[thin] (A')--(X);
%% \draw[thin] (A)--(B');
%% \draw[thin] (A')--(B);

%% \draw[thin,dashed] (O)--(A);
%% \draw[thin,dashed] (O)--(A');

%% \draw (0,0) circle (2cm);

%% \tkzLabelPoints[left](A,A')
%% \tkzLabelPoints[above right](B)
%% \tkzLabelPoints[below right](B')
%% \tkzLabelPoints[right](X)
%% \tkzLabelPoints[right](O)
%% \end{tikzpicture}
%% \end{image}
%% since $\angle ABA'$ and $\angle AB'A'$ share the same central angle,
%% they are congruent. Hence $\triangle AXB'\cong \triangle A'XB'$ are similar. 
%% This tells us that 
%% \[
%% \frac{|AX|}{|XB'|} = \frac{|A'X|}{|XB|}
%% \]
%% so we see that $|AX|\cdot|XB|=|A'X|\cdot|XB'|$.
%% \end{freeResponse}
%% \end{problem}




\begin{problem}
Summarize the results from this section. In particular, indicate which
results follow from the others.
\begin{freeResponse}
\end{freeResponse}
\end{problem}

\end{document}
